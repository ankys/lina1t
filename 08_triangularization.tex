
\chapter{三角化}

\section{フロベニウスの定理とケイリー・ハミルトンの定理}

\begin{theorem}[フロベニウスの定理]
$A$を$K$上の$N$次正方行列、$f(x)$を$K$上の多項式とする。
このとき、$A$の固有値$c \in K$に対して、$f(c)$は$f(A)$の固有値である。
より詳しくは$A$の固有多項式が$c_1, \cdots, c_N \in K$を使って\eqref{e:eigenfactor}と因数分解されるならば、
$f(A)$の固有多項式は
$$
\det(x I-f(A)) = (x-f(c_1))\cdots(x-f(c_N))
$$
と因数分解される。
\end{theorem}

\begin{proof}
$c$を$A$の固有値とするとき、$A\vb*{v} = c\vb*{v}$となるベクトル$\vb*{v} \ne \vb*{0}$が取れる。
$n = 0, 1, 2, 3, \cdots$に対して、
$$
A^n\vb*{v} = c^n\vb*{v}
$$
であるからスカラー倍して和を取ることで、
$$
f(A)\vb*{v} = f(c)\vb*{v}
$$
である。
よって、$f(c)$は$f(A)$の固有値である。

$A$の固有多項式が\eqref{e:eigenfactor}と因数分解されるとき、定理\ref{t:tri}より\eqref{e:tri}と三角化される。
$n = 0, 1, 2, 3, \cdots$に対して、
$$
A^n
= P\mqty(c_1 & \cdots & * \\ & \ddots & \vdots \\ & & c_N)^n P^{-1}
= P\mqty(c_1^n & \cdots & * \\ & \ddots & \vdots \\ & & c_N^n)P^{-1}
$$
であるからスカラー倍して和を取ることで、
$$
f(A) = P\mqty(f(c_1) & \cdots & * \\ & \ddots & \vdots \\ & & f(c_N))P^{-1}
$$
である。
よって、命題\ref{t:eigenfactor}より、この定理の結論が従う。
\end{proof}

\begin{theorem}[ケイリー・ハミルトンの定理]
$A$を$K$上の$N$次正方行列として、固有多項式$f_A(x) = \det(x I-A)$が\eqref{e:eigenfactor}と因数分解されたとする。
このとき、$f_A(A) = O$が成り立つ。
\end{theorem}

\begin{remark}
この定理は$\det(x I-A)$の$x$に$A$を代入できると拡大解釈すれば$\det(A I-A) = \det O = 0$となり正しそうであるが、
実際の証明はしっかり$f_A(A)$の定義に則って行う必要がある。
\end{remark}

\begin{proof}
まず$A$が右上三角行列
$$
T = \mqty(c_1 & \cdots & * \\ & \ddots & \vdots \\ & & c_N)
$$
の場合に示す。
このとき固有多項式は
$$
f_T(x) = (x-c_1)\cdots(x-c_N)
$$
であり、
$$
f_T(T) = (T-c_1 I_N)\cdots(T-c_N I_N)
$$
が成り立つ。
あとはこれが零行列であることを$N$についての数学的帰納法で示す。
$N = 1$の時は$T = \mqty(c_1)$なので$T-c_1 I_1 = O_1$である。
$N-1$で成立する時、
$$
(T-c_1 I_N)\cdots(T-c_{N-1} I_N) = \mqty(O_{N-1} & * \\ \vb*{0}_{N-1} & *),
\quad (T-c_N I_N) = \mqty(* & * \\ \vb*{0}_{N-1} & 0)
$$
なので、積を取ると零行列になる。
以上より$f_T(T) = O$である。
一般の$A$に対しては定理\ref{t:tri}より、$A$は\eqref{e:tri}と三角化されて右上三角行列を$T$とおくと、
$f_A(x) = f_T(x) = (x-c_1)\cdots(x-c_N)$で
$$
f_A(A) = P f_A(T) P^{-1}
$$
より、$f_A(A) = O$がわかる。
\end{proof}

以上の二つの定理を使えば例えば以下のことがわかる。

\begin{proposition}[べき零行列]
代数的閉体$K$上の$N$次正方行列$A$がある$n = 1, 2, 3, \cdots$で
$$
A^n = O
$$
を満たす時、$A^N = O$が成り立つ。
\end{proposition}

このような行列のことを\emph{べき零行列}という。

\begin{proof}
$f(x) = x^n$とおいてフロベニウスの定理を用いると、$A$の固有値を$c \in K$とおくと$f(c) = c^n$は$f(A) = A^n$の固有値である。
ここで$A^n = O$の固有値は$0$しかないので、$c^n = 0$がわかり$A$の固有値は全て$0$であることがわかる。
よって、$A$の固有多項式は$f_A(x) = x^N$であり、ケイリー・ハミルトンの定理より$f_A(A) = A^N = O$が従う。
\end{proof}

\section{広義固有空間}

固有多項式の零点の重複度(代数的重複度)と固有空間の次数(幾何学的重複度)には差がある可能性があるのであった。
その差を埋めるにはどうすればいいだろうか。
そのためのアイデアが固有空間を拡張した広義固有空間である。

\begin{definition}[広義固有空間]
$A$を$K$上の正方行列として、$x \in K$に対して
$$
\tilde{W}(x) = \lrset{ \vb*{v} \in K^N \mid (x I-A)^n\vb*{v} = \vb*{0}, n = 0, 1, 2, 3, \cdots }
$$
を定め、固有値$x = c$に対して$\tilde{W}(c)$を$A$の固有値$c$に対する\emph{広義固有空間}という。
\end{definition}

$\tilde{W}(x)$は$K^N$の線形部分空間である。

\begin{lemma}[広義固有空間の線形独立性]
$T$を$K$上の線形空間$V$上の線形変換として、$c, d \in K$を異なる固有値とする。
このとき広義固有空間について$\tilde{W}(c)\cap \tilde{W}(d) = O_V$が成り立つ。
\end{lemma}

\begin{proof}
\end{proof}

\section{ジョルダン標準形}
