\documentclass{jsbook}
\usepackage{mathtools}
\usepackage{amsmath}
\usepackage{amssymb}
\usepackage{amsthm}
\usepackage{physics}
\usepackage{makeidx}
\usepackage{mdframed}

% \usepackage{comment}
% \usepackage{lineno}
% \pagewiselinenumbers
% \usepackage{showkeys}
% \newcommand{\todo}[1]{\marginpar{TODO: #1}}
\newcommand{\todo}[1]{}
\usepackage[dvipdfmx]{hyperref}
\usepackage{pxjahyper}

% \graphicspath{{fig/}}

\setlength{\textwidth}{\fullwidth}
\setlength{\evensidemargin}{\oddsidemargin}

% \newtheorem{theorem}{定理}[section]
\newmdtheoremenv[linecolor=blue]{theorem}{定理}[section]
\newmdtheoremenv[linecolor=blue]{lemma}[theorem]{補題}
\newmdtheoremenv[linecolor=blue]{proposition}[theorem]{命題}
\newmdtheoremenv[linecolor=blue]{corollary}[theorem]{系}

\theoremstyle{definition}
% \newtheorem{definition}[theorem]{定義}
\newmdtheoremenv[linecolor=blue]{definition}[theorem]{定義}
\newtheorem{example}[theorem]{例}
\newtheorem{xca}[theorem]{練習問題}

\theoremstyle{remark}
\newtheorem{remark}[theorem]{注意}

\renewcommand{\proofname}{証明}
\renewcommand{\qedsymbol}{【証明終わり】}
% \makeatletter
% \AtBeginEnvironment{proof}{\let\@addpunct\@gobble}
% \makeatother
\makeatletter
\renewenvironment{proof}[1][\proofname]{\par
  \pushQED{\qed}%
  \normalfont \topsep6\p@\@plus6\p@\relax
  \trivlist
  \item\relax
  {\itshape
  【#1】}\hspace\labelsep\ignorespaces
}{%
  \popQED\endtrivlist\@endpefalse
}
\makeatother

\numberwithin{equation}{section}

\def\M{\mathrm{M}}
\def\GL{\mathrm{GL}}
\def\id{\mathrm{id}}
\def\Id{\mathrm{Id}}

\DeclareMathOperator{\sgn}{sgn}
\DeclareMathOperator{\Ker}{Ker}
\DeclareMathOperator{\Img}{Img}

\DeclarePairedDelimiter{\lrset}{\{}{\}}
\DeclarePairedDelimiter{\lrangle}{\langle}{\rangle}

\makeindex

\begin{document}

\title{線形代数学1}
\author{中安淳}
\date{\today}

\maketitle

% \input{00_preface.tex}

\tableofcontents


\chapter{二次の理論}

\section{連立一次方程式と逆行列}

$2$つの未知数からなる連立一次方程式
$$
\begin{cases}
x+y = 8, \\
2 x+4 y = 26 \\
\end{cases}
$$
を考えよう。
この問題は鶴と亀が合わせて$8$匹いて脚の数が合計して$26$本であるという、鶴亀算の典型的な例題として小学校で紹介され、いったん$8$匹すべてが鶴としてから徐々に亀に変えていって脚の数の帳尻を合わせるなどという解法を学習する。
その後、中学校で連立一次方程式を習うと鶴の数を$x$匹で亀の数を$y$匹とすることで上記の方程式に帰着され、加減法あるいは代入法などの解法を適用できることを学ぶ。
大学の教科書である本テキストではこの連立一次方程式を行列の理論、特に逆行列を使う解法を紹介する。

係数を一般化して連立一次方程式
$$
\begin{cases}
a x+b y = e, \\
c x+d y = f \\
\end{cases}
$$
を考える。
これを行列に結びつける第一歩は次のように書くことである。
$$
\begin{pmatrix}a x+b y \\ c x+d y\end{pmatrix} = \begin{pmatrix}e \\ f\end{pmatrix}.
$$
ここで両辺に現れる、$2$つの数を縦に並べて括弧でくくったものを($2$次の)(縦)\emph{ベクトル}という。
二つのベクトル$\begin{pmatrix}p \\ q\end{pmatrix}$と$\begin{pmatrix}u \\ v\end{pmatrix}$が等しいとは$p = u$かつ$q = v$が成り立つとすることで上記の変形が可能となる。
さらに左辺を次のように変形できたとしよう。
$$
\begin{pmatrix}a x+b y \\ c x+d y\end{pmatrix} = \begin{pmatrix}a & b \\ c & d\end{pmatrix}\begin{pmatrix}x \\ y\end{pmatrix}.
$$
この時の右辺に現れる$4$つの数$a, b, c, d$を$2\times 2$に並べたものを($2\times 2$型のあるいは$2$次正方)\emph{行列}といい、上記の式は行列とベクトルの積を定義する。
以上によって最初の連立一次方程式は行列とベクトルを使って
$$
\begin{pmatrix}a & b \\ c & d\end{pmatrix}\begin{pmatrix}x \\ y\end{pmatrix} = \begin{pmatrix}e \\ f\end{pmatrix}
$$
と表すことができる。
この時の行列を\emph{係数行列}といい、左辺のベクトルを\emph{未知数ベクトル}、右辺のベクトルを\emph{定数ベクトル}という。
係数行列、未知数ベクトル、定数ベクトルをそれぞれ$A$, $\vb*{x}$, $\vb*{b}$とおくと連立一次方程式は
$$
A \vb*{x} = \vb*{b}
$$
とあたかも未知数が$1$つの場合の一次方程式であるかのように書くことができる(現時点では書くことができるだけだが)。

以下では行列やベクトルでない実数や複素数などの数単体のことを\emph{スカラー}と呼ぶ。
スカラーの一次方程式は係数で両辺を割ればすぐに解けるのであった。
しかしながら今回の場合では行列で割るというのはどういうことだろうか。
それに答えるには、$0$でない実数や複素数には逆数があるように、行列に対して逆行列なるものを導入できればよいことに気がつく。
$2$次正方行列に対しては逆行列は以下のようにして得られる。

\begin{definition}[$2$次の逆行列]
$2$次正方行列$A = \begin{pmatrix}a & b \\ c & d\end{pmatrix}$が$a d-b c \ne 0$を満たすとき定義される次の行列
$$
A^{-1} = \frac{1}{a d-b c}\begin{pmatrix}d & -b \\ -c & a\end{pmatrix}
$$
を$A$の\emph{逆行列}という。
また、$a d-b c$のことを$2$次正方行列の\emph{行列式}といい、逆行列を持つ正方行列は\emph{可逆行列}または\emph{正則行列}と呼ばれる。
\end{definition}

この逆行列は次の性質を満たす。
ただし、二つの$2$次正方行列の\emph{積}を
$$
\begin{pmatrix}a & b \\ c & d\end{pmatrix}\begin{pmatrix}e & f \\ g & h\end{pmatrix}
= \begin{pmatrix}a e+b g & a f+b h \\ c e+d g & c f+d h\end{pmatrix}
$$
として定め、次の特殊な行列
$$
I = I_2 = \begin{pmatrix}1 & 0 \\ 0 & 1\end{pmatrix}
$$
を($2$次の)\emph{単位行列}という。
任意の$2$次正方行列$A$に対して、
$$
A I = I A = A
$$
が成り立つことに注意する。

\begin{proposition}[逆行列]
逆行列$A^{-1}$を持つ$2$次正方行列$A$に対して、
$$
A A^{-1} = A^{-1} A = I
$$
が成り立つ。
\end{proposition}

\begin{proof}
計算すると
$$
A A^{-1}
= \begin{pmatrix}a & b \\ c & d\end{pmatrix}\begin{pmatrix}\frac{d}{a d-b c} & \frac{-b}{a d-b c} \\ \frac{-c}{a d-b c} & \frac{a}{a d-b c}\end{pmatrix}
= \begin{pmatrix}\frac{a d-b c}{a d-b c} & \frac{-a b+b a}{a d-b c} \\ \frac{c d-d c}{a d-b c} & \frac{-c b+d a}{a d-b c}\end{pmatrix}
= I,
$$
$$
A^{-1} A
= \frac{1}{a d-b c}\begin{pmatrix}d & -b \\ -c & a\end{pmatrix}\begin{pmatrix}a & b \\ c & d\end{pmatrix}
= \frac{1}{a d-b c}\begin{pmatrix}d a-b c & d b-b d \\ -c a+a c & -c b+a d\end{pmatrix}
= I.
$$
\end{proof}

これにより連立一次方程式$A \vb*{x} = \vb*{b}$の両辺に係数行列$A$の逆行列$A^{-1}$を(左から)かけることで
$$
A^{-1} A \vb*{x} = \vb*{x} = A^{-1} \vb*{b}
$$
となり、$2$つの未知数からなる連立一次方程式は逆行列を作って定数ベクトルにかけることで解くことができる。

\begin{example}
鶴亀算の方程式
$$
\begin{cases}
x+y = 8, \\
2 x+4 y = 26 \\
\end{cases}
$$
は
$$
\begin{pmatrix}1 & 1 \\ 2 & 4\end{pmatrix}\begin{pmatrix}x \\ y\end{pmatrix}
= \begin{pmatrix}8 \\ 26\end{pmatrix}
$$
と表示でき、係数行列の逆行列は
$$
\begin{pmatrix}1 & 1 \\ 2 & 4\end{pmatrix}^{-1}
= \frac{1}{1\cdot 4-1\cdot 2}\begin{pmatrix}4 & -1 \\ -2 & 1\end{pmatrix}
= \frac{1}{2}\begin{pmatrix}4 & -1 \\ -2 & 1\end{pmatrix}
$$
なので、解は
$$
\begin{pmatrix}x \\ y\end{pmatrix}
= \frac{1}{2}\begin{pmatrix}4 & -1 \\ -2 & 1\end{pmatrix}\begin{pmatrix}8 \\ 26\end{pmatrix}
= \frac{1}{2}\begin{pmatrix}4\cdot 8+(-1)\cdot 26 \\ (-2)\cdot 8+1\cdot 26\end{pmatrix}
= \begin{pmatrix}3 \\ 5\end{pmatrix}
$$
と求まる。
\end{example}

\section{線形漸化式と対角化}

$n = 0, 1, 2, 3, \cdots$として、$2$つの数列$x_n$と$y_n$が二項間斉次線形漸化式
$$
\begin{cases}
x_{n+1} = a x_n+b y_n, & x_0 = e, \\
y_{n+1} = c x_n+d y_n, & y_0 = f \\
\end{cases}
$$
を満たす状況を考える。
例えば、$1$リットルと$2$リットルの液体が入った二つの容器があるとして、
一回の操作でそれぞれの容器から$4$割の液体を取り出しもう片方の容器に移すことを考える。
$n$回目の操作が終わった段階での容器の中の液体の量を$x_n$リットルと$y_n$リットルとすると、
$$
\begin{cases}
x_{n+1} = 0.6 x_n+0.4 y_n, & x_0 = 1, \\
y_{n+1} = 0.4 x_n+0.6 y_n, & y_0 = 2 \\
\end{cases}
$$
が成り立つ。
一般論に戻ると前節同様これは行列とベクトルを用いて
$$
\begin{pmatrix}x_{n+1} \\ y_{n+1}\end{pmatrix}
= \begin{pmatrix}a & b \\ c & d\end{pmatrix}\begin{pmatrix}x_n \\ y_n\end{pmatrix},
\quad \begin{pmatrix}x_0 \\ y_0\end{pmatrix} = \begin{pmatrix}e \\ f\end{pmatrix}
$$
と表される。
数列ベクトル、係数行列、初項ベクトルをそれぞれ$\vb*{x}_n$, $A$, $\vb*{b}$とおくと漸化式は
$$
\vb*{x}_{n+1} = A \vb*{x}_n, \quad \vb*{x}_0 = \vb*{b}
$$
と表現できる。
これは$n$を一つ増やすごとに$A$を左からかけているので、$2$次正方行列$A$の$n$乗$A^n$を単位行列$I$に$A$を$n$回かけたものとして定義することで、
$$
\vb*{x}_n = A^n \vb*{b}
$$
となる。
つまり二項間(斉次線形)漸化式を解くことは$A^n$を計算することに帰着される。

この行列の$n$乗などの計算をする際に重要になるのが対角化と呼ばれる手法である。
ここで対角とは($2$次)正方行列$A = \begin{pmatrix}a & b \\ c & d\end{pmatrix}$の左上から右下にかけての対角線上に位置する成分のことで、
対角成分以外の成分が全て$0$である正方行列を\emph{対角行列}という。
\emph{対角化}とは($2$次)正方行列$A$を対角行列$D = \begin{pmatrix}s & 0 \\ 0 & t\end{pmatrix}$と可逆行列$P$を使って
$$
A = P D P^{-1} = P\begin{pmatrix}s & 0 \\ 0 & t\end{pmatrix}P^{-1}
$$
と変形することである。
もし対角化ができたとしたら$n$乗は
$$
A^n = (P D P^{-1})^n = P D^n P^{-1} = P\begin{pmatrix}s^n & 0 \\ 0 & t^n\end{pmatrix}P^{-1}
$$
と容易に計算することができる。

\begin{example}
先述の液体の移し替えの漸化式
$$
\begin{cases}
x_{n+1} = 0.6 x_n+0.4 y_n, & x_0 = 1, \\
y_{n+1} = 0.4 x_n+0.6 y_n, & y_0 = 2 \\
\end{cases}
$$
は
$$
\begin{pmatrix}x_{n+1} \\ y_{n+1}\end{pmatrix}
= \begin{pmatrix}0.6 & 0.4 \\ 0.4 & 0.6\end{pmatrix}\begin{pmatrix}x_n \\ y_n\end{pmatrix},
\quad \begin{pmatrix}x_0 \\ y_0\end{pmatrix} = \begin{pmatrix}1 \\ 2\end{pmatrix}
$$
と表示でき、
詳細は省略するが係数行列は
$$
\begin{pmatrix}0.6 & 0.4 \\ 0.4 & 0.6\end{pmatrix}
= \begin{pmatrix}1 & 1 \\ 1 & -1\end{pmatrix}\begin{pmatrix}1 & 0 \\ 0 & 0.2\end{pmatrix}\begin{pmatrix}1 & 1 \\ 1 & -1\end{pmatrix}^{-1}
$$
と対角化される。
よって、この数列の一般項は
$$
\begin{pmatrix}x_{n+1} \\ y_{n+1}\end{pmatrix}
= \begin{pmatrix}0.6 & 0.4 \\ 0.4 & 0.6\end{pmatrix}^n\begin{pmatrix}1 \\ 2\end{pmatrix}
= \begin{pmatrix}1 & 1 \\ 1 & -1\end{pmatrix}\begin{pmatrix}1 & 0 \\ 0 & 0.2^n\end{pmatrix}\begin{pmatrix}1 & 1 \\ 1 & -1\end{pmatrix}^{-1}\begin{pmatrix}1 \\ 2\end{pmatrix}
= \frac{1}{2}\begin{pmatrix}3-0.2^n \\ 3+0.2^n\end{pmatrix}
$$
と計算できる。
\end{example}

なお、対角化のための重要な過程が固有値と固有ベクトルを計算することであり、その際には行列式が重要な役割を果たす。

また、三項間斉次線形漸化式
$$
x_{n+2} = a x_{n+1}+b x_n
$$
は$y_n = x_{n+1}$を導入すると$y_{n+1} = a y_n+b x_n$より、
$$
\begin{pmatrix}x_{n+1} \\ y_{n+1}\end{pmatrix}
= \begin{pmatrix}0 & 1 \\ b & a\end{pmatrix}\begin{pmatrix}x_n \\ y_n\end{pmatrix},
$$
を得る。
これにより行列での二項間線形漸化式は幅広い問題に対応できることがわかる。

\section{回転行列}

$2$次正方行列の中でも$\theta$を実数として
$$
R(\theta) = \begin{pmatrix}\cos\theta & -\sin\theta \\ \sin\theta & \cos\theta\end{pmatrix}
$$
と表される行列を$\theta$回転の\emph{回転行列}という。
この行列が回転行列と呼ばれるのは座標平面上の点$(a, b)$に対して、回転行列をかけるつまり
$$
\begin{pmatrix}x \\ y\end{pmatrix}
= R(\theta)\begin{pmatrix}a \\ b\end{pmatrix}
$$
として得られる点$(x, y)$は点$(a, b)$を原点中心に$\theta$回転させた点に一致するためである。

この回転行列は上記のように回転にまつわる問題で登場するほか、積の計算が楽である。

\begin{proposition}[回転行列の積]
$\alpha$, $\beta$を実数とする時、
$$
R(\alpha)R(\beta) = R(\alpha+\beta)
$$
が成り立つ。
\end{proposition}

\begin{proof}
計算すると
$$
R(\alpha)R(\beta)
= \begin{pmatrix}\cos\alpha & -\sin\alpha \\ \sin\alpha & \cos\alpha\end{pmatrix}\begin{pmatrix}\cos\beta & -\sin\beta \\ \sin\beta & \cos\beta\end{pmatrix}
= \begin{pmatrix}\cos\alpha\cos\beta-\sin\alpha\sin\beta & -\cos\alpha\sin\beta-\sin\alpha\cos\beta \\ \sin\alpha\cos\beta+\cos\alpha\sin\beta & -\sin\alpha\sin\beta+\cos\alpha\cos\beta\end{pmatrix}.
$$
よって加法定理により
$$
R(\alpha)R(\beta)
= \begin{pmatrix}\cos(\alpha+\beta) & -\sin(\alpha+\beta) \\ \sin(\alpha+\beta) & \cos(\alpha+\beta)\end{pmatrix}
= R(\alpha+\beta)
$$
である。
\end{proof}

特に
$$
R(\theta)R(-\theta) = R(0) = I
$$
なので、$\theta$回転の回転行列の逆行列は反対向きに$\theta$回転の回転行列である、つまり
$$
R(\theta)^{-1} = R(-\theta) = \begin{pmatrix}\cos\theta & \sin\theta \\ -\sin\theta & \cos\theta\end{pmatrix}
$$
が成り立つ。
また、
$$
R(\theta)^n = R(n\theta)
$$
も成り立つ。

\begin{example}
連立一次方程式
$$
\begin{cases}
x-\sqrt{3}y = 1, \\
\sqrt{3}x+y = 0 \\
\end{cases}
$$
を考える。
係数行列は
$$
\begin{pmatrix}1 & -\sqrt{3} \\ \sqrt{3} & 1\end{pmatrix}
= 2\begin{pmatrix}\frac{1}{2} & -\frac{\sqrt{3}}{2} \\ \frac{\sqrt{3}}{2} & \frac{1}{2}\end{pmatrix}
= 2 R(\frac{\pi}{3})
$$
なので、逆行列は$\frac{1}{2}R(-\frac{\pi}{3})$であり、解は
$$
\begin{pmatrix}x \\ y\end{pmatrix}
= \frac{1}{2}\begin{pmatrix}\frac{1}{2} & \frac{\sqrt{3}}{2} \\ -\frac{\sqrt{3}}{2} & \frac{1}{2}\end{pmatrix}\begin{pmatrix}1 \\ 0\end{pmatrix}
= \begin{pmatrix}\frac{1}{4} \\ -\frac{\sqrt{3}}{4}\end{pmatrix}
$$
と計算できる。
\end{example}


\chapter{行列}

\section{体}

行列を定義する前にその構成要素である体について触れる。
\emph{体}とは集合に四則演算(加減乗除)の構造が入ったもののことで、つまり本テキストでは集合$K$に以下を満たす加法$+$、乗法$\cdot$が存在するものとして話を進める。

\begin{enumerate}
\item
(加法の結合法則)任意の$a, b, c \in K$に対して$(a+b)+c = a+(b+c)$が成り立つ。
\item
(加法の交換法則)任意の$a, b \in K$に対して$a+b = b+a$が成り立つ。
\item
(乗法の結合法則)任意の$a, b, c \in K$に対して$(a\cdot b)\cdot c = a\cdot (b\cdot c)$が成り立つ。
\item
(乗法の交換法則)任意の$a, b \in K$に対して$a\cdot b = b\cdot a$が成り立つ。
\item
(分配法則)任意の$a, b, c \in K$に対して$a\cdot (b+c) = a\cdot b+a\cdot c$と$(a+b)\cdot c = a\cdot c+b\cdot c$が成り立つ。
\item
(零元)$0 \in K$がただ一つ存在して、任意の$a \in K$に対して$a+0 = 0+a = a$と$a\cdot 0 = 0\cdot a = 0$が成り立つ。
\item
(反数)任意の$a \in K$に対して$a+x = x+a = 0$が成り立つような$x = -a \in K$がただ一つ存在する。
\item
(単位元)$1 \in K$がただ一つ存在して、任意の$a \in K$に対して$a\cdot 1 = 1\cdot a = a$が成り立つ。
\item
(逆数)任意の$a \in K$に対して$a \ne 0$ならば$a\cdot x = x\cdot a = 1$が成り立つような$x = a^{-1} \in K$がただ一つ存在する。
\end{enumerate}

ここで$1 = 0$の時は体の元はそれのみという場合になるが、
それではいくつかの議論で不具合があるので本テキストでは厳密には$1 \ne 0$も仮定する。

行列や線形代数の理論を展開するうえではこのような体であれば類似のものがいくつもできる。
体の典型例としては有理数や実数、複素数の集合が挙げられ、それぞれ\emph{有理数体}$\mathbb{Q}$、\emph{実数体}$\mathbb{R}$、\emph{複素数体}$\mathbb{C}$と呼ばれる。
有理数や実数、複素数の構成は\cite{N}などを参照のこと。
なお、実数においては四則演算のほかに大小関係や極限操作の構造が入っているが、本テキストの範囲では四則演算のみを用いて展開できる理論のみ紹介する。
その他、体の例としては素数$p$に対して整数を$p$で割った余り$0, \cdots, p-1$に対して和や積を取るたびに$p$で割った余りを取れば体ができる。
このように集合としての元の個数が有限個の体を\emph{有限体}といい、重要な例である。
また、本テキストではいくつかの命題や定理において、割り算(逆数)は必要なく、成分は整数などでもよい。
しかしながら、抽象化や一般化してもきりがないので本テキストでは基本的に体$K$上の行列や線形代数について議論する。

\section{行列}

$M = 1, 2, 3, \cdots$, $N = 1, 2, 3, \cdots$として、$M\times N$個の体$K$の元
$$
a_{1 1}, a_{1 2}, a_{1 3}, \cdots a_{1 N}, a_{2 1}, a_{2 2}, a_{2 3}, \cdots a_{2 N}, a_{3 1}, a_{3 2}, a_{3 3}, \cdots, a_{3 N}, \cdots, a_{M 1}, a_{M 2}, a_{M 3}, \cdots a_{M N} \in K
$$
を縦に$M$個、横に$N$個、長方形状にならべた
$$
\begin{pmatrix}
a_{1 1} & a_{1 2} & a_{1 3} & \cdots & a_{1 N} \\
a_{2 1} & a_{2 2} & a_{2 3} & \cdots & a_{2 N} \\
a_{3 1} & a_{3 2} & a_{3 3} & \cdots & a_{3 N} \\
\vdots & \vdots & \vdots & \ddots & \vdots \\
a_{M 1} & a_{M 2} & a_{M 3} & \cdots & a_{M N} \\
\end{pmatrix}
$$
を$K$上の$M\times N$型の\emph{行列}という。
この行列は$(a_{i j})^{i = 1, \cdots, M}_{j = 1, \cdots, N}$やいくつかの部分を省略して$(a_{i j})^i_j$, $(a_{i j})$のようにも記述したりする。
行列をなす各$a_{i j}$ ($i = 1, \cdots, M$, $j = 1, \cdots, N$)は\emph{成分}といい、上から$i$行目左から$j$行目の成分$a_{i j}$は第$(i, j)$成分と呼ばれる。

行列の例としては、
$$
\begin{pmatrix}
1 & 2 & 3 & 4 & 5 \\
6 & 7 & 8 & 9 & 10 \\
11 & 12 & 13 & 14 & 15 \\
\end{pmatrix}
$$
は実数上の$3\times 5$型の行列であり(整数上でもある)、
$$
\begin{pmatrix}
\cos\theta & -\sin\theta \\
\sin\theta & \cos\theta \\
\end{pmatrix}
\quad (\theta \in K)
$$
は実数上の$2\times 2$型の行列である。

本テキストでは$0$である成分はしばしば省略され、値を記入したり文字を割り当てるほどではない成分に関しては$*$で表したりする。
つまり、
$$
\begin{pmatrix}
1 & 0 & 0 & 0 \\
0 & 2 & 0 & 0 \\
0 & 0 & 3 & 0 \\
0 & 0 & 0 & 4 \\
\end{pmatrix}
=
\begin{pmatrix}
1 & & & \\
& 2 & & \\
& & 3 & \\
& & & 4 \\
\end{pmatrix},
\quad
\begin{pmatrix}
1 & 2 & 0 & 4 & 5 \\
0 & 0 & 1 & 9 & 0 \\
0 & 0 & 0 & 1 & 0 \\
\end{pmatrix}
=
\begin{pmatrix}
1 & * & 0 & * & * \\
& & 1 & * & 0 \\
& & & 1 & 0 \\
\end{pmatrix}
$$
などである。

二つの行列に対して、型が等しく対応する成分が全て等しい時、それらの行列は等しいという。
つまり、
$$
\begin{pmatrix}
a_{1 1} & \cdots & a_{1 N} \\
\vdots & \ddots & \vdots \\
a_{M 1} & \cdots & a_{M N} \\
\end{pmatrix}
=
\begin{pmatrix}
b_{1 1} & \cdots & b_{1 N} \\
\vdots & \ddots & \vdots \\
b_{M 1} & \cdots & b_{M N} \\
\end{pmatrix}
$$
とは各$i = 1, \cdots, M$, $j = 1, \cdots, N$に対して
$$
a_{i j} = b_{i j}
$$
が成り立つことをいう。
また、全ての成分が$0$である行列は\emph{零行列}と呼ばれる。
行列はしばしば大文字の文字が割り当てられるが、$M\times N$型の零行列は$O_{M\times N}$あるいは単に$O$と表される。
つまり、
$$
O_{4\times 2} =
\begin{pmatrix}
0 & 0 \\
0 & 0 \\
0 & 0 \\
0 & 0 \\
\end{pmatrix}
$$
などである。

$K$上の$M\times N$型の行列全体を$\M_{M\times N}(K)$や$K^{M\times N}$などと書く。

$M = N$のときの$M\times N$型の行列を$N$次\emph{正方行列}という。

$N$次正方行列の成分のうち、第$(i, i)$成分($i = 1, \cdots, N$)であるものを\emph{対角成分}といい、それ以外を\emph{非対角成分}という。
非対角成分が全て$0$である正方行列
$$
\begin{pmatrix}
* &        &  \\
  & \ddots &  \\
  &        & *\\
\end{pmatrix}
$$
は\emph{対角行列}と呼ばれ、応用上の扱いが便利である。
対角行列の中でも対角成分が全て$1$である正方行列
$$
\begin{pmatrix}
1 &        &  \\
  & \ddots &  \\
  &        & 1\\
\end{pmatrix}
$$
は\emph{単位行列}と呼ばれ特別な扱いを受ける。
$N$次の単位行列は$I_N$あるいは$E_N$または単に$I$や$E$と表される。
また、正方行列のうち対角成分より左下または右上の成分がすべて$0$である行列
$$
\begin{pmatrix}
* & \cdots & *\\
  & \ddots & \vdots\\
  &        & *\\
\end{pmatrix},
\quad
\begin{pmatrix}
* &        &  \\
\vdots & \ddots &  \\
* & \cdots & *\\
\end{pmatrix}
$$
をまとめて\emph{三角行列}という。
より詳しくは前者を\emph{右上三角行列}、後者を\emph{左下三角行列}という。
以上をまとめると正方行列の中に三角行列があり、さらにその中に対角行列があり、対角行列の特殊なものとして単位行列がある。

$K$上の$N$次正方行列全体を$\M_N(K)$と表す。
また、零行列$O_{N\times N}$である$N$次正方行列は$O_N$とも書かれる。

$N = 1$の時の$M\times 1$型の行列は$M$次の\emph{縦ベクトル}と呼ばれ、
$M = 1$の時の$1\times N$型の行列は$N$次の\emph{横ベクトル}と呼ばれる。
縦ベクトルと横ベクトルを総称して\emph{ベクトル}と言ったりする。

また、全ての成分が$0$であるベクトルは\emph{零ベクトル}と呼ばれる。
ベクトルはしばしば太字の小文字($\vb*{a}, \vb*{b}, \vb*{x}$など)が割り当てられるが、$N$次の零ベクトルは$\vb*{0}_N$あるいは単に$\vb*{0}$と表される。
つまり、
$$
\vb*{0}_3 = \begin{pmatrix}0 \\ 0 \\ 0\end{pmatrix}
$$
などである。

$N$個の$N$次のベクトル$\vb*{e}_1, \cdots, \vb*{e}_N$を$i = 1, \cdots, N$に対して第$i$行が$1$でそれ以外の成分がすべて$0$であるものを$\vb*{e}_i$となるように定義し、しばしば\emph{標準基底ベクトル}と呼ばれる。
つまり縦ベクトルならば、
$$
\vb*{e}_1 = \begin{pmatrix}1 \\ 0 \\ \vdots \\ 0\end{pmatrix},
\quad \cdots,
\quad \vb*{e}_N = \begin{pmatrix}0 \\ \vdots \\ 0 \\ 1\end{pmatrix}
$$
である。

$M = N = 1$の時の$1\times 1$型の行列はただ一つの成分からなり、行列を$\begin{pmatrix}a\end{pmatrix}$とおくとこれを$a$と同一視してしまう。
このような行列に対してその成分のことや$1\times 1$型の行列のことを\emph{スカラー}と呼ぶ。
例えば本来ならば(行列の積を取った時などに)$1\times 1$型の行列になるべきだが$K$の元として扱われることがあるので注意する。

\section{行列の演算}

これから行列の三つの演算、つまり和とスカラー倍と行列の積を導入する。

まず、行列の和は成分ごとの和として定義される。
ここで二つの行列の型は同じである必要がある。
つまり、$K$上の$M\times N$型の行列$A = (a_{i j})$と$B = (b_{i j})$に対して、
$$
A+B =
\begin{pmatrix}
a_{1 1} & \cdots & a_{1 N} \\
\vdots & \ddots & \vdots \\
a_{M 1} & \cdots & a_{M N} \\
\end{pmatrix}
+
\begin{pmatrix}
b_{1 1} & \cdots & b_{1 N} \\
\vdots & \ddots & \vdots \\
b_{M 1} & \cdots & b_{M N} \\
\end{pmatrix}
=
\begin{pmatrix}
a_{1 1}+b_{1 1} & \cdots & a_{1 N}+b_{1 N} \\
\vdots & \ddots & \vdots \\
a_{M 1}+b_{M 1} & \cdots & a_{M N}+b_{M N} \\
\end{pmatrix}
$$
とする。

スカラー倍とは$K$上の$M\times N$型の行列$A = (a_{i j})$とスカラー$c \in K$に対して定まるもので、行列の各成分を$c$倍する。
つまり、
$$
c A =
c
\begin{pmatrix}
a_{1 1} & \cdots & a_{1 N} \\
\vdots & \ddots & \vdots \\
a_{M 1} & \cdots & a_{M N} \\
\end{pmatrix}
=
\begin{pmatrix}
c a_{1 1} & \cdots & c a_{1 N} \\
\vdots & \ddots & \vdots \\
c a_{M 1} & \cdots & c a_{M N} \\
\end{pmatrix}
$$
である。

この二つの演算で行列の集合$\M_{M\times N}(K)$は線形性と呼ばれる構造を持ち、以下の計算法則が成立する。

\begin{enumerate}
\item
(行列加法の結合法則)任意の$A, B, C \in \M_{M\times N}(K)$に対して$(A+B)+C = A+(B+C)$が成り立つ。
\item
(行列加法の交換法則)任意の$A, B \in \M_{M\times N}(K)$に対して$A+B = B+A$が成り立つ。
\item
(行列スカラー乗法の結合法則)任意の$A \in \M_{M\times N}(K)$と$c, d \in K$に対して$c(d A) = (c d)A$が成り立つ。
\item
(分配法則1)任意の$A, B \in \M_{M\times N}(K)$と$c \in K$に対して$c(A+B) = c A+c B$が成り立つ。
\item
(分配法則2)任意の$A \in \M_{M\times N}(K)$と$c, d \in K$に対して$(c+d)A = c A+d A$が成り立つ。
\item
(零行列)$O = O_{M\times N}$は零元である。つまり任意の$A \in \M_{M\times N}(K)$に対して$A+O = O+A = A$と$0 A = O$が成り立つ。
\item
(反行列)任意の$A \in \M_{M\times N}(K)$に対して$A+X = X+A = O$が成り立つような$X = -A \in \M_{M\times N}(K)$がただ一つ存在し、それは$-A = (-1)A$である。
\item
(単位数)$1$は単位数である。つまり任意の$A \in \M_{M\times N}(K)$に対して$1 A = A$が成り立つ。
\end{enumerate}

行列の積は行列を行列たらしめる特徴的な演算である。
$L = 1, 2, 3, \cdots$として、$M\times L$型の行列$A = (a_{i j})$と$M\times L$型の行列$B = (b_{i j})$に対して、
$$
A B =
\begin{pmatrix}
a_{1 1} & \cdots & a_{1 L} \\
\vdots & \ddots & \vdots \\
a_{M 1} & \cdots & a_{M L} \\
\end{pmatrix}
\begin{pmatrix}
b_{1 1} & \cdots & b_{1 N} \\
\vdots & \ddots & \vdots \\
b_{L 1} & \cdots & b_{L N} \\
\end{pmatrix}
=
\begin{pmatrix}
a_{1 1}b_{1 1}+\cdots+a_{1 L}b_{L 1} & \cdots & a_{1 1}b_{1 N}+\cdots+a_{1 L}b_{L N} \\
\vdots & \ddots & \vdots \\
a_{M 1}b_{1 1}+\cdots+a_{M L}b_{L 1} & \cdots & a_{M 1}b_{1 N}+\cdots+a_{M L}b_{L N} \\
\end{pmatrix}
$$
とする。
つまり、演算結果は$M\times N$型の行列$C = A B = (c_{i j})$で
$$
c_{i j} = a_{i 1}b_{1 j}+a_{i 2}b_{2 j}+a_{i 3}b_{3 j}+\cdots+a_{i L}b_{L j}
$$
となっている。

このように一つの成分を見てもたくさんのかけ算と足し算で成り立っており一見妥当性がわからないが、
実際にはさまざまな場面でこの行列の積が現れる(第1章参照)。

\begin{enumerate}
\item
(行列積の結合法則)任意の$A \in \M_{M\times K}(K), B \in \M_{K\times L}(K), C \in \M_{L\times N}(K)$に対して$(A B)C = A(B C)$が成り立つ。
\item
(分配法則1)任意の$A, B \in \M_{M\times L}(K)$と$B \in \M_{L\times N}(K)$に対して$(A+B)C = A C+B C$が成り立つ。
\item
(分配法則2)任意の$A \in \M_{M\times L}(K)$と$B, C \in \M_{L\times N}(K)$に対して$A(B+C) = A B+A C$が成り立つ。
\item
(零行列)任意の$A \in \M_{M\times N}(K)$に対して$A O_{N\times L} = O_{M\times L}$と$O_{L\times M} A = O_{L\times N}$が成り立つ。
\item
(単位行列)任意の$A \in \M_{M\times N}(K)$に対して$I_M A = A I_N = A$が成り立つ。
\end{enumerate}

ここで重要な注意として行列の積に対して交換法則はない。
つまり、$A B = B A$は一般には成り立たない。
ここで$A$を$M\times N$型で$B$を$K\times L$型とすると、積を定義して両辺の型を合わせるためには$K = L = M = N$とするしかないが、$A, B \in \M_N(K)$の場合でも$A B = B A$は一般には成り立たない。
実数でいうところの逆数に相当する逆行列は行列の理論の重要な論点であり次の節やその後の章で詳しく述べる。

行列$A$と$B$の積$A B$は、行列$B$に行列$A$を左からかけて得られる行列といい、同時に行列$A$に行列$B$を右からかけて得られる行列ともいう。

$N$次正方行列$A$に対しては自分自身との積$A A$が定義できこれも$N$次正方行列なのでさらに$A$をかけられる。
結合法則より左からかけても右からかけてもこの場合は同じ$A(A A) = (A A)A = A A A$が得られることに注意する。
このようにして$N$次正方行列$A$の$n = 0, 1, 2, 3, \cdots$乗$A^n$を単位行列$I_N$に$A$を$n$回かけたものとして定義する。

\begin{example}
対角行列
$$
A =
\begin{pmatrix}
a_1 &        &    \\
    & \ddots &    \\
    &        & a_N\\
\end{pmatrix}
$$
の$n$乗は
$$
A^n =
\begin{pmatrix}
a_1^n &        &    \\
    & \ddots &    \\
    &        & a_N^n\\
\end{pmatrix}
$$
である。
\end{example}

やろうと思えば三角行列、特に後の章で出てくるジョルダン標準形と呼ばれる行列の$n$乗も比較的簡単に計算できる。
一般の行列の$n$乗を計算する時には手で計算するならば対角化または三角化して対角行列またはジョルダン標準形の$n$乗に帰着させることになるが、
他にも次のようにして計算量を削減できることが知られている。
つまり、行列$A$の$2^k$乗($k = 0, 1, 2, 3, \cdots$)を
$$
A^1 = A,
\quad A^2 = A^1 A^1,
\quad A^4 = A^2 A^2,
\quad A^8 = A^4 A^4,
\quad \cdots
$$
の要領であらかじめ計算して、$n$を$2$進数表示することで$A$の$2^k$乗をいくつかかけ合わせることで$A^n$を得る。
例えば、
$$
A^10 = A^8 A^2
$$
といった具合である。
この方法を\emph{二進数法}という。
これを使えば$2\log_2 n$回以内の行列の積で$A^n$が計算できることになり、計算機での計算に広く使われている。

$K$上の$n = 0, 1, 2, 3, \cdots$次の\emph{多項式}とはスカラー$c_0, c_1, c_2, c_3, \cdots, c_n \in K$ ($c_n \ne 0$)を使って
$$
f(x) = c_0+c_1 x+c_2 x^2+c_3 x^3+\cdots+c_n x^n
$$
と表される式のことであるが、
この多項式$f(x)$と$K$上の$N$次正方行列$A$に対して多項式の変数$x$に行列$A$が代入できると拡大解釈して、$N$次正方行列
$$
f(A) = c_0 I_N+c_1 A+c_2 A^2+c_3 A^3+\cdots+c_n A^n
$$
を定める。
定数項$c_0$は単位行列の項$c_0 I$になることに注意する。

一般には行列の積は交換できないが、上記の多項式は行列が$A$(と$I$)しか登場しないので以下が成り立つ。

\begin{proposition}
多項式$f(x)$, $g(x)$と$N$次正方行列$A$に対して、
$$
f(A)g(A) = g(A)f(A)
$$
が成り立つ。
\end{proposition}

以上の内容を駆使して行列の特徴を利用すると以下のように行列の計算を簡単に行える場合がある。

\begin{example}
\label{t:polyex}
$X = \begin{pmatrix}1 & 1 & 1 \\ 1 & 1 & 1 \\ 1 & 1 & 1\end{pmatrix}$として$3$次正方行列$(X-I)(X+2 I)(X-I)$を計算する。
$X^2 = 3 X$であることに注意して、
$$
(X-I)(X+2 I)(X-I) = (X-I)^2(X+2 I) = (X^2-2 X+I)(X+2 I) = (X+I)(X+2 I) = X^2+3 X+2 I = 2(3 X+I).
$$
よって答えは$(X-I)(X+2 I)(X-I) = \begin{pmatrix}8 & 2 & 2 \\ 2 & 8 & 2 \\ 2 & 2 & 8\end{pmatrix}$である。
\end{example}

\section{逆行列}

実数$a$の逆数とは$a\cdot x = x\cdot a = 1$が成り立つような実数$x$のことであった。
行列$A$に対しても同様に逆行列の概念を定義するが、定義の都合上正方行列にしか定義されないことに注意する。

\begin{definition}
$A$を$K$上の$N$次正方行列とする。
$$
A X = X A = I_N
$$
が成り立つような$K$上の$N$次正方行列$X$を$A$の\emph{逆行列}といい(後に述べるように一意なので)$A^{-1}$と表す。
\end{definition}

逆行列は存在すれば一意である。

\begin{proposition}
$N$次正方行列$A$は逆行列を持つならば一つしかない、つまり$N$次正方行列$X$と$Y$が$A$の逆行列ならば$X = Y$である。
\end{proposition}

\begin{proof}
仮定より$A X = A Y = I_N$であり、左から$X$をかけることで$X A X = X A Y$である。
ここで$X A = I_N$なので、$X = Y$となる。
\end{proof}

その一方で逆行列は常に存在するとは限らない。
もちろん零行列$O$は何をかけても零行列で単位行列にはならないので逆行列を持たない。

\begin{example}
$2$次正方行列
$$
A = \begin{pmatrix}0 & 1 \\ 0 & 0\end{pmatrix}
$$
は逆行列を持たない。
なぜなら
$$
\begin{pmatrix}0 & 1 \\ 0 & 0\end{pmatrix}\begin{pmatrix}* & * \\ * & *\end{pmatrix}
= \begin{pmatrix}* & * \\ 0 & 0\end{pmatrix}
$$
なので第$(2, 2)$成分が$1$にならないためである。
\end{example}

\begin{example}
対角成分$a_i$がいずれも$0$でない対角行列
$$
A =
\begin{pmatrix}
a_1 &        &    \\
    & \ddots &    \\
    &        & a_N\\
\end{pmatrix}
$$
の逆行列は
$$
A^{-1} =
\begin{pmatrix}
a_1^{-1} &        &    \\
    & \ddots &    \\
    &        & a_N^{-1}\\
\end{pmatrix}
$$
である。
\end{example}

逆行列を持つような行列は\emph{正則行列}または\emph{可逆行列}と呼ばれる。
$K$上の$N$次正方行列全体$\M_N(K)$のうち正則行列全体を$\GL_N(K)$と表す。

すぐわかる逆行列の性質として以下がある。

\begin{enumerate}
\item
(逆行列の逆行列)任意の$A \in \GL_N(K)$に対して$A^{-1} \in \GL_N(K)$であり$(A^{-1})^{-1} = A$が成り立つ。
\item
(スカラー倍の逆行列)任意の$A \in \GL_N(K)$と$0$でない$c \in K$に対して$c A \in \GL_N(K)$であり$(c A)^{-1} = c^{-1} A^{-1}$が成り立つ。
\item
(積の逆行列)任意の$A, B \in \GL_N(K)$に対して$A B \in \GL_N(K)$であり$(A B)^{-1} = B^{-1} A^{-1}$が成り立つ。
\end{enumerate}

三番目の性質は逆行列を取ると積の順序が逆になることに注意する。

\begin{theorem}
正方行列$A$に対して、$1$次以上の多項式$f(x) = c_0+c_1 x+c_2 x^2+c_3 x^3+\cdots+c_n x^n$であって$f(A) = O$となるものが存在したとする。
ここで$c_0 \ne 0$の時、$A$は正則であり、逆行列は
$$
A^{-1} = -c_0^{-1}(c_1 I+c_2 A+c_3 A^2+\cdots+c_n A^{n-1})
$$
で与えられる。
\end{theorem}

\begin{example}
$3$次正方行列$A = \begin{pmatrix}4 & -1 & -1 \\ -1 & 4 & -1 \\ -1 & -1 & 4\end{pmatrix}$の逆行列を計算してみよう。
次章で述べる掃き出し法を実行してもよいが、ここでは以下のように考えてみる。
つまり、$X = \begin{pmatrix}1 & 1 & 1 \\ 1 & 1 & 1 \\ 1 & 1 & 1\end{pmatrix}$とおくと、$A = 5 I-X$で$X^2 = 3 X$なので$A^2-7 A+10 I = O$、つまり
$$
A^{-1} = -\frac{1}{10}(A-7 I) = \frac{1}{10}(X+2 I).
$$
よって逆行列は$A = \frac{1}{10}\begin{pmatrix}3 & 1 & 1 \\ 1 & 3 & 1 \\ 1 & 1 & 3\end{pmatrix}$である。
\end{example}

逆行列の定義において、行列の積の交換法則がないので$A X = I_N$と$X A = I_N$の両方を要求していたが、実際には片方が成立すればもう片方も従う。
この証明には基本変形を用いる必要があるので証明は第3章で行う。

\section{転置行列と対称行列}

行列の行と列を入れ替える操作を転置といい、定義しておけば便利になることが多い。
例えば行列式の基本変形は行について示せば転置をして列に対しても同様のものが成り立つことがわかる。
また、スカラー積を考えるときも転置によって行列の積の話に帰着できる。

\begin{definition}[転置行列]
$M\times N$行列$A = (a_{i j})^{i = 1, \cdots, M}_{j = 1, \cdots, N}$に対して$N\times M$行列
$$
A^T = (a_{j i})^{i = 1, \cdots, N}_{j = 1, \cdots, M}
$$
を$A$の\emph{転置行列}という。
つまり、
$$
A =
\begin{pmatrix}
a_{1 1} & \cdots & a_{1 N} \\
\vdots & \ddots & \vdots \\
a_{M 1} & \cdots & a_{M N} \\
\end{pmatrix}
$$
に対して、
$$
A^T =
\begin{pmatrix}
a_{1 1} & \cdots & a_{M 1} \\
\vdots & \ddots & \vdots \\
a_{1 N} & \cdots & a_{M N} \\
\end{pmatrix}
$$
である。
\end{definition}

すぐわかる転置の性質として以下がある。

\begin{enumerate}
\item
(転置の転置)任意の$A \in \M_{M\times N}(K)$に対して$(A^T)^T = A$が成り立つ。
\item
(和の転置)任意の$A, B \in \M_{M\times N}(K)$に対して$(A+B)^T = A^T+B^T$が成り立つ。
\item
(スカラー倍の転置)任意の$A \in \M_{M\times N}(K)$と$c \in K$に対して$(c A)^T = c A^T$が成り立つ。
\item
(積の転置)任意の$A \in \M_{M\times L}(K)$と$B \in \M_{L\times N}(K)$に対して$(A B)^T = B^T A^T$が成り立つ。
\end{enumerate}

四番目の性質は転置を取ると積の順序が逆になることに注意する。

また、この性質を使えば可逆行列$A$に対して転置行列$A^T$も可逆で
$$
(A^T)^{-1} = (A^{-1})^T
$$
が成り立つことがわかる。

\begin{definition}[対称行列]
$N$次正方行列$A = (a_{i j})^{i = 1, \cdots, N}_{j = 1, \cdots, N}$に対して転置をとったものが元の行列と等しい、つまり
$$
A^T = A
$$
の場合、$A$は\emph{対称}あるいは\emph{対称行列}であるという。
成分で書けば
$$
a_{j i} = a_{i j} \quad (i, j = 1, \cdots, N)
$$
である。
\end{definition}

スカラーは常に対称である。

\section{行列の区分け}

行列をいくつかの小さな行列の並びとみなせると計算が楽になることが多い。
そこでまず小さな行列をもとに大きな行列を作ることを許すことにする。
つまり、$M_1, \cdots, M_I = 1, 2, 3, \cdots$と$N_1, \cdots, N_J = 1, 2, 3, \cdots$として、$K$上の$M_i\times N_j$型の行列$A_{i j}$が与えられたとする($i = 1, \cdots, I$, $j = 1, \cdots, J$)、
このとき$A_{i j}$の成分をすべて並べて得られる$K$上の$(M_1+\cdots+M_I)\times(N_1+\cdots+N_J)$型の行列を
$$
\begin{pmatrix}
A_{1 1} & \cdots & A_{1 J} \\
\vdots & \ddots & \vdots \\
A_{I 1} & \cdots & A_{I J} \\
\end{pmatrix}
$$
と書く。
反対に$(M_1+\cdots+M_I)\times(N_1+\cdots+N_J)$型の行列$A$が与えられた時に上記のように小さな行列に分けることを行列の\emph{区分け}という。

\begin{example}
$\vb*{e}_1, \cdots, \vb*{e}_N$を$N$次の縦の標準基底ベクトルとする。
つまり、
$$
\vb*{e}_1 = \begin{pmatrix}1 \\ 0 \\ \vdots \\ 0\end{pmatrix},
\quad \cdots,
\quad \vb*{e}_N = \begin{pmatrix}0 \\ \vdots \\ 0 \\ 1\end{pmatrix}
$$
である。
このとき$N$次の単位行列$I_N$は
$$
I_N = \begin{pmatrix}\vb*{e}_1 & \cdots & \vb*{e}_N\end{pmatrix}
$$
と区分けされる。
\end{example}

区分けされた行列に対しては演算規則が自然と拡張され、
特に区分けされた行列の積は以下のように計算できる。

\begin{proposition}[区分けされた行列の積]
区分けされた$(M_1+\cdots+M_I)\times(L_1+\cdots+L_H)$型の行列$A = (A_{i j})$と$(L_1+\cdots+L_H)\times(N_1+\cdots+N_J)$型の行列$B = (B_{i j})$に対して、
$$
\begin{pmatrix}
A_{1 1} & \cdots & A_{1 H} \\
\vdots & \ddots & \vdots \\
A_{I 1} & \cdots & A_{I H} \\
\end{pmatrix}
\begin{pmatrix}
B_{1 1} & \cdots & B_{1 J} \\
\vdots & \ddots & \vdots \\
B_{H 1} & \cdots & B_{H J} \\
\end{pmatrix}
=
\begin{pmatrix}
A_{1 1}B_{1 1}+\cdots+A_{1 H}B_{H 1} & \cdots & A_{1 1}B_{1 J}+\cdots+A_{1 H}b_{H J} \\
\vdots & \ddots & \vdots \\
A_{I 1}B_{1 1}+\cdots+A_{I H}B_{H 1} & \cdots & A_{I 1}B_{1 J}+\cdots+A_{I H}b_{H J} \\
\end{pmatrix}
$$
が成り立つ。
特に$M\times L$型の行列$A$と$L\times N$型の行列$B$に対して、$B$の$N$個の列を構成する$L$次の縦ベクトルを$\vb*{b}_1, \cdots, \vb*{b}_N$とすると、
$$
A B
= A\begin{pmatrix}\vb*{b}_1 & \cdots & \vb*{b}_N\end{pmatrix}
= \begin{pmatrix}A\vb*{b}_1 & \cdots & A\vb*{b}_N\end{pmatrix}
$$
が成り立つ。
\end{proposition}

\begin{proof}
\end{proof}

\begin{example}
\label{t:blockinv}
$A$を$N_1$次の正則行列、$D$を$N_2$次の正則行列として$N_1+N_2$次の正方行列
$$
\begin{pmatrix}A & B \\ O & D\end{pmatrix}
$$
は正則であり、逆行列は
$$
\begin{pmatrix}A^{-1} & -A^{-1}B D^{-1} \\ O & D^{-1}\end{pmatrix}
$$
で与えられる。
実際、積を計算すると
$$
\begin{pmatrix}A & B \\ O & D\end{pmatrix}\begin{pmatrix}A^{-1} & -A^{-1}B D^{-1} \\ O & D^{-1}\end{pmatrix}
= \begin{pmatrix}A A^{-1} & -A A^{-1}B D^{-1}+B D^{-1} \\ O & D D^{-1}\end{pmatrix}
= \begin{pmatrix}I & O \\ O & I\end{pmatrix},
$$
$$
\begin{pmatrix}A^{-1} & -A^{-1}B D^{-1} \\ O & D^{-1}\end{pmatrix}\begin{pmatrix}A & B \\ O & D\end{pmatrix}
= \begin{pmatrix}A^{-1}A & A^{-1}B-A^{-1}B D^{-1}D \\ O & D^{-1}D\end{pmatrix}
= \begin{pmatrix}I & O \\ O & I\end{pmatrix}
$$
であることからわかる。
\end{example}

\begin{example}
実数$\theta$に対して、$4$次正方行列
$$
A =
\begin{pmatrix}
\cos\theta & -\sin\theta & -\cos\theta & -\sin\theta \\
\sin\theta & \cos\theta & \sin\theta & -\cos\theta \\
\cos\theta & \sin\theta & \cos\theta & -\sin\theta \\
-\sin\theta & \cos\theta & \sin\theta & \cos\theta \\
\end{pmatrix}
$$
は
$$
A =
\begin{pmatrix}
R(\theta) & -R(-\theta) \\
R(-\theta) & R(\theta) \\
\end{pmatrix}
$$
とみなすことができる。
\end{example}


\chapter{連立一次方程式}

\section{拡大係数行列}

$N = 1, 2, 3, \cdots$個の未知数$x_1, \cdots, x_N$と$M = 1, 2, 3, \cdots$個の等式からなる次の形の連立方程式
$$
\begin{cases}
a_{1 1}x_1+\cdots+a_{1 N}x_N = b_1, \\
\quad \vdots \\
a_{M 1}x_1+\cdots+a_{M N}x_N = b_M \\
\end{cases}
$$
のことを\emph{連立一次方程式}あるいは\emph{線形方程式系}などという。
$M$と$N$が異なる場合は解がなかったり複数あったりするのでそのような場合を考える必要を感じないかもしれないが、さまざまな場合を扱えた方が良いのでここでは$M$と$N$が異なる場合を含めて考える(そもそも$M = N$でも解がなかったり複数あったりする場合が存在する)。
第1章で扱ったように連立一次方程式は行列とベクトルを用いて
$$
\begin{pmatrix}
a_{1 1} & \cdots & a_{1 N} \\
\vdots & \ddots & \vdots \\
a_{M 1} & \cdots & a_{M N} \\
\end{pmatrix}
\begin{pmatrix}
x_1 \\
\vdots \\
x_N \\
\end{pmatrix}
=
\begin{pmatrix}
b_1 \\
\vdots \\
b_M \\
\end{pmatrix}
$$
と表すことができる。
ここで
$$
A =
\begin{pmatrix}
a_{1 1} & \cdots & a_{1 N} \\
\vdots & \ddots & \vdots \\
a_{M 1} & \cdots & a_{M N} \\
\end{pmatrix},
\quad
\vb*{x} =
\begin{pmatrix}
x_1 \\
\vdots \\
x_N \\
\end{pmatrix},
\quad
\vb*{b} =
\begin{pmatrix}
b_1 \\
\vdots \\
b_M \\
\end{pmatrix}
$$
として、$M\times N$型の行列$A$を連立一次方程式の\emph{係数行列}、$M$次ベクトル$\vb*{b}$を\emph{定数ベクトル}、$N$次ベクトル$\vb*{x}$を\emph{未知数ベクトル}とそれぞれ呼ぶ。
また、連立一次方程式は
$$
\begin{pmatrix}
a_{1 1} & \cdots & a_{1 N} & b_1 \\
\vdots & \ddots & \vdots & \vdots \\
a_{M 1} & \cdots & a_{M N} & b_M \\
\end{pmatrix}
\begin{pmatrix}
x_1 \\
\vdots \\
x_N \\
-1 \\
\end{pmatrix}
=
\begin{pmatrix}
0 \\
\vdots \\
0 \\
\end{pmatrix}
$$
のようにも変形できる。
この時の係数行列$A$と定数ベクトル$\vb*{b}$を並べて得られる$M\times(N+1)$型の行列
$$
\tilde{A}
= \begin{pmatrix}A & \vb*{b}\end{pmatrix}
=
\begin{pmatrix}
a_{1 1} & \cdots & a_{1 N} & b_1 \\
\vdots & \ddots & \vdots & \vdots \\
a_{M 1} & \cdots & a_{M N} & b_M \\
\end{pmatrix}
$$
を連立一次方程式の\emph{拡大係数行列}という。
連立一次方程式は係数行列と定数ベクトルが与えられたら定まるので、拡大係数行列は連立一次方程式の必要な情報をすべて表現する。

\section{基本変形}

第1章で紹介した連立一次方程式の解法は方程式
$$
A\vb*{x} = \vb*{b}
$$
の両辺に係数行列$A$の逆行列を左からかけるというものであったが、
この方法が適用できるのは$A$が正方行列つまり$M = N$であって逆行列を持つ場合である。
しかしながら今回は一般の状況で方程式を解くために別の方法を考える必要がある。
加えて逆行列の簡便な公式があるのは$2$次の場合だけで、$3$次以上だと連立一次方程式を発展させた方程式
$$
A X = I
$$
を解く必要がある。

連立一次方程式を解くために使われる方法が基本変形である。
基本変形による方法は掃き出し法やガウスの消去法とも呼ばれる重要な解法である。

基本変形の基本的な着想は中学校で習う加減法である。
つまり、$M$個の等式からなる連立一次方程式に対して次の三つの操作を考える。
\begin{itemize}
\item[(1)]
二つの等式を入れ替える。
\item[(2)]
ある等式を$0$でない定数倍する。
\item[(3)]
ある等式に、別の等式の定数倍を加える。
\end{itemize}
これらをまとめて連立方程式の\emph{基本変形}という。
基本変形は同値な変形になっており、繰り返し用いることで連立一次方程式を解くことができる。

\begin{example}
鶴亀算の方程式
$$
\begin{cases}
x+y = 8, \\
2 x+4 y = 26 \\
\end{cases}
$$
を考える。
第2式に第1式の$-2$倍を足して、
$$
\begin{cases}
x+y = 8, \\
2 y = 10. \\
\end{cases}
$$
第2式を$\frac{1}{2} \ne 0$倍して、
$$
\begin{cases}
x+y = 8, \\
y = 5. \\
\end{cases}
$$
第1式に第2式の$-1$倍を足して、
$$
\begin{cases}
x = 3, \\
y = 5. \\
\end{cases}
$$
これにて解$(x, y) = (3, 5)$が得られた。
\end{example}

これを拡大係数行列への操作に置き換えると次が対応する。
\begin{itemize}
\item[(1)]
二つの行を入れ替える。
\item[(2)]
ある行を$0$でない定数倍する。
\item[(3)]
ある行に、別の行の定数倍を加える。
\end{itemize}
これらをまとめて行列の\emph{行基本変形}という。
なお、行のところを列に変えることで\emph{列基本変形}も定義でき、行基本変形と列基本変形を合わせて行列の\emph{基本変形}というが、
連立一次方程式の基本変形に対応するのは行基本変形であることに注意する。

行列の基本変形を導入する利点は基本変形は以下の通りの可逆行列をかけることとして定式化できることにある。
つまり$M\times N$型の行列に対する行基本変形は$i, j = 1, 2, 3, \cdots, M$で$c$をスカラーとして次の操作と考えられる。
\begin{itemize}
\item[(1)]
第$i$行と第$j$行を入れ替える基本変形は、正方行列
$$
P_M(i, j) =
\begin{pmatrix}
\ddots &   &        &   &       \\
       & 0 &        & 1 &       \\
       &   & \ddots &   &       \\
       & 1 &        & 0 &       \\
       &   &        &   & \ddots\\
\end{pmatrix}
$$
を左からかけることである(三点で省略した部分は$1$)。
\item[(2)]
第$i$行を$c \ne 0$倍する基本変形は、正方行列
$$
Q_M(i, c) =
\begin{pmatrix}
\ddots &   &        &   &       \\
       & c &        &   &       \\
       &   & \ddots &   &       \\
       &   &        & 1 &       \\
       &   &        &   & \ddots\\
\end{pmatrix}
$$
を左からかけることである(三点で省略した部分は$1$)。
\item[(3)]
第$i$行に第$j \ne i$行の$c$倍を加える基本変形は、正方行列
$$
R_M(i, j, c) =
\begin{pmatrix}
\ddots &   &        &   &       \\
       & 1 &        & c &       \\
       &   & \ddots &   &       \\
       &   &        & 1 &       \\
       &   &        &   & \ddots\\
\end{pmatrix}
$$
を左からかけることである(三点で省略した部分は$1$)。
\end{itemize}
これらの正方行列$P_M(i, j)$, $Q_M(i, c)$, $R_M(i, j, c)$をまとめて\emph{基本行列}という。
なお、行基本行列と呼ばないのは列基本変形は基本行列$P_N(i, j)$, $Q_N(i, c)$, $R_N(i, j, c)$を右からかけることに相当するので行基本変形と共通であるためである。

基本行列はいずれも可逆である。
実際、
$$
P_M(i, j)^{-1} = P_M(i, j),
\quad Q_M(i, c)^{-1} = Q_M(i, c^{-1}),
\quad R_M(i, j, c)^{-1} = R_M(i, j, -c)
$$
である。
そのため行列の基本変形は同値な変形となっている。

\begin{example}
やっていることは連立一次方程式の基本変形と同じだが鶴亀算の方程式の拡大係数行列を行基本変形すると以下のようになる。
$$
\begin{pmatrix}1 & 1 & 8 \\ 2 & 4 & 26\end{pmatrix}
\to \begin{pmatrix}1 & 1 & 8 \\ 0 & 2 & 10\end{pmatrix}
\to \begin{pmatrix}1 & 1 & 8 \\ 0 & 1 & 5\end{pmatrix}
\to \begin{pmatrix}1 & 0 & 3 \\ 0 & 1 & 5\end{pmatrix}
$$
ここから解は$(x, y) = (3, 5)$が結論づけられる。
\end{example}

\section{掃き出し法と行列の階数}

基本変形を用いて一定の手続きで連立一次方程式を解く手法が掃き出し法(ガウスの消去法)である。
$A$を$M\times N$型の行列として第$(i, j)$成分について\emph{掃き出す}ことを以下で定義する。

\begin{enumerate}
\item
$A$の第$(i, j)$成分$a_{i j}$で第$i$行を割る、つまり逆数$a_{i j}^{-1}$をかける。
\item
$A$の第$i$行以外の行(第$k$行とする)から第$i$行の$a_{k j}$倍を引く、つまり$-a_{k j}$倍を足す。
\end{enumerate}

この操作により第$j$列は第$i$行だけ$1$でそれ以外の成分はすべて$0$となるように変形される。
なお、第$(i, j)$成分$a_{i j}$が$0$の場合はこの操作は行えないことに注意する。

この掃き出しを用いて$M\times N$型の行列$A$を以下のように変形していく。

\begin{enumerate}
\item
最初$i = 1$, $j = 1$とする。
\item
第$(i, j)$成分が零の場合はそれより下に、つまり$k = i+1, \cdots, M$として第$(k, j)$成分が非零の行を見つけて第$i$行と第$k$行を入れ替える。
\item
第$(i, j)$成分が非零になったところで第$(i, j)$成分について掃き出し、$i$を$1$増やす。
\item
非零の行が見つからない場合は$j$を$1$増やす。
\item
上記のことを繰り返して$i$が$M$を超えたり$j$が$N$を超えたところで変形を終了する。
\end{enumerate}

この掃き出しを\emph{行簡約化}といい、それよって行列は以下の形になる。

\begin{definition}[行簡約行列]
$M\times N$型の行列が\emph{行簡約行列}であるとは以下の条件が満たされることをいう。
各行(第$i$行)に対して成分が非零である一番左の成分の列数を$j(i)$として(ただし全ての成分が零の時は$j(i) = \infty$とする)、
$j(i)$は単調増加かつ第$j(i)$列は第$(i, j(i))$成分だけが$1$でそれ以外はすべて零である。
またこの時、第$(i, j(i))$成分を行簡約行列の\emph{主成分}といい、その個数を行簡約行列の\emph{階数}という。
\end{definition}

例えば次は行簡約行列であり階数は$2$である。

$$
\begin{pmatrix}
0 & 1 & * & 0 & * \\
0 & 0 & 0 & 1 & * \\
0 & 0 & 0 & 0 & 0 \\
\end{pmatrix}.
$$

重要なことはこの行簡約行列は元の行列に対して一意に決まることである。
行簡約行列$\bar{A}$を$\begin{pmatrix}\tilde{A} & *\end{pmatrix}$と区分けすると前半部分の$\tilde{A}$も行簡約行列であることに注意する。

\begin{theorem}[行簡約行列の一意性]
行列$A$に行基本変形を繰り返し用いることで簡約行列$\bar{A}_1$が得られて、別の変形方法によって簡約行列$\bar{A}_2$が得られたとする時、$\bar{A}_1 = \bar{A}_2$である。
\end{theorem}

\begin{proof}
行基本変形は可逆なので、正則行列$P_1, P_2$が存在して$P_1 A = \bar{A}_1$, $P_2 A = \bar{A}_2$が成立する。
よって、正則行列$P = P_2 P_1^{-1}$により$\bar{A}_2 = P\bar{A}_1$とでき、このとき$\bar{A}_2 = \bar{A}_1$を示せばよい。
列数$N$についての数学的帰納法で証明する。
すぐわかることとして$N = 1$の時は成立する(行簡約行列は縦零ベクトル$\vb*{0}$か$\vb*{e}_1$しかない)。
列数が$N$で成立するとき列数が$N+1$の場合について考える。
$\bar{A}_1 = \begin{pmatrix}\tilde{A}_1 & \vb*{a}_1\end{pmatrix}$, $\bar{A}_2 = \begin{pmatrix}\tilde{A}_2 & \vb*{a}_2\end{pmatrix}$と区分けして、$\bar{A}_2 = P\bar{A}_1$より$\tilde{A}_2 = P\tilde{A}_1$かつ$\vb*{a}_2 = P\vb*{a}_1$である。
よって数学的帰納法の仮定より$\tilde{A}_1 = \tilde{A}_2$であり、これを$\tilde{A}$としてその階数を$R$とおく。
このとき、行簡約行列の定義から各$i = 1, \cdots, R$に対して$\tilde{A}$の第$j(i)$行は$\vb*{e}_i$で、$\tilde{A} = P\tilde{A}$から$P\vb*{e}_i = \vb*{e}_i$を得る。
ここで階数の定義から$\bar{A}_1$と$\bar{A}_2$の階数は$R$または$R+1$である。
$\bar{A}_1$の階数が$R$の場合、$\vb*{a}_1$の第$R+1$行目以降はすべて零なので、スカラー$c_1, \cdots, c_R$を使って$\vb*{a}_1 = c_1\vb*{e}_1+\cdots+c_R\vb*{e}_R$とできる。
よって、
$$
\vb*{a}_2 = P\vb*{a}_1 = c_1 P\vb*{e}_1+\cdots+c_R P\vb*{e}_R = c_1\vb*{e}_1+\cdots+c_R\vb*{e}_R = \vb*{a}_1
$$
である。
$\bar{A}_1 = P^{-1}\bar{A}_2$なので同様の議論をすれば$\bar{A}_2$の階数が$R$の場合も$\bar{A}_1 = \bar{A}_2$がわかる。
残るは$\bar{A}_1, \bar{A}_2$の階数がともに$R+1$の場合であるが、この時は$\vb*{a}_1 = \vb*{a}_2 = \vb*{e}_{R+1}$しかありえない。
以上より定理の証明が完成した。
\end{proof}

このことにより、任意の$M\times N$型の行列$A$に対して簡約行列$\bar{A}$が一意に定まり、
その簡約行列$\bar{A}$の階数をもって元の行列$A$の\emph{階数}として$\rank(A)$と表す。

\begin{remark}
本によっては上記の行に対する掃き出し法と同様にして列に対する掃き出し法や列簡約行列を導入して、
さらに行簡約行列かつ列簡約行列である行列、すなわち簡約行列の階数により行列の階数を定める。
この場合、階数$R$の簡約行列は
$$
\begin{pmatrix}I_R & O \\ O & O\end{pmatrix}
$$
と表される。
\end{remark}

\begin{proposition}[転置の階数]
行列$A$に対して、
$$
\rank(A^T) = \rank(A)
$$
が成り立つ。
\end{proposition}

\begin{proof}
$A^T$に対応する行簡約行列は$A$の列簡約行列の転置である。
$A$の行簡約行列の階数も列簡約行列の階数も、簡約行列の階数に等しいので、二つは一致する。
したがって、$\rank(A^T) = \rank(A)$がいえる。
\end{proof}

\begin{proposition}[正則行列の階数]
$N$次正方行列が正則であることはその階数が$N$であることと同値であり、
その際の行簡約行列は単位行列$I_N$である。
また、正則行列は基本行列の積になる。
\end{proposition}

\begin{proof}
$N$次正方行列を$A$、その行簡約行列を$\bar{A}$とする。
階数の定義から$A$の階数は$\bar{A}$の階数に等しく$N$以下であることに注意する。
$A$の階数が$N$でないとすると、$\bar{A}$の階数も$N$でなく$N-1$以下であり$\bar{A}$の第$N$行は零ベクトルとなる。
したがって$A$の第$N$行も零ベクトルとなりこれは正則でない。
以上より$A$が正則の時、$\rank(A) = N$である。
次に$A$の階数が$N$の時、$N$次正方行列の行簡約行列$\bar{A}$の階数も$N$であり、階数の定義より
$$
1 \le j(1) < j(2) < j(3) < \cdots < j(N) \le N
$$
がわかるので、$j(1) = 1, \cdots, j(N) = N$なので最終的に$\bar{A} = I_N$がわかる。
また$A$は基本行列の積になるので正則であり、逆に正則ならば基本行列の積になっている。
\end{proof}

\begin{proposition}[区分けと行簡約行列]
$A$を$M\times N$型の行列、$B$を$M\times L$型の行列とする。
$M\times (N+L)$型の行列$\tilde{A} = \begin{pmatrix}A & B\end{pmatrix}$の行簡約行列を$\begin{pmatrix}\bar{A} & \bar{B}\end{pmatrix}$とおくと、
$\bar{A}$は$A$の行簡約行列である。
さらに$\rank(\tilde{A}) \ge \rank(A)$である。
\end{proposition}

\begin{remark}
$\bar{B}$の方は$B$の行簡約行列とは限らない。
\end{remark}

この節の最後に第2章で触れた逆行列の同値な条件の証明を与える。

\begin{theorem}[逆行列の同値な条件]
$A, X$を$N$次正方行列とする。
ここで$A X = I_N$か$X A = I_N$のどちらか片方が成立したら、もう片方も成立して$X$は$A$の逆行列である。
\end{theorem}

\begin{proof}
議論は行と列を入れ替えれば同様なので、$X A = I_N$ならば$A X = I_N$を示す。
$N = 1, 2, 3, \cdots$についての数学的帰納法による。
$N = 1$の時は$N$次正方行列はスカラーにほかならなく、スカラーの乗法の交換法則より$A X = X A$なので成立する。
$N$次正方行列に対して成り立つとき、$N+1$次正方行列$A$を考える。
第$1$列が零ベクトルだと$X A = I_N$は成り立たないので、第$(k, 1)$成分が非零となるような第$k$行が存在し、第$1$行と第$k$行を入れ替えてから第$(1, 1)$成分について掃き出すことで
$$
P A = \begin{pmatrix}1 & \vb*{a} \\ \vb*{0} & \tilde{A}\end{pmatrix}
$$
となる$N+1$次の正則行列$P$が存在することがわかる。
ここで$X A = (X P^{-1})(P A)$より$X P^{-1} = \begin{pmatrix}x_{1 1} & \vb*{x}_{1 2} \\ \vb*{x}_{2 1} & \tilde{X}\end{pmatrix}$と区分けすると
$$
X A
= \begin{pmatrix}x_{1 1} & \vb*{x}_{1 2} \\ \vb*{x}_{2 1} & \tilde{X}\end{pmatrix}\begin{pmatrix}1 & \vb*{a} \\ \vb*{0} & \tilde{A}\end{pmatrix}
= \begin{pmatrix}x_{1 1} & x_{1 1}\vb*{a}+\vb*{x}_{1 2}\tilde{A} \\ \vb*{x}_{2 1} & \vb*{x}_{2 1}\vb*{a}+\tilde{X}\tilde{A}\end{pmatrix}
= \begin{pmatrix}1 & \vb*{0} \\ \vb*{0} & I_N\end{pmatrix}.
$$
よって、
$$
x_{1 1} = 1,
\quad x_{1 1}\vb*{a}+\vb*{x}_{1 2}\tilde{A} = \vb*{0},
\quad \vb*{x}_{2 1} = \vb*{0},
\quad \vb*{x}_{2 1}\vb*{a}+\tilde{X}\tilde{A} = I_N.
$$
特に$N$次正方行列$\tilde{A}, \tilde{X}$に対して$\tilde{X}\tilde{A} = I_N$なので、$\tilde{A}\tilde{X} = I_N$が成り立つ。
また、$\vb*{a} = -\vb*{x}_{1 2}\tilde{A}$であることに注意する。
したがって$A X$を計算すると
$$
A X
= \begin{pmatrix}1 & \vb*{a} \\ \vb*{0} & \tilde{A}\end{pmatrix}\begin{pmatrix}x_{1 1} & \vb*{x}_{1 2} \\ \vb*{x}_{2 1} & \tilde{X}\end{pmatrix}
= \begin{pmatrix}x_{1 1}+\vb*{a}\vb*{x}_{1 2} & \vb*{x}_{1 2}+\vb*{a}\tilde{X} \\ \tilde{A}\vb*{x}_{2 1} & \tilde{A}\tilde{X}\end{pmatrix}
= \begin{pmatrix}1 & \vb*{0} \\ \vb*{0} & I_N\end{pmatrix}.
$$
以上より数学的帰納法より定理が示された。
\end{proof}

\section{連立一次方程式の解}

連立一次方程式を解くことを考えると拡大係数行列に対する行基本変形は同値変形になる。
その結果として行簡約行列が得られたとすると実はそれで連立一次方程式はもう解かれた形になっていることがわかる。

つまり以下の流れで解くことができる。

\begin{enumerate}
\item
連立一次方程式が与えられる。
\item
対応する拡大係数行列を書く。
\item
行基本変形して行簡約行列を求める。
\item
対応する連立一次方程式を書く。
\item
解を得る。
\end{enumerate}

いくつかの具体例を見てみる。

\begin{example}[連立一次方程式の解法]
\end{example}

前の方で述べた通り解はただ一つに定まるとは限らず、解がない場合と複数ある場合が存在する。
複数ある場合は未知数のうちいくつかが自由に値を取れる状態になり解は無数にあることになる。

\begin{example}[解を持たない場合]
\end{example}

\begin{example}[解が無数にある場合]
\end{example}

これらの状況を一般化してまとめると以下の定理が得られる。

\begin{theorem}[連立一次方程式の解]
$K$上の$M\times N$型の行列$A$を係数行列、$M$次の縦ベクトル$\vb*{b}$を定数項ベクトルとして連立一次方程式$A\vb*{x} = \vb*{b}$を考える。
$\tilde{A} = \begin{pmatrix}A & \vb*{b}\end{pmatrix}$を拡大係数行列とする。
\begin{itemize}
\item
$\rank(\tilde{A}) > \rank(A)$の時、$A\vb*{x} = \vb*{b}$は解を持たない。
\item
$\rank(\tilde{A}) = \rank(A)$の時、$N$次のベクトル$\bar{\vb*{x}}$と$L = N-\rank(A)$個の$N$次のベクトル$\vb*{y}_1, \cdots \vb*{y}_L$が存在して、解は$L$個のスカラー$c_1, \cdots c_L \in K$を使って
$$
\vb*{x} = \bar{\vb*{x}}+c_1\vb*{y}_1+\cdots+c_L\vb*{y}_L
$$
と書ける。
特に$\rank(A) = N$の時、解はただ一つである。
\end{itemize}
\end{theorem}

定数項がすべて零である連立一次方程式
$$
A\vb*{x} = \vb*{0}
$$
を\emph{斉次連立一次方程式}という。
この場合拡大係数行列$\tilde{A} = \begin{pmatrix}A & \vb*{0}\end{pmatrix}$を行基本変形しても一番右の列は常に零ベクトルである。
そのため拡大係数行列を考える必要はなく係数行列$A$を行基本変形したので十分である。
特に$\rank(\tilde{A}) = \rank(A)$である。
よく考えてみると零ベクトル$\vb*{x} = \vb*{0}$は解なので解を持たないことはない。
そのため前の定理は斉次連立一次方程式に対しては以下のように単純になる。

\begin{theorem}[斉次連立一次方程式の解]
\label{t:homlinsys}
$K$上の$M\times N$型の行列$A$を係数行列として斉次連立一次方程式$A\vb*{x} = \vb*{0}$を考える。
このとき、$L = N-\rank(A)$個の$N$次のベクトル$\vb*{y}_1, \cdots \vb*{y}_L$が存在して、解は$L$個のスカラー$c_1, \cdots c_L \in K$を使って
$$
\vb*{x} = c_1\vb*{y}_1+\cdots+c_L\vb*{y}_L
$$
と書ける。
特に$\rank(A) = N$の時、解は零ベクトル$\vb*{0}$ただ一つである。
\end{theorem}

係数行列の等しい連立一次方程式と斉次連立一次方程式
$$
A\vb*{x} = \vb*{b},
\quad A\vb*{x} = \vb*{0}
$$
を考える。
ここでもし斉次とは限らない連立一次方程式$A\vb*{x} = \vb*{b}$の解$\bar{\vb*{x}}$が一つ見つかったとする。
このとき、斉次連立一次方程式$A\vb*{x} = \vb*{0}$の任意の解$\vb*{y}$とスカラー$c$に対して$\bar{\vb*{x}}+c\vb*{y}$は連立一次方程式$A\vb*{x} = \vb*{b}$の解である。
そのため前の定理によって斉次連立一次方程式$A\vb*{x} = \vb*{0}$の全ての解
$$
\vb*{x} = c_1\vb*{y}_1+\cdots+c_L\vb*{y}_L
$$
を得たとすると、連立一次方程式$A\vb*{x} = \vb*{b}$の全ての解は
$$
\vb*{x} = \bar{\vb*{x}}+c_1\vb*{y}_1+\cdots+c_L\vb*{y}_L
$$
となる。
つまり方程式の解を一つ見つけて斉次方程式を完全に解くと元の方程式も完全に解かれる。
このような解のことを\emph{特解}という。
連立一次方程式だと拡大係数行列を行簡約化する手間も係数行列を行簡約化する手間とあまり変わらないので斉次方程式のありがたみを感じないが、
微分方程式に対しても同様のことができしばしば重要になる。

\section{逆行列の計算}

この章の始めの方で述べたように$N$次正方行列$A$の逆行列を計算することは
$$
A X = I_N
$$
を満たす$N$次正方行列$X$を見つけることである。
正方行列$X$と単位行列$I_N$を
$$
X = \begin{pmatrix}\vb*{x}_1 & \cdots & \vb*{x}_N\end{pmatrix},
\quad I_N = \begin{pmatrix}\vb*{e}_1 & \cdots & \vb*{e}_N\end{pmatrix}
$$
と区分けする($\vb*{e}_1, \cdots, \vb*{e}_N$は$N$次の標準基底ベクトル)と、
$$
A\vb*{x}_1 = \vb*{e}_1, \cdots, A\vb*{x}_N = \vb*{e}_N
$$
となる。
これらは連立一次方程式になっていて、つまり逆行列を求めることは上記の$N$個の連立一次方程式を解くことに帰着される。

しかしながら$N$個の連立一次方程式を順番に解くことは効率が悪い、なぜなら拡大係数行列は$\begin{pmatrix}A & \vb*{e}_1\end{pmatrix}, \cdots, \begin{pmatrix}A & \vb*{e}_1\end{pmatrix}$であり係数行列が$A$で同じであるため同じ計算を何度もしてしまうためである。
そこで拡大係数行列をまとめて次の$N\times (N+N)$型の行列
$$
\begin{pmatrix}A & \vb*{e}_1 & \cdots & \vb*{e}_N\end{pmatrix} = \begin{pmatrix}A & I_N\end{pmatrix}
$$
を行簡約化する。
もしこの行列が$N$次正方行列$Y = \begin{pmatrix}\vb*{y}_1 & \cdots & \vb*{y}_N\end{pmatrix}$を使って
$$
\begin{pmatrix}I_N & Y\end{pmatrix} = \begin{pmatrix}I_N & \vb*{y}_1 & \cdots & \vb*{y}_N\end{pmatrix}
$$
と行簡約化されたとすると、各$i = 1, \cdots, N$に対して$\begin{pmatrix}A & \vb*{e}_i\end{pmatrix}$は$\begin{pmatrix}I_N & \vb*{y}_i\end{pmatrix}$に簡約化されるので、
連立一次方程式$A\vb*{x} = \vb*{e}_i$の解は$\vb*{x} = \vb*{y}_i$であり先述の議論により$Y$こそが$A$の逆行列であることがわかる。
一方でこの形に行簡約化されない場合は逆行列を持たないこともわかり、まとめると以下になる。

\begin{theorem}[逆行列の計算]
$A$を$N$次正方行列とする。
\begin{itemize}
\item
$\rank(A) = N$の時、$A$は正則であり、$N\times (N+N)$型の行列$\begin{pmatrix}A & I_N\end{pmatrix}$は$\begin{pmatrix}I_N & X\end{pmatrix}$の形に行簡約化されて、$X$は$A$の逆行列である。
\item
$\rank(A) \ne N$の時、$A$は正則でない。
\end{itemize}
\end{theorem}

\begin{remark}
さらに$P\begin{pmatrix}A & I_N\end{pmatrix} = \begin{pmatrix}I_N & X\end{pmatrix}$とおくと$X = P$なので、$A$の逆行列は基本行列の積として表される。
特にすべての正則行列は基本行列の積として表される。
\end{remark}

\section{区分けと基本変形}

行列の行基本変形は一行ずつ処理していくが、まとめて処理すると区分けされた行列の行簡約化がやりやすい場合がある。
そこで次の三つの変形を導入する。

\begin{itemize}
\item[(1)]
いくつかの行をまとめて入れ替える。
\item[(2)]
$R$個の行に$R$次の正則行列を左からかける。
\item[(3)]
ある$L$個の行に、別の$R$個の行に$L\times R$型の行列を左からかけたものを加える。
\end{itemize}

これらはいずれも行基本変形の繰り返しとして実現できる。
(1)と(3)は容易にわかるが、(2)は前節で述べた通り正則行列が基本行列の積として表されることにより証明できる。

\begin{example}
例\ref{t:blockinv}で見た通り$A$を$N_1$次の正則行列、$D$を$N_2$次の正則行列として$N_1+N_2$次の正方行列
$$
\begin{pmatrix}A & B \\ O & D\end{pmatrix}
$$
の逆行列は
$$
\begin{pmatrix}A^{-1} & -A^{-1}B D^{-1} \\ O & D^{-1}\end{pmatrix}
$$
で与えられる。
これを導出しようとすると上記の変形により
$$
\begin{pmatrix}A & B & I & O \\ O & D & O & I\end{pmatrix}
\to \begin{pmatrix}I & A^{-1}B & A^{-1} & O \\ O & D & O & I\end{pmatrix}
\to \begin{pmatrix}I & A^{-1}B & A^{-1} & O \\ O & I & O & D^{-1}\end{pmatrix}
\to \begin{pmatrix}I & O & A^{-1} & -A^{-1}B D^{-1} \\ O & I & O & D^{-1}\end{pmatrix}
$$
とできるからわかる。
\end{example}


\chapter{行列式}

\section{行列式の導入}

行列式は正方行列に対して定まる特徴的な量で、その正方行列の可逆性の判定や正方行列が定める線形変換による体積の拡大率に応用される重要な量である。

実は$2$次正方行列の行列式はすでに登場していて、逆行列の公式の係数の分母がそれである。
つまり$2$次正方行列$A = \begin{pmatrix}a & b \\ c & d\end{pmatrix}$の行列式の値は
$$
\det A
= \det\begin{pmatrix}a & b \\ c & d\end{pmatrix}
= \begin{vmatrix}a & b \\ c & d\end{vmatrix}
= a d-b c
$$
である。
このように行列式を表すためには$\det$を使ったり、行列を表すのに丸括弧を使う代わりに縦棒を使ったりして記述する。
この行列式の値が$0$でないことが逆行列を持つための必要十分条件になるのであった。
また、行列$A$を構成する$2$つのベクトル$\begin{pmatrix}a \\ c\end{pmatrix}$と$\begin{pmatrix}b \\ d\end{pmatrix}$のなす平行四辺形の面積を考えると、$\abs{a d-b c}$なので行列式の絶対値になっている。
また行列式の計算規則について、
$2$次の縦ベクトル$\vb*{a}, \vb*{a}', \vb*{b}, \vb*{b}'$とスカラー$c, c'$に対して多重線形性と呼ばれる等式
$$
\det\begin{pmatrix}c\vb*{a}+c'\vb*{a}' & \vb*{b}\end{pmatrix}
= c\det\begin{pmatrix}\vb*{a} & \vb*{b}\end{pmatrix}+c'\det\begin{pmatrix}\vb*{a}' & \vb*{b}\end{pmatrix},
\quad \det\begin{pmatrix}\vb*{a} & c\vb*{b}+c'\vb*{b}'\end{pmatrix}
= c\det\begin{pmatrix}\vb*{a} & \vb*{b}\end{pmatrix}+c'\det\begin{pmatrix}\vb*{a} & \vb*{b}'\end{pmatrix}
$$
が成り立つことと、
二つの$2$次正方行列$A$, $B$に対して積の行列式の等式
$$
\det(A B) = \det A \det B
$$
が成り立っていることがわかる。
これらは愚直に計算すればわかるが後で一般化された形で証明するので詳細は省略する。

以上に挙げた性質を$3$次以上の行列に拡張するが、その場合の行列式の定義はやや複雑で、置換を使う方法や余因子展開による方法、基本変形による方法などがある。
どの方法でも定義される行列式の値は同一なので、計算する場面に応じてどの方法を使うか選択したり組み合わせて使うことになる。
本テキストでは標準的な線形代数の教科書の流れに従って、まず置換を使う方法で行列式を定義してから、それが余因子展開を満たすことを示し基本変形による行列式の計算方法を紹介する。
そのために次節では置換というものについて定義する。

\section{置換と符号}

$N = 1, 2, 3, \cdots$として、$1, \cdots, N$の並べ替えを$N$次の\emph{置換}という。
より詳しくは$N$次の置換$s$は$N$個の元の集合$\{ 1, \cdots, N \}$から$\{ 1, \cdots, N \}$への写像であって逆写像$s^{-1}$を持つもの、
つまり各$i = 1, \cdots, N$に対して$s(i) = 1, \cdots, N$がただ一つ対応し$j = 1, \cdots, N$に対して$s(i) = j$となる$i$がただ一つ対応するので$i = s^{-1}(j)$とする。
$N$次の置換は$1, \cdots, N$の並べ替えなので$N!$個あることに注意して、
$N$次の置換全体の集合を$S_N$とおく。
置換$s$を表現するのにしばしば上に$1, \cdots, N$を並べて下に$s(1), \cdots, s(N)$を並べてそれらを丸括弧で括るという記法が採用され、上半分は$1, \cdots, N$で固定されるのでしばしば省略される。
つまり、
$$
s =
\begin{pmatrix}
1 & \cdots & N \\
s(1) & \cdots & s(N) \\
\end{pmatrix}
=
\begin{pmatrix}s(1) & \cdots & s(N)\end{pmatrix}
$$
である。
行列の記法と紛らわしいが文脈で判断する。
二つの置換$s$と$t$に対して置換$s$をしてから置換$t$をするという\emph{合成置換}を$t s$と書く。
$(t s)(i) = t(s(i))$である。
また、$1, \cdots, N$をそのままの並びにする置換を\emph{恒等置換}といい$\id_N$や$\id$で表す。
$$
\id_N =
\begin{pmatrix}
1 & \cdots & N \\
1 & \cdots & N \\
\end{pmatrix}
$$
である。
逆写像$s^{-1}$も置換であり置換$s$の\emph{逆置換}という。
$s s^{-1} = s^{-1} s = \id$に注意する。

$i \ne j$を満たす$i, j = 1, \cdots, N$に対して、$i$と$j$を入れ替えてそれ以外はそのままにする置換を$i$と$j$の\emph{互換}といい$\begin{pmatrix}i & j\end{pmatrix}$と表す。
つまり、
$$
\begin{pmatrix}i & j\end{pmatrix} =
\begin{pmatrix}
\cdots & i & \cdots & j \cdots \\
\cdots & j & \cdots & i \cdots \\
\end{pmatrix}
$$
である。
この互換は今までの内容で言うと、二つの行を入れ替えるという行基本変形に対応する。

任意の置換はいくつかの互換の合成として表される。

\begin{proposition}
$s$を$N$次の置換とする時、$s$は$N-1$個以下の互換の合成として表される。
ただし、$0$個の互換の合成は恒等置換$\id$である。
\end{proposition}

\begin{proof}
$N$に関する数学的帰納法で証明する。
$N = 1$の時は$S_1 = \{ \id \}$なので成立する。
$N$で成立する時、$N+1$次の置換$s$について考える。
$s(N+1) = N+1$の時は$s$を$1, \cdots, N$に制限すると$N$次の置換になっているので$N-1$個以下の互換の合成として表される。
$s(N+1) \ne N+1$の時は$i = s(N+1)$とすると$i = 1, \cdots, N$であり、$s$に$i$と$N+1$の互換をすると$N+1$を$N+1$に移すようになるので$N-1$個以下の互換の合成として表され、従って$s$は$N$個以下の互換の合成として表される。
以上より証明された。
\end{proof}

この命題において$N-1$個以下の部分はあまり重要でなく、
重要なのは$s \in S_N$に対して$s$を互換の合成として表した時の互換の個数の最小値$n(s)$が定まることである。
さらに$n(s)$が偶数の時$s$は\emph{偶置換}といい、奇数の時\emph{奇置換}と呼ぶことにする。
個数$n = 0, \cdots, N-1$に対して$n(s) = n$となる$s \in S_N$全体を$S_N(n)$と表し、
偶置換全体を$S_N^+$で奇置換全体を$S_N^-$でそれぞれ表す。
自明なこととして$S_N^+\cap S_N^- = \emptyset$と$S_N^+\cup S_N^- = S_N$が成り立つことがある。

\begin{example}
$N = 1$の時は
$$
S_1 = S_1(0) = \{ \begin{pmatrix}1\end{pmatrix} \}.
$$
$N = 2$の時は
$$
S_2 = \{ \begin{pmatrix}1 & 2\end{pmatrix}, \begin{pmatrix}2 & 1\end{pmatrix} \},
$$
$$
S_2(0) = \{ \begin{pmatrix}1 & 2\end{pmatrix} \},
\quad S_2(1) = \{ \begin{pmatrix}2 & 1\end{pmatrix} \}.
$$
$N = 3$の時は
$$
S_3 = \{ \begin{pmatrix}1 & 2 & 3\end{pmatrix}, \begin{pmatrix}2 & 1 & 3\end{pmatrix}, \begin{pmatrix}1 & 3 & 2\end{pmatrix}, \begin{pmatrix}3 & 2 & 1\end{pmatrix}, \begin{pmatrix}2 & 3 & 1\end{pmatrix}, \begin{pmatrix}3 & 1 & 2\end{pmatrix} \},
$$
$$
S_3(0) = \{ \begin{pmatrix}1 & 2 & 3\end{pmatrix} \},
\quad S_3(1) = \{ \begin{pmatrix}2 & 1 & 3\end{pmatrix}, \begin{pmatrix}1 & 3 & 2\end{pmatrix}, \begin{pmatrix}3 & 2 & 1\end{pmatrix} \},
\quad S_3(2) = \{ \begin{pmatrix}2 & 3 & 1\end{pmatrix}, \begin{pmatrix}3 & 1 & 2\end{pmatrix} \}.
$$
\end{example}

すぐわかることとして$S_N(0) = \{ \id_N \}$である。

以降では偶置換に対して正の符号を奇置換に対して負の符号を割り当てたいが、今の互換の個数の最小値に基づく定義では扱いづらいので別の方法でいったん符号を定義する。

\begin{definition}[置換の符号]
$N \ge 2$に対して$N$次の置換$s$の符号$\sgn(s)$を以下で定義する。
$$
\sgn(s) = \prod_{i < j}\frac{s(j)-s(i)}{j-i}.
$$
ここで$\prod_{i < j}\frac{s(j)-s(i)}{j-i}$は$i, j = 1, \cdots, N$が$i < j$を満たしながら動く時の実数$\frac{s(j)-s(i)}{j-i}$すべての積である。
そのため、$\sgn(s)$は実数値であるが、のちにすぐわかる通り$\sgn(s) = \pm 1$しか取り得ない。
$N = 1$の時は$\sgn(\id) = +1$と定義する。
\end{definition}

\begin{proposition}
任意の置換$s$の符号$\sgn(s)$は$+1$または$-1$である。
\end{proposition}

\begin{proof}
$\sgn(s)^2 = 1$を示せばよい。
計算すると
$$
\sgn(s)^2
= \prod_{i < j}\frac{s(j)-s(i)}{j-i}\prod_{i < j}\frac{s(j)-s(i)}{j-i}
= \prod_{i < j}\frac{s(j)-s(i)}{j-i}\prod_{i < j}\frac{s(i)-s(j)}{i-j}
= \prod_{i \ne j}\frac{s(i)-s(j)}{i-j}
= \frac{\prod_{i \ne j}(s(i)-s(j))}{\prod_{i \ne j}(i-j)}.
$$
ここで$s$は置換より、$i, j$が$i \ne j$を満たしながら動く時$s(i), s(j)$が$s(i) \ne s(j)$を満たしながら動くので、
$$
\sgn(s)^2
= \frac{\prod_{i \ne j}(i-j)}{\prod_{i \ne j}(i-j)}
= 1.
$$
よって$\sgn(s) = \pm 1$である。
\end{proof}

この符号の定義の利点は次の性質が証明しやすいことである。

\begin{proposition}
任意の二つの置換$s, t$に対して$\sgn(t s) = \sgn(t)\sgn(s)$が成り立つ。
\end{proposition}

\begin{proof}
計算すると
$$
\sgn(t s)
= \prod_{i < j}\frac{t(s(j))-t(s(i))}{j-i}
= \prod_{i < j}\frac{t(s(j))-t(s(i))}{s(j)-s(i)}\frac{s(j)-s(i)}{j-i}
= \sgn(s)\prod_{i < j}\frac{t(s(j))-t(s(i))}{s(j)-s(i)}.
$$
ここで、
$$
\begin{aligned}
\prod_{i < j}\frac{t(s(j))-t(s(i))}{s(j)-s(i)}
&= \prod_{i < j, s(i) < s(j)}\frac{t(s(j))-t(s(i))}{s(j)-s(i)}\prod_{i < j, s(i) > s(j)}\frac{t(s(j))-t(s(i))}{s(j)-s(i)} \\
&= \prod_{i < j, s(i) < s(j)}\frac{t(s(j))-t(s(i))}{s(j)-s(i)}\prod_{i < j, s(i) > s(j)}\frac{t(s(i))-t(s(j))}{s(i)-s(j)} \\
&= \prod_{i < j, s(i) < s(j)}\frac{t(s(j))-t(s(i))}{s(j)-s(i)}\prod_{j < i, s(j) > s(i)}\frac{t(s(j))-t(s(i))}{s(j)-s(i)} \\
&= \prod_{s(i) < s(j)}\frac{t(s(j))-t(s(i))}{s(j)-s(i)}
= \prod_{i < j}\frac{t(j)-t(i)}{j-i}
= \sgn(t)
\end{aligned}
$$
なので、$\sgn(t s) = \sgn(t)\sgn(s)$が従う。
\end{proof}

\begin{remark}
この命題から逆置換について
$$
\sgn(s)\sgn(s^{-1}) = \sgn(s s^{-1}) = \sgn(\id) = \prod_{i < j}\frac{j-i}{j-i} = +1
$$
なので、$\sgn(s^{-1}) = \sgn(s)$である。
\end{remark}

さらに互換の符号は常に負である。

\begin{proposition}
$N \ge 2$の時、互換$t$の符号は$\sgn(t) = -1$である。
\end{proposition}

\begin{proof}
$t = \begin{pmatrix}1 & 2\end{pmatrix}$のとき、
$$
\sgn(t)
= \frac{t(2)-t(1)}{2-1}\prod_{j = 3}^N\frac{t(j)-t(1)}{j-1}\prod_{j = 3}^N\frac{t(j)-t(2)}{j-2}\prod_{2 < i < j}^N\frac{t(j)-t(i)}{j-i}
= \frac{1-2}{2-1}\prod_{j = 3}^N\frac{j-2}{j-1}\prod_{j = 3}^N\frac{j-1}{j-2}
= -1.
$$
一般の互換$t = \begin{pmatrix}k & l\end{pmatrix}$に対しては、$\begin{pmatrix}l & k\end{pmatrix} = \begin{pmatrix}k & l\end{pmatrix}$より$k < l$としてよく、置換$s$を
$$
s =
\begin{cases}
\id & (k = 1, l = 2), \\
\begin{pmatrix}2 & l\end{pmatrix} & (k = 1, l > 2), \\
\begin{pmatrix}2 & l\end{pmatrix}\begin{pmatrix}1 & k\end{pmatrix} & (k > 1) \\
\end{cases}
$$
とすると$t = s^{-1}\begin{pmatrix}1 & 2\end{pmatrix}s$であるので、
$$
\sgn(t) = \sgn(s^{-1})\sgn(\begin{pmatrix}1 & 2\end{pmatrix})\sgn(s) = -1
$$
となる。
\end{proof}

以上のことから偶置換の符号は正で奇置換の符号は負であることがわかる。

\section{置換による行列式}

行列式は置換とその符号を用いて以下のように定義される。

\begin{definition}[行列式]
$K$上の$N$次正方行列$A = (a_{i j})$の\emph{行列式}を以下で定義する。
$$
\det A = \abs{A} =
\begin{vmatrix}
a_{1 1} & \cdots & a_{1 N} \\
\vdots & \ddots & \vdots \\
a_{N 1} & \cdots & a_{N N} \\
\end{vmatrix}
= \sum_{s \in S_N}\sgn(s) a_{1 s(1)}\cdots a_{N s(N)}.
$$
正確には$\sgn(s)$は実数の$\pm 1$として定義されたがこれを$K$の$\pm 1$と同一視して、
行列式は$K$の元つまりスカラーとして定義する。
\end{definition}

$N$が小さいうちはこの定義によって直接行列式を定義することができる。

\begin{proposition}[サラスの公式]
$1$次正方行列(スカラー)の行列式は以下になる。
$$
\begin{vmatrix}a\end{vmatrix}
= +a.
$$
$2$次正方行列の行列式は以下になる。
$$
\begin{vmatrix}a & b \\ c & d\end{vmatrix}
= +a d-b c.
$$
$3$次正方行列の行列式は以下になる。
$$
\begin{vmatrix}a & b & c \\ d & e & f \\ g & h & i\end{vmatrix}
= +a e i+b f g+c d h-c e g-b d i-a f h.
$$
\end{proposition}

これらの公式は行列の成分を斜めがけして方向によって符号を決めるという覚え方がある。
しかしながら$N$が大きくなると置換の個数が$N!$個あるため項数が爆発的に増えるのと、$N \ge 4$では覚え方が通用しない。
そこで行列式が満たす性質をうまく使って効率よく計算する必要がある。

\section{行列式の性質}

重要な行列式の性質として以下が挙げられる。

\begin{proposition}[転置]
$N$次正方行列$A$に対して、
$$
\det A^T = \det A
$$
が成り立つ。
\end{proposition}

\begin{proof}
変形すると
$$
\begin{aligned}
\det A^T
&= \sum_{s \in S_N}\sgn(s) a_{s(1) 1}\cdots a_{s(N) N}
= \sum_{s \in S_N}\sgn(s) a_{1 s^{-1}(1)}\cdots a_{N s^{-1}(N)} \\
&= \sum_{s \in S_N}\sgn(s^{-1}) a_{1 s(1)}\cdots a_{N s(N)}
= \sum_{s \in S_N}\sgn(s) a_{1 s(1)}\cdots a_{N s(N)} \\
&= \det A.
\end{aligned}
$$
\end{proof}

\begin{proposition}
第$1$行または第$1$列が第$(1, 1)$成分を残して他がすべて零の行列の行列式について、
$$
\begin{vmatrix}
a_{1 1} & 0       & \cdots & 0      \\
a_{2 1} & a_{2 2} & \cdots & a_{2 N}\\
\vdots  & \vdots  & \ddots & \vdots \\
a_{N 1} & a_{N 2} & \cdots & a_{N N}\\
\end{vmatrix}
=
\begin{vmatrix}
a_{1 1} & a_{1 2} & \cdots & a_{1 N}\\
0       & a_{2 2} & \cdots & a_{2 N}\\
\vdots  & \vdots  & \ddots & \vdots \\
0       & a_{N 2} & \cdots & a_{N N}\\
\end{vmatrix}
=
a_{1 1}
\begin{vmatrix}
a_{2 2} & \cdots & a_{2 N}\\
\vdots  & \ddots & \vdots \\
a_{N 2} & \cdots & a_{N N}\\
\end{vmatrix}
$$
が成り立つ。
\end{proposition}

\begin{proof}
左辺の行列について示せば、中辺の行列はその転置なので前命題より行列式の値は等しい。
その左辺の行列の行列式について、置換$s \in S_N$に対して$s(1) \ne 1$だと$a_{1 s(1)} = 0$である。
よって、$s(1) = 1$であり行列式で$a_{1 s(1)} = a_{1 1}$がくくり出せて命題が従う。
\end{proof}

この命題を繰り返し用いることで次の命題が従う。

\begin{proposition}[三角行列の行列式]
三角行列の行列式は対角成分を掛け合わせることで得られる。
つまり
$$
\begin{vmatrix}
a_{1 1} & \cdots & a_{1 N}\\
        & \ddots & \vdots \\
        &        & a_{N N}\\
\end{vmatrix}
=
\begin{vmatrix}
a_{1 1} &        &        \\
\vdots  & \ddots &        \\
a_{N 1} & \cdots & a_{N N}\\
\end{vmatrix}
= a_{1 1}\cdots a_{N N}
$$
が成り立つ。
特に単位行列について
$$
\det I_N = 1
$$
である。
\end{proposition}

次の二つの命題は行列式の計算をする上で重要である。

\begin{proposition}[多重線形性]
$N$次の横ベクトル$\vb*{a}_1, \cdots, \vb*{a}_N, \vb*{a}'_1, \cdots, \vb*{a}'_N$とスカラー$c_1, \cdots, c_N, c'_1, \cdots, c'_N$に対して、
$$
\begin{vmatrix}\vdots \\ c_i\vb*{a}_i+c'_i\vb*{a}'_i \\ \vdots\end{vmatrix}
= c_i\begin{vmatrix}\vdots \\ \vb*{a}_i \\ \vdots\end{vmatrix}+c'_i\begin{vmatrix}\vdots \\ \vb*{a}'_i \\ \vdots\end{vmatrix}
$$
が成り立つ。
また、$N$次の縦ベクトル$\vb*{a}_1, \cdots, \vb*{a}_N, \vb*{a}'_1, \cdots, \vb*{a}'_N$とスカラー$c_1, \cdots, c_N, c'_1, \cdots, c'_N$に対して、
$$
\begin{vmatrix}\cdots & c_i\vb*{a}_i+c'_i\vb*{a}'_i & \cdots\end{vmatrix}
= c_i\begin{vmatrix}\cdots & \vb*{a}_i & \cdots\end{vmatrix}+c'_i\begin{vmatrix}\cdots & \vb*{a}'_i & \cdots\end{vmatrix}
$$
が成り立つ。
\end{proposition}

\begin{proof}
成分を設定して行列式の定義に従って計算するとすぐわかるので詳細は省略する。
\end{proof}

\begin{proposition}[交代性]
$N$次の横ベクトル$\vb*{a}_1, \cdots, \vb*{a}_N$と$i \ne j$を満たす$i, j = 1, \cdots, N$に対して
$$
\begin{vmatrix}\vdots \\ \vb*{a}_j \\ \vdots \\ \vb*{a}_i \\ \vdots\end{vmatrix}
= -\begin{vmatrix}\vdots \\ \vb*{a}_i \\ \vdots \\ \vb*{a}_j \\ \vdots\end{vmatrix},
\quad \begin{vmatrix}\vdots \\ \vb*{a} \\ \vdots \\ \vb*{a} \\ \vdots\end{vmatrix} = 0
$$
が成り立つ。
また、$N$次の縦ベクトル$\vb*{a}_1, \cdots, \vb*{a}_N, \vb*{a}$と$i \ne j$を満たす$i, j = 1, \cdots, N$に対して
$$
\begin{vmatrix}\cdots & \vb*{a}_j & \cdots & \vb*{a}_i & \cdots\end{vmatrix}
= -\begin{vmatrix}\cdots & \vb*{a}_i & \cdots & \vb*{a}_j & \cdots\end{vmatrix},
\quad \begin{vmatrix}\cdots & \vb*{a} & \cdots & \vb*{a} & \cdots\end{vmatrix} = 0
$$
が成り立つ。
\end{proposition}

この命題の一つ目の等式は行を交換したら行列式の符号が変わることを述べ、二つ目の等式は二つの行が同じだったら行列式は零であることを述べている。
一見すると二つ目の式は一つ目の特別な場合であるが、$2 = 0$となるような体に配慮して二つ目を先に示し、そのことを利用して一つ目を示す。

\begin{proof}
転置すればよいので、前半部分だけ示す。
一つ目の等式を示すために$A = (a_{i, j})^{i = 1, \cdots, N}_{j = 1, \cdots, N}$は第$i$行と第$j$行が等しい、つまり$a_{i k} = a_{j k}$がすべての$k = 1, \cdots, N$に対して成り立つとすると、
$$
\det A
= \det A^T
= \sum_{s \in S_N}\sgn(s) a_{s(1) 1}\cdots a_{s(N) N}
= \sum_{s \in S_N^+}a_{s(1) 1}\cdots a_{s(N) N}-\sum_{s \in S_N^-}a_{s(1) 1}\cdots a_{s(N) N}.
$$
ここで奇置換の方にだけ互換$t = \begin{pmatrix}i & j\end{pmatrix}$の操作をすると、$A$の仮定から
$$
\sum_{s \in S_N^-}a_{s(1) 1}\cdots a_{s(N) N}
= \sum_{s \in S_N^-}a_{t(s(1)) 1}\cdots a_{t(s(N)) N}
= \sum_{s \in S_N^+}a_{s(1) 1}\cdots a_{s(N) N}.
$$
よって$\det A = 0$である。

二つ目を示すために第$i$行と第$j$行がともに$\vb*{a}_i+\vb*{a}_j$である行列の行列式を考えると多重線形性と先ほど示したことより、
$$
\begin{vmatrix}\vdots \\ \vb*{a}_i+\vb*{a}_j \\ \vdots \\ \vb*{a}_i+\vb*{a}_j \\ \vdots\end{vmatrix}
=
\begin{vmatrix}\vdots \\ \vb*{a}_i \\ \vdots \\ \vb*{a}_i \\ \vdots\end{vmatrix}
+\begin{vmatrix}\vdots \\ \vb*{a}_i \\ \vdots \\ \vb*{a}_j \\ \vdots\end{vmatrix}
+\begin{vmatrix}\vdots \\ \vb*{a}_j \\ \vdots \\ \vb*{a}_i \\ \vdots\end{vmatrix}
+\begin{vmatrix}\vdots \\ \vb*{a}_j \\ \vdots \\ \vb*{a}_j \\ \vdots\end{vmatrix}
= \begin{vmatrix}\vdots \\ \vb*{a}_i \\ \vdots \\ \vb*{a}_j \\ \vdots\end{vmatrix}
+\begin{vmatrix}\vdots \\ \vb*{a}_j \\ \vdots \\ \vb*{a}_i \\ \vdots\end{vmatrix}
$$
でこれが$0$に等しいので、証明すべき等式が得られる。
\end{proof}

\begin{proposition}[積の行列式]
\label{t:proddet}
二つの$N$次正方行列$A$と$B$に対して、
$$
\det(A B) = \det A\det B
$$
が成り立つ。
\end{proposition}

\begin{proof}
まず、積$A B$は成分を使って以下のように書くことができる。
$$
A B =
\begin{pmatrix}
a_{1 1} & \cdots & a_{1 N}\\
\vdots  & \ddots & \vdots \\
a_{N 1} & \cdots & a_{N N}\\
\end{pmatrix}
\begin{pmatrix}
\vb*{b}_1\\
\vdots   \\
\vb*{b}_N\\
\end{pmatrix}
=
\begin{pmatrix}
a_{1 1}\vb*{b}_1+\cdots+a_{1 N}\vb*{b}_N \\
\vdots   \\
a_{N 1}\vb*{b}_1+\cdots+a_{N N}\vb*{b}_N \\
\end{pmatrix}.
$$
よって、
$$
\det(A B) =
\begin{vmatrix}
\sum_{j_1 = 1}^N a_{1 j_i}\vb*{b}_{j_1} \\
\vdots   \\
\sum_{j_N = 1}^N a_{N j_N}\vb*{b}_{j_N} \\
\end{vmatrix}.
$$
多重線形性より、
$$
\det(A B) =
\sum_{j_1 = 1}^N \cdots \sum_{j_N = 1}^N a_{1 j_1}\cdots a_{N j_N}\begin{vmatrix}\vb*{b}_{j_1} \\ \vdots \\ \vb*{b}_{j_N}\end{vmatrix}.
$$
交代性より同じ行がある場合の行列式は$0$なので、
$$
\det(A B) =
\sum_{s \in S_N} a_{1 s(1)}\cdots a_{N s(N)}\begin{vmatrix}\vb*{b}_{s(1)} \\ \vdots \\ \vb*{b}_{s(N)}\end{vmatrix}.
$$
行を並べ替えて、
$$
\det(A B) =
\sum_{s \in S_N} \sgn(s)a_{1 s(1)}\cdots a_{N s(N)}\begin{vmatrix}\vb*{b}_1 \\ \vdots \\ \vb*{b}_N\end{vmatrix}.
$$
よって$\det(A B) = \det A\det B$である。
\end{proof}

\section{余因子展開}

$N$次正方行列$A$と$i, j = 1, \cdots, N$に対して、$A$の第$i$行と第$j$列を取り除いて得られる$N-1$次正方行列の行列式を$(-1)^{i+j}$倍した数を$A$の$(i, j)$\emph{余因子}という。
つまり、$A$を
$$
A =
\begin{pmatrix}
A_{U L} & * & A_{U R}\\
* & a_{i j} & * \\
A_{L L} & * & A_{L R}\\
\end{pmatrix}
$$
と区分けした時に
$$
\tilde{A}_{i j} =
(-1)^{i+j}
\begin{vmatrix}
A_{U L} & A_{U R}\\
A_{L L} & A_{L R}\\
\end{vmatrix}
$$
を定義する。

元の行列$A$の行列式は余因子を使って次のように表現される。

\begin{theorem}[余因子展開]
$N$次正方行列$A$と$i, j = 1, \cdots, N$に対して
$$
\det A
= a_{i 1}\tilde{A}_{i 1}+\cdots+a_{i N}\tilde{A}_{i N}
= a_{1 j}\tilde{A}_{1 j}+\cdots+a_{N j}\tilde{A}_{N j}
$$
が成り立つ。
\end{theorem}

この式の中で$i$を使った式を第$i$行についての余因子展開といい、$j$を使った式を第$j$列についての余因子展開という。

\begin{proof}
第$i$行についての余因子展開を示す。
第$i$行を
$$
\begin{pmatrix}a_{i 1} & a_{i 2} & \cdots & a_{i N}\end{pmatrix}
= a_{i 1}\begin{pmatrix}1 & 0 & \cdots & 0\end{pmatrix}+a_{i 2}\begin{pmatrix}0 & 1 & \cdots & 0\end{pmatrix}+a_{i N}\begin{pmatrix}0 & 0 & \cdots & 1\end{pmatrix}
$$
と分解すると、多重線形性より示される。
列についての余因子展開は転置を取ればよい。
\end{proof}

\begin{example}
\end{example}

\begin{remark}
\end{remark}

余因子は逆行列とも結びつく。
まず、余因子を次のように並べて得られる行列
$$
\tilde{A} =
\begin{pmatrix}
\tilde{A}_{1 1} & \cdots & \tilde{A}_{N 1} \\
\vdots & \ddots & \vdots \\
\tilde{A}_{1 N} & \cdots & \tilde{A}_{N N} \\
\end{pmatrix}
$$
を$A$の\emph{余因子行列}という。
並べ方が転置を取ったようになっているので注意。

\begin{theorem}[余因子行列と逆行列]
\label{t:cofactinv}
$N$次正方行列$A$とその余因子行列$\tilde{A}$について
$$
A \tilde{A} = \tilde{A} A = \det(A)I_N
$$
が成り立つ。
特に$A$が可逆であることと$\det A \ne 0$であることは同値であり、$A$の逆行列は
$$
A^{-1} = \frac{1}{\det A}\tilde{A}
$$
で与えられる。
\end{theorem}

この定理は逆行列を持つための必要十分条件を与えて非常に重要であるが、
逆行列の計算という観点からは行列式の値の計算が大変なので基本変形の方が一般に効率が良い。

証明のために次の補題を用意する。

\begin{lemma}
\label{t:cofactvec}
$A$を$N$次正方行列、$\tilde{A}$をその余因子行列として$\vb*{b}$を$N$次ベクトルとする。
この時、
$$
\tilde{A}\vb*{b}
= \begin{pmatrix}\det A_{1, \vb*{b}} \\ \vdots \\ \det A_{N, \vb*{b}}\end{pmatrix}
$$
が成り立つ。
ただし、$A_{k, \vb*{b}}$は行列$A = \begin{pmatrix}\vb*{a}_1 & \cdots & \vb*{a}_N\end{pmatrix}$の第$k = 1, \cdots, N$列をベクトル$\vb*{b}$で置き換えて得られる行列
$$
A_{k, \vb*{b}} = \begin{pmatrix}\vb*{a}_1 & \cdots & \vb*{a}_{k-1} & \vb*{b} & \vb*{a}_{k+1} & \cdots & \vb*{a}_N\end{pmatrix}
$$
である。
\end{lemma}

\begin{proof}
ベクトル$\tilde{A}\vb*{b}$の第$k$成分は
$$
\tilde{A}_{1 j}b_1+\cdots+\tilde{A}_{N j}b_N
$$
であり、これは$\det A_{k, \vb*{b}}$の第$k$列についての余因子展開に一致する。
\end{proof}

\begin{proof}[定理\ref{t:cofactinv}の証明]
$\tilde{A} A = \det(A)I_N$を示せば十分である。
$\tilde{A}A$の第$(i, j)$成分は、$A$の第$j$列を$\vb*{a}_j$とすると、$\tilde{A}\vb*{a}_j$の第$i$成分なので、補題\ref{t:cofactvec}より、$\det A_{i, \vb*{a}_j}$に等しい。
これは$j \ne i$の時は同じ列が二つあるので$0$であり、$j = i$の時は$\det A$に他ならない。
以上より証明される。
\end{proof}

この節の最後に行列式を用いた連立一次方程式の解の公式を紹介する。

\begin{theorem}[クラメルの公式]
$A$を$N$次正則行列つまり逆行列を持つとして、連立一次方程式
$$
A\vb*{x} = \vb*{b}
$$
の一意な解$\vb*{x} = \begin{pmatrix}x_1 \\ \vdots \\ x_N\end{pmatrix}$の成分$x_k$は
$$
x_k = \frac{\det A_{k, \vb*{b}}}{\det A}
$$
で与えられる。
\end{theorem}

\begin{proof}
$A$は逆行列を持ち、それは余因子行列を用いて$A = \frac{1}{\det A}\tilde{A}$で与えられるので、解は$\vb*{x} = \frac{1}{\det A}\tilde{A}\vb*{b}$である。
よって後は補題\ref{t:cofactvec}により計算される。
\end{proof}

\section{基本変形と行列式}

行列の基本変形は基本行列との積を取ることと考えられることと基本行列の行列式は簡単な計算で
$$
\det P_N(i, j) = -1,
\quad \det Q_N(i, c) = c,
\quad \det R_N(i, j, c) = 1
$$
と求まることから、基本変形をすると行列式の値は次のように変化する。
\begin{itemize}
\item[(1)]
二つの行(または列)を入れ替えると行列式の符号が反転する。
\item[(2)]
ある行(または列)を定数$c \in K$倍すると行列式の値が$c$倍になる。
\item[(3)]
ある行(または列)に、別の行(列)の定数倍を加えても行列式は変化しない。
\end{itemize}

基本変形して行列をより単純な形にすることで、行列式の計算をやりやすくすることができる。

\begin{example}
\end{example}

また、基本変形をまとめて処理すると行列式の値は次のように変化する。
\begin{itemize}
\item[(1)]
置換$s$に従っていくつかの行(または列)をまとめて入れ替えると行列式の値は$\sgn(s)$倍になる。
\item[(2)]
$R$個の行(または列)に$R$次の正則行列$A$を左からかけると行列式の値は$\det A$倍になる。
\item[(3)]
ある$L$個の行(または列)に、別の$R$個の行(列)に$L\times R$型の行列を左からかけたものを加えても行列式は変化しない。
\end{itemize}

ここで(2)は正則行列が基本行列の積として表されることから正当化される。

\begin{example}
\end{example}

\section{種々の行列式}

重要な行列式の公式として以下がある。

\begin{theorem}[ヴァンデルモンドの行列式]
$a_1, \cdots, a_N \in K$に対して、
$$
\mqty|
1 & a_1 & a_1^2 & \cdots & a_1^{N-1} \\
1 & a_2 & a_2^2 & \cdots & a_2^{N-1} \\
\vdots & \vdots & \vdots & & \vdots \\
1 & a_N & a_N^2 & \cdots & a_N^{N-1} \\
|
= \prod_{i < j}(a_j-a_i)
$$
が成り立つ。
\end{theorem}

\begin{proof}
$N$についての数学的帰納法で示す。
$N = 1$の時は両辺ともに$1$である。
$N-1$次で成立する時、第$(1, 1)$成分で第$1$行を掃き出すことで、
$$
\begin{aligned}
\mqty|
1 & a_1 & a_1^2 & \cdots & a_1^{N-1} \\
1 & a_2 & a_2^2 & \cdots & a_2^{N-1} \\
\vdots & \vdots & \vdots & & \vdots \\
1 & a_N & a_N^2 & \cdots & a_N^{N-1} \\
|
&=
\mqty|
1 & a_1 & a_1^2 & \cdots & a_1^{N-1} \\
0 & a_2-a_1 & a_2^2-a_1^2 & \cdots & a_2^{N-1}-a_1^{N-1} \\
\vdots & \vdots & \vdots & & \vdots \\
0 & a_N-a_1 & a_N^2-a_1^2 & \cdots & a_N^{N-1}-a_1^{N-1} \\
| \\
&=
\mqty|
a_2-a_1 & a_2^2-a_1^2 & \cdots & a_2^{N-1}-a_1^{N-1} \\
\vdots & \vdots & & \vdots \\
a_N-a_1 & a_N^2-a_1^2 & \cdots & a_N^{N-1}-a_1^{N-1} \\
| \\
&=
(a_2-a_1)\cdots(a_N-a_1)\mqty|
1 & a_2+a_1 & \cdots & a_2^{N-2}+a_2^{N-3}a_1\cdots+a_1^{N-2} \\
\vdots & \vdots & & \vdots \\
1 & a_N+a_1 & \cdots & a_N^{N-2}+a_N^{N-3}a_1\cdots+a_1^{N-2} \\
|.
\end{aligned}
$$
ここで最右辺の行列式は$N-1$次で、第$N-1$列から第$N-2$列の$a_1$倍を引き、第$N-2$列から第$N-3$列の$a_1$倍を引き、ということを続けると、
$$
\mqty|
1 & a_1 & a_1^2 & \cdots & a_1^{N-1} \\
1 & a_2 & a_2^2 & \cdots & a_2^{N-1} \\
\vdots & \vdots & \vdots & & \vdots \\
1 & a_N & a_N^2 & \cdots & a_N^{N-1} \\
|
=
(a_2-a_1)\cdots(a_N-a_1)\mqty|
1 & a_2 & \cdots & a_2^{N-2} \\
\vdots & \vdots & & \vdots \\
1 & a_N & \cdots & a_N^{N-2} \\
|.
$$
よって、数学的帰納法の仮定より、
$$
\mqty|
1 & a_1 & a_1^2 & \cdots & a_1^{N-1} \\
1 & a_2 & a_2^2 & \cdots & a_2^{N-1} \\
\vdots & \vdots & \vdots & & \vdots \\
1 & a_N & a_N^2 & \cdots & a_N^{N-1} \\
|
= (a_2-a_1)\cdots(a_N-a_1)\prod_{2 \le i < j}(a_j -a_i)
= \prod_{i < j}(a_j -a_i)
$$
を得る。
\end{proof}

\begin{proposition}[三重対角行列]
$a, b, c \in K$に対して$N$次正方行列の行列式
$$
D_N =
\mqty|
a & b & 0 & \cdots & 0 \\
c & a & b & \cdots & 0 \\
0 & c & a & \cdots & 0 \\
\vdots & \vdots & \vdots & & \vdots \\
0 & 0 & 0 & \cdots & a \\
|
$$
とおくと、
$$
D_{n+2} = a D_{n+1}-b c D_n, \qq{$D_1 = a$, $D_2 = a^2-b c$}
$$
が成り立つ。
ただし、この行列は対角成分が$a$でその右上成分が$b$で左下成分が$c$であり他は全て$0$となっている。
\end{proposition}

\begin{remark}
計算すると
$$
D_1 = a,
\quad D_2 = a^2-b c,
\quad D_3 = a^3-2 a b c,
\quad \cdots
$$
である。
\end{remark}

\begin{proof}
$D_1$, $D_2$はサラスの公式より成立する。
第$1$行に関する余因子展開をして、
$$
D_N =
a
\mqty|
a & b & \cdots & 0 \\
c & a & \cdots & 0 \\
\vdots & \vdots & & \vdots \\
0 & 0 & \cdots & a \\
|
-b
\mqty|
c & b & \cdots & 0 \\
0 & a & \cdots & 0 \\
\vdots & \vdots & & \vdots \\
0 & 0 & \cdots & a \\
|.
$$
さらに後ろの行列式は第$1$列に関する余因子展開をして、
$$
D_N =
a
\mqty|
a & b & \cdots & 0 \\
c & a & \cdots & 0 \\
\vdots & \vdots & & \vdots \\
0 & 0 & \cdots & a \\
|
-b c
\mqty|
a & \cdots & 0 \\
\vdots & & \vdots \\
0 & \cdots & a \\
|
= a D_{N-1}-b c D_{N-2}.
$$
よって主張が示された。
\end{proof}

\begin{example}
$a = 2, b = c = 1$の時、
$$
D_{n+2} = 2 D_{n+1}-D_n, \qq{$D_1 = 2$, $D_2 = 3$}
$$
であり、この漸化式を解いて
$$
D_N = N+1
$$
を得る。
\end{example}

\begin{proposition}
$x, a_1, \cdots, a_N \in K$に対して、
$$
\mqty|
x & -1 & 0 & \cdots & 0 & 0 \\
0 & x & -1 & \cdots & 0 & 0 \\
0 & 0 & x & \cdots & 0 & 0 \\
\vdots & \vdots & \vdots & & \vdots & \vdots \\
0 & 0 & 0 & \cdots & x & -1 \\
a_1 & a_2 & a_3 & \cdots & a_{N-1} & x+a_N \\
|
= x^N+a_N x^{N-1}+\cdots+a_1
$$
が成り立つ。
\end{proposition}

\begin{proof}
$N$についての数学的帰納法で示す。
$N = 1$の時は成立する。
$N-1$次で成立する時、$N$次を考える。
第$1$列についての余因子展開をすると、
$$
\begin{aligned}
\mqty|
x & -1 & 0 & \cdots & 0 \\
0 & x & -1 & \cdots & 0 \\
0 & 0 & x & \cdots & 0 \\
\vdots & \vdots & \vdots & & \vdots \\
0 & 0 & 0 & \cdots & -1 \\
a_1 & a_2 & a_3 & \cdots & x+a_N \\
|
&=
x
\mqty|
x & -1 & \cdots & 0 \\
0 & x & \cdots & 0 \\
\vdots & \vdots & & \vdots \\
0 & 0 & \cdots & -1 \\
a_2 & a_3 & \cdots & x+a_N \\
|
+(-1)^{N+1}a_1
\mqty|
-1 & 0 & \cdots & 0 \\
x & -1 & \cdots & 0 \\
0 & x & \cdots & 0 \\
\vdots & \vdots & & \vdots \\
0 & 0 & \cdots & -1 \\
| \\
&= x(x^{N-1}+a_N x^{N-2}+\cdots+a_2)+a_1 \\
&= x^N+a_N x^{N-1}+\cdots+a_1.
\end{aligned}
$$
よって、主張が示された。
\end{proof}


\chapter{線形空間}

\section{線形空間と例}

今後行列の理論を展開していく上で必要になってくるのが次元の概念でそれを導入するためこの章では抽象的な線形空間の概念を導入する。
スカラーの属する体などを$K$とする。
なお、「線形」という単語は「線型」や「一次」とも書かれたりする。

\begin{definition}[線形空間]
集合$V$に加法$+$($\vb*{u}, \vb*{v} \in V$に対して$\vb*{u}+\vb*{v} \in V$)とスカラー乗法($c \in K$と$\vb*{v} \in V$に対して$c\vb*{v} \in V$)が入っていて次を満たす時、$V$を$K$上の\emph{線形空間}という。
\begin{enumerate}
\item
(加法の結合法則)任意の$\vb*{u}, \vb*{v}, \vb*{w} \in V$に対して$(\vb*{u}+\vb*{v})+\vb*{w} = \vb*{u}+(\vb*{v}+\vb*{w})$が成り立つ。
\item
(加法の交換法則)任意の$\vb*{u}, \vb*{v} \in V$に対して$\vb*{u}+\vb*{v} = \vb*{v}+\vb*{u}$が成り立つ。
\item
(スカラー乗法の結合法則)任意の$c, d \in K$と$\vb*{v} \in V$に対して$(c d)\vb*{v} = c(d\vb*{v})$が成り立つ。
\item
(分配法則1)任意の$c \in K$と$\vb*{u}, \vb*{v} \in V$に対して$c(\vb*{u}+\vb*{v}) = c\vb*{u}+c\vb*{v}$が成り立つ。
\item
(分配法則2)任意の$c, d \in K$と$\vb*{v} \in V$に対して$(c+d)\vb*{v} = c\vb*{v}+d\vb*{v}$が成り立つ。
\item
(零ベクトル)$\vb*{0} \in V$がただ一つ存在して、任意の$\vb*{v} \in V$に対して$\vb*{v}+\vb*{0} = \vb*{0}+\vb*{v} = \vb*{v}$が成り立つ。
\item
(反ベクトル)任意の$\vb*{v} \in V$に対して$\vb*{v}+\vb*{x} = \vb*{x}+\vb*{v} = \vb*{0}$が成り立つような$\vb*{x} = -\vb*{v} \in V$がただ一つ存在する。
\item
(単位元)$1 \in K$は任意の$\vb*{v} \in K$に対して$1\vb*{v} = \vb*{v}$を満たす。
\end{enumerate}
線形空間$V$の元を\emph{ベクトル}と呼ぶ。
線形空間がいくつかある時$V$の零ベクトル$\vb*{0}$は$\vb*{0}_V$と表されたりする。
\end{definition}

この線形空間の定義は非常に抽象度が高くてさまざまな具体例が考えられる。

\begin{example}[自明な線形空間]
線形空間$V$は必ず零ベクトル$\vb*{0}$を元として含んでいるが、逆に$V$として零ベクトルのみからなる集合$O = \{ \vb*{0} \}$として加法を$\vb*{0}+\vb*{0} = \vb*{0}$、スカラー乗法を$c\vb*{0} = \vb*{0}$とすると$O$は線形空間である。
このような線形空間$O$を自明な線形空間という。
\end{example}

\begin{example}[スカラーの空間]
$V$を$K$として設定し加法を$K$での加法、スカラー乗法を$K$での乗法によって定めると$V$は$K$上の線形空間である。
\end{example}

\begin{example}[数ベクトル空間]
$N = 1, 2, 3, \cdots$とする。
$V$として直積$K^N$として設定するつまり$N$個の$K$の元$a_1, \cdots, a_N$を並べたもの$\vb*{v} = (a_1, \cdots, a_N)$の集合として、加法とスカラー乗法を
$$
(a_1, \cdots, a_N)+(b_1, \cdots, b_N) = (a_1+b_1, \cdots, a_N+b_N),
\quad c(a_1, \cdots, a_N) = (c a_1, \cdots, c a_N)
$$
によって定義すると$V$は$K$上の線形空間である。
この$K^N$を$K$上の$N$次元\emph{数ベクトル空間}といい、その元を$N$次元\emph{数ベクトル}という。
数ベクトルは縦ベクトルや横ベクトルと同一視されうる。
$N = 1$の時の$K^1$は先述の線形空間としての$K$と同一視される。
また、拡大解釈して$N = 0$の時の$K^0$が自明な線形空間$O$と見なすことができる。
\end{example}

\begin{example}[行列空間]
同様にして$M, N = 1, 2, 3, \cdots$とする。
$V$として$M\times N$型の$K$上の行列の集合$\M_{M\times N}(K) = K^{M\times N}$とすると$V$は$K$上の線形空間である。
\end{example}

\begin{example}[数列空間]
$V$を$K$の元を並べた数列$(a_n)_{n \in \mathbb{N}}$全体$V = K^\mathbb{N}$として加法とスカラー乗法を
$$
(a+b)_n = a_n+b_n,
\quad (c a)_n = c a_n
$$
とすると$V$は$K$上の線形空間である。
\end{example}

\begin{example}[多項式空間]
$K$の元を係数とする多項式$f(x)$の集合$V = K[x]$は加法とスカラー乗法を
$$
(f+g)(x) = f(x)+g(x),
\quad (c f)(x) = c f(x)
$$
とすると$V$は$K$上の線形空間である。
\end{example}

\begin{example}[関数空間]
同様にして$V$を集合$X$から$K$への関数$f(x)$全体$V = K^X$として加法とスカラー乗法を
$$
(f+g)(x) = f(x)+g(x),
\quad (c f)(x) = c f(x)
$$
とすると$V$は$K$上の線形空間である。
\end{example}

以下は一見すると線形空間に思えないが、定義上線形空間になる例である。

\begin{example}
実数は和と有理数倍が定義できるので、$\mathbb{R}$は有理数体$\mathbb{Q}$上の線形空間である。
\end{example}

\begin{example}
正の実数$\mathbb{R}_+$の「和」を$x\boxplus y = x y$で「実数$c$倍」を$c\boxdot x = x^c$で定義すると、
$\mathbb{R}_+$は実数体$\mathbb{R}$上の線形空間である。
なお、この時の「零ベクトル」は$x = 1$である。
\end{example}

\begin{definition}[線形部分空間]
$K$上の線形空間$V$に対して、その部分集合$W$であって$V$の加法とスカラー乗法について線形空間になっているものを$V$の\emph{線形部分空間}という。
つまり、次の条件を満たせばよい。
\begin{enumerate}
\item
任意の$\vb*{v}, \vb*{w} \in W$に対して$\vb*{v}+\vb*{w} \in W$が成り立つ。
\item
任意の$c \in K$と$\vb*{v} \in W$に対して$c\vb*{v} \in W$が成り立つ。
\end{enumerate}
この二つの条件はまとめて、任意の$c, d \in K$と$\vb*{v}, \vb*{w} \in W$に対して$c\vb*{v}+d\vb*{w} \in W$とも書ける。
本テキストでは$W$が$V$の線形部分空間であることを部分集合の記号を使って$W \subset V$と表すことがある。
\end{definition}

線形空間$V$に対して、$V$自身と零ベクトルのみからなる集合$O = O_V = \{ \vb*{0}_V \}$はいずれも$V$の線形部分空間であり、自明な線形部分空間と呼ばれる。

\begin{example}[連立一次方程式の解空間]
$N, M = 1, 2, 3, \cdots$として$A$を$M\times N$型の$K$上の行列とする。
この時、斉次連立一次方程式
$$
A\vb*{x} = \vb*{0}
$$
の解$\vb*{x} \in K^N$全体からなる集合は$K^N$の線形部分空間である。
\end{example}

\begin{example}[行列空間の部分空間]
$N = 1, 2, 3, \cdots$次の対角行列、右上三角行列、左下三角行列、対称行列の集合はいずれも$\M_N(K)$の線形部分空間である。
\end{example}

\begin{example}[有限次多項式空間]
$K$の元を係数とする$N = 0, 1, 2, 3, \cdots$次以下の多項式
$$
f(x) = a_0+a_1 x+a_2 x^2+a_3 x^3+\cdots+a_N x^N,
\quad a_0, a_1, a_2, a_3, \cdots, a_N \in K
$$
の集合$K^N[x]$は$K[x]$の線形部分空間である。
\end{example}

\begin{example}
$\mathbb{R}$の部分集合
$$
\mathbb{Q}[\sqrt{2}] = \lrset{ a+b\sqrt{2} \mid a, b \in \mathbb{Q} }
$$
は有理数体$\mathbb{Q}$上の線形空間としての$\mathbb{R}$の線形部分空間である。
\end{example}

一般論で重要な線形部分空間として次がある。

\begin{definition}[線形結合]
$V$を$K$上の線形空間として、いくつかの元(ベクトル)$\vb*{v}_1, \cdots, \vb*{v}_N$を考える。
ここでスカラー$c_1, \cdots, c_N \in K$を使って
$$
c_1 \vb*{v}_1+\cdots+c_N \vb*{v}_N
$$
と表される$V$の元(ベクトル)を$\vb*{v}_1, \cdots, \vb*{v}_N$の\emph{線形結合}という。
また、$\vb*{v}_1, \cdots, \vb*{v}_N$の線形結合全体からなる集合
$$
\lrangle{ \vb*{v}_1, \cdots, \vb*{v}_N } = \lrset{ c_1 \vb*{v}_1+\cdots+c_N \vb*{v}_N \mid c_1, \cdots, c_N \in K }
$$
は$V$の線形部分空間であり、$\vb*{v}_1, \cdots, \vb*{v}_N$が\emph{張る}または\emph{線形生成する}線形部分空間と呼ばれる。
また、$N = 0$の時は線形結合は零ベクトルとして、張る線形部分空間は自明な$O$と考える。
\end{definition}

また、線形部分空間の演算として次がある。

\begin{definition}[交わりと和]
$U$を$K$上の線形空間として、$V, W$を$U$の部分空間とする。
\begin{itemize}
\item
共通部分$V\cap W = \lrset{ \vb*{v} \in U \mid \vb*{v} \in V, \vb*{v} \in W }$は$U$の線形部分空間であり、$V$と$W$の\emph{交空間}と呼ばれる。
\item
$V+W = \lrset{ \vb*{v}+\vb*{w} \mid \vb*{v} \in V, \vb*{w} \in W }$は$U$の線形部分空間であり、$V$と$W$の\emph{和空間}と呼ばれる。
\item
交空間が$V\cap W = O_U$を満たす時、和空間$V+W$は$V\oplus W$と書かれ、$V$と$W$の\emph{直和空間}と呼ばれる。
\end{itemize}
\end{definition}

\begin{remark}
和空間は和集合$V\cup W$とは違うことに注意する。
和集合$V\cup W$は一般に線形空間ではない。
\end{remark}

次は集合でいう補集合に相当する。

\begin{definition}[線形補空間]
$V$を$K$上の線形空間として、$W$を$V$の線形部分空間とする。
ここで$V$の線形部分空間$U$であって$W\oplus U = V$、つまり$W\cap U = O_V$かつ$W+U = V$を満たすものを$W$の\emph{線形補空間}という。
\end{definition}

\section{次元と基底}

次元は線形空間を特徴づける(拡大)自然数である。

\begin{definition}[次元]
$V$を線形空間として、いくつかのベクトル$\vb*{v}_1, \cdots, \vb*{v}_N$ ($N = 0, 1, 2, 3, \cdots$)が存在して
$$
\lrangle{ \vb*{v}_1, \cdots, \vb*{v}_N } = V
$$
つまり$\vb*{v}_1, \cdots, \vb*{v}_N$が$V$を線形生成するとする時、$V$は\emph{有限次元}であるという。
有限次元でない線形空間は\emph{無限次元}であるという。
線形空間の\emph{次元}を以下で定めて$\dim V$と表す。
有限次元線形空間$V$に対して上記の$N$の最小値が存在するのでそれを$V$の次元とする。
無限次元線形空間$V$に対しては$V$の次元は$\infty$とする。
\end{definition}

すぐわかることとして自明な線形空間$O$の次元は$0$であり、次元が$0$の線形空間は$O$しかない。
また、後でわかることとして
$$
\dim K^N = N,
\quad \dim K^{M\times N} = M N,
\quad \dim K^{\mathbb{N}} = \infty,
\quad \dim K[x] = \infty,
\quad \dim K^X = \# X,
\quad \dim K^N[x] = N+1
$$
である。
ただし$\# X$は集合$X$の元の個数(有限集合でない場合は$\# X = \infty$)である。
$N$個のベクトル$\vb*{v}_1, \cdots, \vb*{v}_N \in V$が$V$を線形生成する時、$N \ge \dim V$であることに注意する。

\begin{definition}[基底]
$V$を有限次元線形空間として$N$をその次元とする。
ここで$N$個のベクトル$\vb*{v}_1, \cdots, \vb*{v}_N \in V$であって
$$
\lrangle{ \vb*{v}_1, \cdots, \vb*{v}_N } = V
$$
を満たすものを$V$の\emph{基底}という。
\end{definition}

次元の定義から有限次元線形空間には基底が存在するが、無限次元線形空間には(本テキストでは)そもそも基底を導入しない。
また、有限次元線形空間の次元は一意的であるが、基底は一意ではない(例としては$K^2$に対して$(1, 0), (0, 1)$と$(1, 1), (1, -1)$はそれぞれ基底である)。

\begin{definition}[線形独立]
$V$を$K$上の線形空間として、いくつかのベクトル$\vb*{v}_1, \cdots, \vb*{v}_N \in V$を考える。
ここでスカラー$c_1, \cdots, c_N \in K$であって
$$
c_1 \vb*{v}_1+\cdots+c_N \vb*{v}_N = \vb*{0}
$$
となるのは$c_1 = \cdots = c_N = 0$の場合のみとする時、
$\vb*{v}_1, \cdots, \vb*{v}_N$は\emph{線形独立}であるという。
$N = 0$の時は$0$個のベクトルは常に線形独立していると理解する。
線形独立でない時、\emph{線形従属}しているという。
\end{definition}

\begin{example}
$N = 0, 1, 2, 3, \cdots$として$1, x, x^2, x^3, \cdots, x^N \in K[x]$は線形独立である。
\end{example}

基底と線形独立性には関係がある。

\begin{proposition}
$K$上の線形空間$V$の基底は線形独立である。
\end{proposition}

\begin{proof}
$V$の次元を$N \ge 1$、基底を$\vb*{v}_1, \cdots, \vb*{v}_N \in V$とする。
もし$\vb*{v}_1, \cdots, \vb*{v}_N$が線形独立でないつまり線形従属している時は、$c_1 = \cdots = c_N = 0$でないスカラー$c_1, \cdots, c_N \in K$であって
$$
c_1 \vb*{v}_1+\cdots+c_N \vb*{v}_N = \vb*{0}
$$
となるものが存在する。
ここで$c_i \ne 0$となる$i = 1, \cdots, N$が存在するが、$\vb*{v}_1, \cdots, \vb*{v}_N$を並べ替えて、$i = N$の場合のみ考えればよい。
この時、
$$
\vb*{v}_N = -c_N^{-1}(c_1 \vb*{v}_1+\cdots+c_{N-1} \vb*{v}_{N-1})
$$
となるため、$\vb*{v}_N$は$\vb*{v}_1, \cdots, \vb*{v}_{N-1}$の線形結合として書け、$\vb*{v}_1, \cdots, \vb*{v}_{N-1}$は$V$を線形生成する。
実際、$\vb*{v}_1, \cdots, \vb*{v}_N$が$V$を線形生成するため任意の$w \in V$は$\vb*{v}_1, \cdots, \vb*{v}_N$の線形結合として書けるので、$d_1, \cdots, d_N \in K$を使って
$$
w = d_1\vb*{v}_1+\cdots+d_N\vb*{v}_N
= d_1\vb*{v}_1+\cdots+d_{N-1}\vb*{v}_{N-1}-d_N c_N^{-1}(c_1 \vb*{v}_1+\cdots+c_{N-1} \vb*{v}_{N-1})
$$
より$\vb*{v}_1, \cdots, \vb*{v}_{N-1}$は$V$を線形生成する。
ここで次元の定義より$N$は$V$を線形生成するベクトルの個数の最小だったが、$N-1$個でも線形生成されてしまったので矛盾である。
従って、基底$\vb*{v}_1, \cdots, \vb*{v}_N$は線形独立している。
\end{proof}

逆に線形独立で$V$を線形生成するならばそれは$V$の基底である。
まず次を示す。

\begin{lemma}[線形独立と次元]
$V$を$K$上の線形空間として、$N = 0, 1, 2, 3, \cdots$個のベクトル$\vb*{v}_1, \cdots, \vb*{v}_N \in V$が線形独立ならば$N \le \dim V$である。
\end{lemma}

この証明のために第2章で学んだ連立一次方程式の理論を用いる。

\begin{proof}
$M = \dim V$として$M, N = 1, 2, 3, \cdots$の場合を考えればよい。
つまり、$V$は次元が$M$の有限次元である。
そこで$V$の基底$\vb*{w}_1, \cdots, \vb*{w}_M$を取ってくる。
各$j = 1, \cdots, N$に対して$\vb*{v}_i \in V$は$\vb*{w}_1, \cdots, \vb*{w}_M$の線形結合として表されるので、
$$
\vb*{v}_j = c_{j 1}\vb*{w}_1+\cdots+c_{j M}\vb*{w}_M
$$
となる$c_{j 1}, \cdots, c_{j M} \in K$が存在する。
ここで$\vb*{v}_1, \cdots, \vb*{v}_N$は線形独立なので、$d_1, \cdots, d_N \in K$が$d_1\vb*{v}_1+\cdots+d_N\vb*{v}_N = \vb*{0}$を満たすならば$d_1 = \cdots = d_N = 0$である。
$$
\begin{aligned}
d_1\vb*{v}_1+\cdots+d_N\vb*{v}_N
&= d_1(c_{1 1}\vb*{w}_1+\cdots+c_{1 M}\vb*{w}_M)+\cdots+d_N(c_{N 1}\vb*{w}_1+\cdots+c_{N M}\vb*{w}_M) \\
&= (d_1 c_{1 1}+ d_N c_{N 1})\vb*{w}_1+\cdots+(d_1 c_{1 M}+ d_N c_{N M})\vb*{w}_M
\end{aligned}
$$
なので、$\vb*{w}_1, \cdots, \vb*{w}_M$が線形独立であることから
$$
d_1 c_{1 1}+ d_N c_{N 1} = \cdots = d_1 c_{1 M}+ d_N c_{N M} = 0
$$
で、これを変形して斉次連立一次方程式
$$
\begin{pmatrix}
c_{1 1} & \cdots & c_{N 1} \\
\vdots & \ddots & \vdots \\
c_{1 M} & \cdots & c_{N M} \\
\end{pmatrix}
\begin{pmatrix}
d_1 \\
\vdots \\
d_N \\
\end{pmatrix}
=
\begin{pmatrix}
0 \\
\vdots \\
0 \\
\end{pmatrix}
$$
を得る。
ここで定理\ref{t:homlinsys}より係数行列の階数を$R$とすると、この方程式の解が$d_1 = \cdots = d_N = 0$に限られる条件は$R = N$であることである。
よって、係数行列の列数は$M$であることから$N = R \le M = \dim V$を得る。
\end{proof}

\begin{proposition}
$V$を線形空間として、いくつかのベクトル$\vb*{v}_1, \cdots, \vb*{v}_N \in V$が線形独立かつ$V$を線形生成するならば、$N = \dim V$であり$\vb*{v}_1, \cdots, \vb*{v}_N$は$V$の基底である。
\end{proposition}

\begin{proof}
$\vb*{v}_1, \cdots, \vb*{v}_N$は線形独立なので$N \le \dim V$で、
$V$を線形生成するので$N \ge \dim V$である。
したがって$N = \dim V$であり、基底の定義から$\vb*{v}_1, \cdots, \vb*{v}_N$は$V$の基底である。
\end{proof}

\begin{proposition}[線形補空間]
\label{t:complement}
$V$を$K$上の有限次元線形空間とする。
このとき$V$の任意の線形部分空間$W$に対して線形補空間$U$が存在する。
\end{proposition}

証明は$W$の基底を取ってきて、そこに$V$の元をいくつか足して$V$の基底を作るという基底の延長と呼ばれる手法で行う。
そのために次の補題を用意する。

\begin{lemma}
$K$上の線形空間$V$のいくつかのベクトル$\vb*{v}_1, \cdots, \vb*{v}_N \in V$が線形独立で$V$を生成しないとき、任意の$\vb*{w} \in V\setminus\lrangle{ \vb*{v}_1, \cdots, \vb*{v}_N }$に対して$\vb*{v}_1, \cdots, \vb*{v}_N, \vb*{w}$は線形独立である。
\end{lemma}

\begin{proof}
スカラー$c_1, \cdots, c_N, d \in K$を使って$c_1\vb*{v}_1+\cdots+c_N\vb*{v}_N+d\vb*{w} = \vb*{0}_V$となったとする。
ここで$d \ne 0$とすると
$$
\vb*{w} = -d^{-1}(c_1\vb*{v}_1+\cdots+c_N\vb*{v}_N)
$$
となり仮定に反する。
よって$d = 0$であり、$\vb*{v}_1, \cdots, \vb*{v}_N$が線形独立なので$c_1 = \cdots = c_N = 0$である。
以上より$\vb*{v}_1, \cdots, \vb*{v}_N, \vb*{w}$は線形独立である。
\end{proof}

この補題の応用として次が示される。

\begin{proposition}
$V$を$K$上の有限次元線形空間として、$N = \dim V$とおく。
ここで$N$個のベクトル$\vb*{v}_1, \cdots, \vb*{v}_N \in V$が線形独立なとき、それは$V$の基底である。
\end{proposition}

\begin{proof}
$V$を生成しないとすると、補題より$N+1$個の線形独立なベクトルを得るが、これは補題に反する。
よって、$\vb*{v}_1, \cdots, \vb*{v}_N$は$V$を生成するので基底である。
\end{proof}

\begin{proof}[命題の証明]
有限次元なので$V$の次元を$N$、$W$の次元を$M$として、$W$の基底$\vb*{w}_1, \cdots, \vb*{w}_M$を取る。
ここで補題を$N-M$回繰り返すことで線形独立な$\vb*{w}_1, \cdots, \vb*{w}_M, \vb*{u}_{M+1}, \cdots, \vb*{u}_N \in V$を作ることができ(次元の関係で補題の仮定を満たし続ける)、次元の関係でこれは$V$の基底になる。
あとは$U = \lrangle{ \vb*{u}_{M+1}, \cdots, \vb*{u}_N }$とすればよい。
\end{proof}

和空間の次元について次が成り立つ。

\begin{theorem}[和空間の次元]
$K$上の線形空間$U$の線形部分空間$V, W$について次が成り立つ。
$$
\dim(V+W)+\dim(V\cap W) = \dim V+\dim W.
$$
特に
$$
\dim(V\oplus W) = \dim V+\dim W
$$
である。
\end{theorem}

\begin{proof}
まず、定義からすぐわかることとして
$$
\max\lrset{ \dim V, \dim W } \le \dim(V+W) \le \dim V+\dim W,
\quad \dim(V\cap W) \le \min\lrset{ \dim V, \dim W }
$$
がある(詳細省略)。
これにより$V$または$W$が無限次元の時は$V+W$も無限次元となり等式は成立する。
よって、以降では$V$と$W$がともに有限次元の場合を考えればよく、$V+W$と$V\cap W$も有限次元である。
$N = \dim V$, $M = \dim W$とおく。

定理の後半部分の内容を先に示しておく。
つまり$V\cap W = O_U$の時、$V$の基底$\vb*{v}_1, \cdots, \vb*{v}_N$と$W$の基底$\vb*{w}_1, \cdots, \vb*{w}_M$を取ってくると$\vb*{v}_1, \cdots, \vb*{v}_N, \vb*{w}_1, \cdots, \vb*{w}_M$が線形独立を示せばよい。
スカラー$c_1, \cdots, c_N, d_1, \cdots, d_M$について
$$
c_1\vb*{v}_1+\cdots+c_N\vb*{v}_N+d_1\vb*{w}_1+\cdots+d_M\vb*{w}_M = \vb*{0}_U
$$
の時、
$$
c_1\vb*{v}_1+\cdots+c_N\vb*{v}_N = -d_1\vb*{w}_1-\cdots-d_M\vb*{w}_M
$$
であり、仮定からこれは$V\cap W = O_U$の元より$\vb*{0}_U$に等しく、
$\vb*{v}_1, \cdots, \vb*{v}_N$と$\vb*{w}_1, \cdots, \vb*{w}_M$の線形独立性から$\vb*{v}_1, \cdots, \vb{v}_N, \vb*{w}_1, \cdots, \vb*{w}_M$が線形独立がわかる。
よって、この場合$\dim(V+W) = \dim V+\dim W$である。

一般の場合を示す。
$V\cap W$は有限次元線形空間$W$の線形部分空間なので、命題\ref{t:complement}より、線形補空間が存在しそれぞれ$W'$とおく。
ここで$V+W = V\oplus W'$を示す。
まず$v \in V\cap W'$とすると特に$v \in V\cap W$なので$v \in (V\cap W)\cap W' = O_U$である。
次に$v+w \in V+W$とすると$w = u+w'$, $u \in V\cap W$, $w' \in W'$とでき、$v+w = (v+u)+w' \in V+W'$である。
以上より$V+W = V\oplus W'$かつ$W = (V\cap W)\oplus W'$なので、先ほど示したことから
$$
\dim(V+W) = \dim V+\dim W',
\quad \dim W = \dim(V\cap W)+\dim W'.
$$
よって、定理の主張を得る。
\end{proof}

\section{線形写像}

集合に線形性が導入されるように写像にも線形性を導入する。

\begin{definition}[線形写像]
$V$と$W$を$K$上の線形空間とする。
$V$から$W$への写像$F$であって以下を満たすものを$V$から$W$への\emph{線形写像}という。
\begin{enumerate}
\item
任意の$\vb*{u}, \vb*{v} \in V$に対して$F(\vb*{u}+\vb*{v}) = F(\vb*{u})+F(\vb*{v})$が成り立つ。
\item
任意の$c \in K$と$\vb*{v} \in V$に対して$F(c\vb*{v}) = c F(\vb*{v})$が成り立つ。
\end{enumerate}
この二つの条件はまとめて、任意の$c, d \in K$と$\vb*{v}, \vb*{w} \in W$に対して$F(c\vb*{v}+d\vb*{w}) = c F(\vb*{v})+d F(\vb*{w})$とも書ける。
\end{definition}

線形写像に対してはしばしば丸括弧を省略して$F\vb*{v}$と記述される。

\begin{example}
$\mathbb{R}$を$\mathbb{R}$上の線形空間と考えて関数$f: \mathbb{R} \to \mathbb{R}$を考える。
\begin{itemize}
\item
$f(x) = x$は線形写像である。
\item
$f(x) = x+1$は線形写像でない。
\item
$f(x) = x^2$は線形写像でない。
\end{itemize}
\end{example}

線形写像の合成は線形写像である。

\begin{proposition}[合成線形写像]
$U, V, W$を$K$上の線形空間として、$F$を$U$から$V$への線形写像とし$G$を$V$から$W$への線形写像とする。
この時、合成写像$G F = G\circ F: U \to W$を
$$
G F\vb*{v} = G(F(\vb*{v}))
$$
で定義する時、$G F$は$U$から$W$への線形写像である。
\end{proposition}

線形写像による線形部分空間の像や逆像は線形部分空間である。

\begin{proposition}[像と逆像]
$V, W$を$K$上の線形空間として、$F$を$V$から$W$への線形写像とする。
\begin{itemize}
\item
$U$を$V$の線形部分空間とする時、像$F U = \lrset{ F\vb*{v} \mid \vb*{v} \in U }$は$W$の線形部分空間である。
\item
$U$を$W$の線形部分空間とする時、逆像$F^{-1} U = \lrset{ \vb*{v} \in V \mid F\vb*{v} \in U }$は$V$の線形部分空間である。
\end{itemize}
\end{proposition}

\begin{proof}
一つ目について、$c, d \in K$と$F\vb*{v}, F\vb*{w} \in F U$に対して$c F\vb*{v}+d F\vb*{w} = F(c \vb*{v}+d \vb*{w})$であり、
$U$が線形部分空間なので$c \vb*{v}+d \vb*{w} \in U$より$c F\vb*{v}+d F\vb*{w} \in F U$である。
よって$F U$も線形部分空間である。

二つ目について、$c, d \in K$と$\vb*{v}, \vb*{w} \in F^{-1} U$に対して$F\vb*{v}, F\vb*{w} \in U$で$F(c\vb*{v}+d\vb*{w}) = c F\vb*{v}+d F\vb*{w} \in U$なので、$c\vb*{v}+d\vb*{w} \in F^{-1} U$である。
よって$F^{-1} U$も線形部分空間である。
\end{proof}

この性質に基づいて線形写像から定まる特徴的な線形空間を導入する。

\begin{definition}[像と核]
$V, W$を$K$上の線形空間として、$F$を$V$から$W$への線形写像とする。
\begin{itemize}
\item
$F$による$V$全体の像$F V$は$W$の線形部分空間で$F$の\emph{像}といい$\Img F$と表す。
\item
$F$による$O_W$の逆像$F^{-1} O_W$は$V$の線形部分空間で$F$の\emph{核}といい$\Ker F$と表す。
\end{itemize}
\end{definition}

像と核はそれぞれ全射性と単射性と結びつく。

\begin{proposition}
$V, W$を$K$上の線形空間として、$F$を$V$から$W$への線形写像とする。
\begin{itemize}
\item
$F$が全射である、つまり任意の$\vb*{w} \in W$に対して$\vb*{w} = F\vb*{v}$となる$\vb*{v} \in V$が存在する、ための必要十分条件は$\Img F = W$が成り立つことである。
\item
$F$が単射である、つまり任意の$\vb*{v}, \vb{w} \in V$に対して$F\vb*{v} = F\vb*{w}$ならば$\vb*{v} = \vb*{w}$が成り立つ、ための必要十分条件は$\Ker F = O_V$が成り立つことである。
\end{itemize}
\end{proposition}

\begin{proposition}
$V, W$を$K$上の線形空間として、$F$を$V$から$W$への線形写像とする。
ここで$F$を全単射つまり可逆とする時、逆写像$F^{-1}$も線形写像である。
\end{proposition}

\begin{proof}
$c, d \in K$と$\vb*{v}, \vb*{w} \in W$を取る。
示すべきことは
$$
F^{-1}(c\vb*{v}+d\vb*{w}) = c F^{-1}\vb*{v}+d F^{-1}\vb*{w}
$$
なので、両辺に$F$を施したものを考えると$F$の線形性より
$$
F(c F^{-1}\vb*{v}+d F^{-1}\vb*{w}) = c F F^{-1}\vb*{v}+d F F^{-1}\vb*{w} = c\vb*{v}+d\vb*{w}
$$
よって示すべき等式が得られて$F^{-1}$は線形写像である。
\end{proof}

上記の命題の内容を踏まえて次の線形同型性を定義する。

\begin{definition}[線形同型]
$V, W$を$K$上の線形空間とする。
$V$から$W$への線形写像$F$が可逆で逆写像$F^{-1}$も線形写像である時、$F$を\emph{線形同型写像}という。
$V, W$に対してその間の線形同型写像が存在する時$V$と$W$は\emph{線形同型}であるといい、$V \cong W$と表す。
\end{definition}

\begin{theorem}[次元定理]
$V, W$を$K$上の線形空間、$F$を$V$から$W$への線形写像とする。
このとき、
$$
\dim V = \dim \Img F+\dim \Ker F
$$
が成り立つ。
\end{theorem}

証明を見ればわかる通りこの定理の要点は$V$から$\Ker F$の部分を除くと($U$としている)$\Img F$と線形同型になっていることである。
$\dim \Img F$はしばしば$\rank F$と表される。

\begin{proof}
$\Img F$が無限次元の時は任意の$N = 0, 1, 2, 3, \cdots$に対して線形独立なベクトル$\vb*{w}_1, \cdots, \vb*{w}_N \in \Img F$が存在する。
ここで各$\vb*{w}_i$に対して$F\vb*{u}_i = \vb*{w}_i$となる$\vb*{u}_i \in V$が存在するので取ってくる。
この時$\vb*{u}_1, \cdots, \vb*{u}_N$は線形独立を示す。
スカラー$c_1, \cdots, c_N \in K$に対して
$$
c_1\vb*{u}_1+\cdots+c_N\vb*{u}_N = \vb*{0}_V
$$
となったとすると$F$で写して
$$
F(c_1\vb*{u}_1+\cdots+c_N\vb*{u}_N)
= c_1 F\vb*{u}_1+\cdots+c_N F\vb*{u}_N
= c_1\vb*{w}_1+\cdots+c_N\vb*{w}_N
= F(\vb*{0}_V)
= \vb*{0}_W.
$$
よって、$\vb*{w}_1, \cdots, \vb*{w}_N$が線形独立なので、$c_1 = \cdots = c_N = 0$であり、$\vb*{u}_1, \cdots, \vb*{u}_N$も線形独立である。
従って$\dim V \ge N$なので、$\dim V = \infty$となり定理の式を満たす。

$\Img F$が有限次元の時、$N = \dim \Img F$として$\Img F$の基底$\vb*{w}_1, \cdots, \vb*{w}_N$を取ってくる。
さらに先ほどと同様にして$F\vb*{u}_i = \vb*{w}_i$となる$\vb*{u}_1, \cdots, \vb*{u}_N \in V$を取ってくるとこれは線形独立している。
よってこれらは$V$の線形部分空間$U = \lrangle{\vb*{u}_1, \cdots, \vb*{u}_N}$の基底になっている。
この時、$F$は$U$から$\Img F$の線形同型写像になっているので、$U \cong \Img F$である。
あとは$U\cap \Ker F = O_V$と$U+\Ker F = V$を示せばよい。
まず$\vb*{v} \in U\cap \Ker F$とすると、
$$
\vb*{v} = c_1\vb*{u}_1+\cdots+c_N\vb*{u}_N
$$
とでき$F$で移すと
$$
F(c_1\vb*{u}_1+\cdots+c_N\vb*{u}_N)
= c_1 F\vb*{u}_1+\cdots+c_N F\vb*{u}_N
= c_1\vb*{w}_1+\cdots+c_N\vb*{w}_N
= F(\vb*{v})
= \vb*{0}_W
$$
よって$c_1 = \cdots = c_N = 0$で$\vb*{v} = \vb*{0}_V$である。
次に$\vb*{v} \in V$に対して、$F\vb*{v} \in \Img F$より$F\vb*{v} = c_1\vb*{w}_1+\cdots+c_N\vb*{w}_N$とでき、
$$
F(\vb*{v}-c_1\vb*{u}_1-\cdots-c_N\vb*{u}_N)
= F\vb*{v}-c_1 F\vb*{u}_1-\cdots-c_N F\vb*{u}_N
= F\vb*{v}-c_1\vb*{w}_1-\cdots-c_N\vb*{w}_N
= \vb*{0}_W.
$$
よって、$\vb*{v}-c_1\vb*{u}_1-\cdots-c_N\vb*{u}_N \in \Ker F$である。
以上より$U\oplus \Ker F = V$なので、
$$
\dim V = \dim U+\dim \Ker F = \rank F+\dim \Ker F
$$
が成り立つ。
\end{proof}

\section{数ベクトル空間}

ここまで抽象的に議論していた線形空間であるが、有限次元であれば数ベクトル空間と同一視でき、線形写像も行列に帰着できる。
まず、数ベクトル空間の次元について述べる。

\begin{proposition}
$N = 0, 1, 2, 3, \cdots$とする。
$K$上の$N$次元数ベクトル空間$K^N$は有限次元で$\dim K^N = N$であり、$N$個のベクトル
$$
\vb*{e}_1 = (1, 0, \cdots, 0),
\quad \cdots,
\quad \vb*{e}_N = (0, \cdots, 0, 1)
$$
が$K^N$の基底である。
\end{proposition}

\begin{proof}
$\vb*{e}_1, \cdots, \vb*{e}_N$が線形独立かつ$K^N$を線形生成ことを示せばよい。
線形結合はスカラー$c_1, \cdots, c_N \in K$に対して
$$
c_1\vb*{e}_1+\cdots+c_N\vb*{e}_N = (c_1, \cdots, c_N)
$$
なので、
これが零ベクトル$(0, \cdots, 0)$となるのは$c_1 = \cdots = c_N = 0$であり、
任意の$K^N$の元は上記の形に書ける。
よって証明できる。
\end{proof}

この時の基底$\vb*{e}_1, \cdots, \vb*{e}_N$を数ベクトル空間$K^N$の\emph{標準基底}という。

\begin{proposition}
$V$を$K$上の有限次元線形空間として次元を$N$とおくと、$V$は$K^N$と線形同型である。
\end{proposition}

\begin{proof}
$V$の基底$\vb*{v}_1, \cdots, \vb*{v}_N$を取ってきて、$K^N$から$V$への線形写像$F$を
$$
F(c_1, \cdots, c_N) = c_1\vb*{v}_1+\cdots+c_N\vb*{v}_N
$$
で定義する。
このとき、$\vb*{v}_1, \cdots, \vb*{v}_N$が$V$を線形生成することから$F$は全射で、
線形独立であることから$\Ker F = O_{K^N}$つまり$F$は単射である。
以上より$F$は線形同型写像であるので、$V \cong K^N$である。
\end{proof}

$M, N = 1, 2, 3, \cdots$として$A$を$K$上の$M\times N$型の行列とする。
ここで$K^N$から$K^M$への写像
$$
F_A(\vb*{v}) = A\vb*{v}
$$
で定めるとこれは線形写像であり、行列$A$が定める線形写像という。

これとは逆に線形写像が与えられたら対応する行列が取れる。

\begin{definition}[表現行列]
$V$と$W$を$K$上の有限次元線形空間として、次元をそれぞれ$N$と$M$とする。
$F$を$V$から$W$への線形写像として$V$の基底$\vb*{v}_1, \cdots, \vb*{v}_N$と$W$の基底$\vb*{w}_1, \cdots, \vb*{w}_M$について、各$j = 1, \cdots, N$に対して
$$
F(\vb*{v}_j) = a_{j 1}\vb*{w}_1+\cdots+a_{j M}\vb*{w}_M
$$
となる$a_{j 1}, \cdots, a_{j M} \in K$が一意に存在して定義される$K$上の$M\times N$型の行列
$$
\begin{pmatrix}
a_{1 1} & \cdots & a_{N 1} \\
\vdots & \ddots & \vdots \\
a_{1 M} & \cdots & a_{N M} \\
\end{pmatrix}
$$
を線形写像$F$の$V$の基底$\vb*{v}_1, \cdots, \vb*{v}_N$と$W$の基底$\vb*{w}_1, \cdots, \vb*{w}_M$に関する\emph{表現行列}という。
\end{definition}

表現行列は基底の取り方によって変わってしまうことに注意する。
$K$上の$M\times N$型の行列$A$が定める$K^N$から$K^M$への線形変換$F_A$の$K^N$と$K^M$の標準基底に関する表現行列は$A$である。

\section{線形変換}

線形写像の中でも定義域の空間と行き先の空間が同じ時、線形変換といい正方行列と同様の扱いができる。

\begin{definition}[線形変換]
$K$上の線形空間$V$からそこへの線形写像$T$を特に\emph{線形変換}という。
\end{definition}

\begin{definition}[基底の変換]
$K$上の有限次元線形空間$V$について、$N = \dim V$として$V$の基底$\vb*{v}_1, \cdots, \vb*{v}_N$と$\vb*{v}'_1, \cdots, \vb*{v}'_N$について
TODO
\end{definition}

$K$上の線形空間$V$の元$\vb*{v}$を$\vb*{v}$に移す線形変換を特に\emph{恒等変換}といい、$\Id_V$あるいは単に$\Id$と表す。


\chapter{固有値問題}

\section{固有値と固有ベクトル}

いよいよこの章から体$K$上の$N$次正方行列$A$の$n = 0, 1, 2, 3, \cdots$乗を計算するのに役に立つ理論を学ぶ。
例えば$A$が$N$次正則行列$P$を使って次のように変形されたとする。
$$
A = P\mqty(\dmat{c_1, \ddots, c_N})P^{-1}.
$$
一般に$N$次正方行列$A$を$N$次正則行列$P$を使って$B = P^{-1} A P$と変形することを\emph{相似変換}といい、このように相似変換して対角行列にすることを\emph{対角化}という。
このとき、$P = \mqty(\vb*{v}_1 & \cdots & \vb*{v}_N)$とすると
$$
A\mqty(\vb*{v}_1 & \cdots & \vb*{v}_N) = \mqty(c_1\vb*{v}_1 & \cdots & c_N\vb*{v}_N)
$$
なので、
\begin{equation}
\label{e:eigen}
A\vb*{v} = x\vb*{v}
\end{equation}
を満たす$(x, \vb*{v}) = (c_1, \vb*{v}_1), \cdots, (c_N, \vb*{v}_N)$であって$\vb*{v}_1, \cdots, \vb*{v}_N$が線形独立であるものを見つけてくれば対角化できることになる。
この時の$x$を$A$の\emph{固有値}、$\vb*{v}$を$A$の\emph{固有ベクトル}といい関係式\eqref{e:eigen}を\emph{固有値問題}という。

それではどうやって固有値・固有ベクトルを見つけてくればよいのだろうか。
\eqref{e:eigen}を変形すると$\vb*{v}$は斉次連立一次方程式
$$
(x I-A)\vb*{v} = \vb*{0}
$$
の解であり、$\vb*{v} \ne \vb*{0}$でないと線形独立にならない。
よって、$N$次正方行列$x I-A$は逆行列を持たないので、
\begin{equation}
\label{e:eigeneq}
\det(x I-A) = 0
\end{equation}
が成立する。
これを$x$についての方程式とみて$A$の\emph{固有方程式}という。
なお、左辺は$x$についての$K$係数の$N$次の多項式になっていて、$A$の\emph{固有多項式}という。
固有方程式の解つまり固有多項式の零点が固有値であり、$x \in K$に対して$K^N$の線形部分空間
$$
W(x) = \Ker(x I-A) = \lrset{ \vb*{v} \in K^N \mid (x I-A)\vb*{v} = \vb*{0} }
$$
を定め、固有値$x = c$に対して$W(c)$を$A$の固有値$c$に対する\emph{固有空間}という。
このとき$\dim W(c) \ge 1$に注意する。

\section{固有空間の線形独立性}

\begin{lemma}[固有空間の線形独立性]
$T$を$K$上の線形空間$V$上の線形変換として、$c, d \in K$を異なる固有値とする。
このとき固有空間について$W(c)\cap W(d) = O_V$が成り立つ。
\end{lemma}

\begin{proof}
$\vb*{v} \in W(c)\cap W(d)$とすると、
$$
A\vb*{v} = c\vb*{v} = d\vb*{v}.
$$
よって、$c \ne d$より$\vb*{v} = \vb*{0}_V$を得る。
\end{proof}

\begin{theorem}[対角化の十分条件]
$A$を$K$上の$N$次元正方行列として、$c_1, \cdots, c_L \in K$を相異なる固有値とする。
ここで固有空間が$W(c_1)\oplus\cdots\oplus W(c_L) = K^N$、つまり$\dim W(c_1)+\cdots+\dim W(c_L) = N$を満たすとき、$A$は対角化可能である。

特に$K$上の$N$次元正方行列$A$が相異なる$N$個の固有値を持つならば、$A$は対角化可能である。
\end{theorem}

\begin{example}
第1章で紹介した液体の入れ替えの問題をもとに片方の容器から$p \in [0, 1]$の割合をもう片方の容器から$q \in [0, 1]$の割合の液体を入れ替える問題を考えると、係数行列は
$$
A = \mqty(1-p & q \\ p & 1-q)
$$
となりこれを対角化することを考える。
$A$の固有多項式は
$$
\det(x I_2-A)
= \mqty|x-1+p & -q \\ -p & x-1+q|
= (x-1+p)(x-1+q)-p q
= x^2-(2-p-q)x+1-p-q
= (x-1)(x-1+p+q).
$$
よって固有値は$x = 1, 1-p-q$である。
ここで$p+q = 0$の時この固有値は$1$が$2$重になるが、$p = q = 0$なので$A = \mqty(1 & 0 \\ 0 & 1)$と最初から対角化されている。
$p+q \ne 0$の場合は$2$つの異なる固有値を持つので$A$は対角化可能である。
この時、固有値$x = 1$に対しては
$$
x I_2-A = \mqty(p & -q \\ -p & q)
$$
より固有ベクトルとして$(q, p)$が取れて、
固有値$x = 1-p-q$に対しては
$$
x I_2-A = \mqty(-q & -q \\ -p & -p)
$$
より固有ベクトルとして$(1, -1)$が取れる。
よって、$A$は
$$
A
= \mqty(1-p & q \\ p & 1-q)
= \mqty(q & 1 \\ p & -1)\mqty(1 & 0 \\ 0 & 1-p-q)\mqty(q & 1 \\ p & -1)^{-1}
$$
と対角化でき、
$$
A^n
= \mqty(q & 1 \\ p & -1)\mqty(1 & 0 \\ 0 & (1-p-q)^n)\mqty(q & 1 \\ p & -1)^{-1}
$$
と計算できる。
\end{example}

\section{三角化可能性}

対角化を緩和して、$N$次正方行列$A$を$N$次正則行列$P$を使って
\begin{equation}
\label{e:tri}
A = P\mqty(c_1 & \cdots & * \\ & \ddots & \vdots \\ & & c_N)P^{-1}
\end{equation}
と変形することを\emph{三角化}という。
本節の見出しは三角化「可能性」となっているが、後で見るように三角化なら(固有方程式が解ければ)常にできることが重要である。

まず、三角化できるならば固有多項式が因数分解されることを紹介する。

\begin{proposition}
\label{t:eigenfactor}
$N$次正方行列$A$が\eqref{e:tri}と三角化されたとすると固有多項式は
\begin{equation}
\label{e:eigenfactor}
\det(x I-A) = (x-c_1)\cdots(x-c_N)
\end{equation}
と因数分解される。
\end{proposition}

\begin{proof}
$x I-A$を書き直すと
$$
x I-A = P\mqty(x-c_1 & \cdots & * \\ & \ddots & \vdots \\ & & x-c_N)P^{-1}.
$$
この行列式を取ると、
$$
\det(x I-A)
= \det P\det\mqty(x-c_1 & \cdots & * \\ & \ddots & \vdots \\ & & x-c_N)\det P^{-1}
= (x-c_1)\cdots(x-c_N)
$$
である。
\end{proof}

逆に固有多項式が因数分解されるならば三角化できる。

\begin{theorem}[一般の三角化]
\label{t:tri}
$N$次正方行列$A$に対して、固有多項式が\eqref{e:eigenfactor}と因数分解されたとすると、$A$は\eqref{e:tri}と三角化される。
\end{theorem}

\begin{proof}
$N$に関する数学的帰納法で証明する。
$N = 1$の時は$A$はスカラーなので三角化されている。
$N-1$次までで三角化されたとする。
$c_1$は固有方程式の解で固有値なので、
$$
A\vb*{v}_1 = c_1\vb*{v}_1
$$
となるベクトル$\vb*{v}_1 \ne \vb*{0}$が存在する。
ここで基底の延長を行い、$K^N$の基底$\vb*{v}_1, \vb*{v}_2, \cdots, \vb*{v}_N$を構成して正則行列$P_0 = \mqty(\vb*{v}_1 & \vb*{v}_2 & \cdots & \vb*{v}_N)$とおくと、
$$
A P_0
= P_0\mqty(c_1 & * & \cdots & * \\ 0 & * & \cdots & * \\ \vdots & \vdots & \ddots & \vdots \\ 0 & * & \cdots & *)
= P_0\mqty(c_1 & \vb*{*} \\ \vb*{0} & A_1)
$$
となる。
前の命題の証明同様に行列式を考えると、
$$
\det(x I_{N-1}-A_1) = (x-c_2)\cdots(x-c_N)
$$
となるので、数学的帰納法の仮定より
$$
A_1 = \tilde{P}_1\mqty(c_2 & \cdots & * \\ & \ddots & \vdots \\ & & c_N)\tilde{P}_1^{-1}
$$
と三角化される。
ここで、
$$
P_1 = \mqty(1 & \vb*{0} \\ \vb*{0} & \tilde{P}_1),
\quad P = P_0 P_1
$$
とおくと、$P$は$N$次正則行列で
$$
A P
= P_0\mqty(c_1 & \vb*{*} \\ \vb*{0} & A_1)\mqty(1 & \vb*{0} \\ \vb*{0} & \tilde{P}_1)
= P_0\mqty(c_1 & \vb*{*} \\ \vb*{0} & A_1\tilde{P}_1)
= P\mqty(c_1 & \cdots & * \\ & \ddots & \vdots \\ & & c_N)
$$
となる。
以上より$A$は三角化される。
\end{proof}

\section{代数的閉体}

命題\ref{t:eigenfactor}でわかるように、行列が三角化されるためには固有多項式が因数分解されることが必要であるが、
因数分解できることを保証する$K$の条件が代数的閉体であることである。

\begin{definition}[代数的閉体]
$K$上の$1$次以上の任意の多項式$f(x) \in K[x]$に対して零点、つまり$f(c) = 0$となる$c \in K$が存在する時、
$K$は\emph{代数的閉体}であるという。
\end{definition}

\begin{proposition}[因数分解]
$K$を代数的閉体として、$f(x)$をその上の$N = 1, 2, 3, \cdots$次の多項式であって$x^N$の係数が$1$であるものとする。
このとき、$c_1, \cdots, c_N \in K$を使って
$$
f(x) = (x-c_1)\cdots(x-c_N)
$$
と因数分解される。
\end{proposition}

\begin{example}
代数的閉体の重要な例は複素数体$\mathbb{C}$である。
これは代数学の基本定理と呼ばれる定理からわかるが、その証明には$\mathbb{C}$あるいは$\mathbb{R}^2$上の微分積分学の知識が必要である。
なお、実数体$\mathbb{R}$や有理数体$\mathbb{Q}$などは代数的閉体ではない。
\end{example}

\begin{example}
複素数の中でも有理数$\mathbb{Q}$上の$1$次以上の多項式$f(x) \in \mathbb{Q}[x]$の零点となる数は\emph{代数的数}と呼ばれる。
代数的数全体の集合
$$
\bar{\mathbb{Q}} = \lrset{ a \in \mathbb{C} \mid a^n+c_{n-1}a^{n-1}+\cdots+c_0 = 0, n = 1, 2, 3, \cdots, c_0, \cdots, c_{n-1} \in \mathbb{Q} }
$$
は複素数体$\mathbb{C}$の部分体であり、代数的閉体になっている。
\end{example}

\section{固有値の重複度}

$A$を$K$上の$N$次正方行列として、$c$をその固有値とする。
このとき、$A$の固有多項式$\det(x I-A)$は$x-c$で割り切れるが、$\det(x I-A)$が$(x-c)^n$で割り切れるような最大の$n$を固有値$c$の\emph{代数的重複度}という。
固有多項式は$N$次なので代数的重複度は$1, \cdots, N$の値を取る。
一方で固有空間$W(c)$の次元を\emph{幾何学的重複度}という。

\begin{proposition}
$A$を$K$上の$N$次正方行列として、$c$をその固有値とする。
このとき$c$の幾何学的重複度は代数的重複度以下である。
つまり$M = \dim W(c)$とすると、$A$の固有多項式$\det(x I-A)$は$(x-c)^M$で割り切れる。
\end{proposition}

\begin{proof}
線形独立な$M$個のベクトル$\vb*{v}_1, \cdots, \vb*{v}_M \in K^N$がとれ、拡張をして$K^N$の基底$\vb*{v}_1, \cdots, \vb*{v}_N$を構成する。
このとき、$P = \mqty(\vb*{v}_1 & \cdots & \vb*{v}_N)$を使って、
$$
A = P\mqty(c I_M & * \\ O & *)P^{-1}
$$
とでき、
$$
x I-A = P\mqty((x-c)I_M & * \\ O & *)P^{-1}
$$
より、$\det(x I-A)$は$(x-c)^M$で割り切れる。
\end{proof}


\chapter{対角化}

\section{共役と内積}

この章では対角化の十分条件などを与えるために線形空間に内積の構造を(入れられる場合に)入れる。
また、複素数体上の線形空間に内積を入れるために共役の概念も合わせて導入する。

\begin{definition}[内積]
$V$を$K$上の線形空間とする。
次を満たす対応$\overline{\cdot}: K \to K$と$\lrangle{\cdot, \cdot}: V\times V \to K$をそれぞれ$K$上の\emph{共役}と$V$上の\emph{内積}といい、$V$を\emph{内積空間}という。
\begin{enumerate}
\item
(第二変数に関する線形性)任意の$c, d \in K$と$\vb*{u}, \vb*{v}, \vb*{w} \in V$に対して、$\lrangle{\vb*{u}, c\vb*{v}+d\vb*{w}} = c\lrangle{\vb*{u}, \vb*{v}}+d\lrangle{\vb*{u}, \vb*{w}}$が成り立つ。
\item
(非退化性)$\vb*{v} \in V$が$\lrangle{\vb*{v}, \vb*{v}} = 0$を満たすならば$\vb*{v} = \vb*{0}$である。
\item
(共役対称性)任意の$\vb*{u}, \vb*{v} \in V$に対して、$\lrangle{\vb*{v}, \vb*{u}} = \overline{\lrangle{\vb*{u}, \vb*{v}}}$が成り立つ。
\item
(加法の共役)任意の$a, b \in K$に対して、$\overline{a+b} = \overline{a}+\overline{b}$が成り立つ。
\item
(乗法の共役)任意の$a, b \in K$に対して、$\overline{a\cdot b} = \overline{a}\cdot\overline{b}$が成り立つ。
\end{enumerate}
\end{definition}

\begin{remark}
内積の第一変数に関する線形性について
$$
\lrangle{c\vb*{v}+d\vb*{w}, \vb*{u}}
= \overline{\lrangle{\vb*{u}, c\vb*{v}+d\vb*{w}}}
= \overline{c\lrangle{\vb*{u}, \vb*{v}}+d\lrangle{\vb*{u}, \vb*{w}}}
= \overline{c}\overline{\lrangle{\vb*{u}, \vb*{v}}}+\overline{d}\overline{\lrangle{\vb*{u}, \vb*{w}}}
= \overline{c}\lrangle{\vb*{v}, \vb*{u}}+\overline{d}\lrangle{\vb*{w}, \vb*{u}}
$$
が成り立ち、共役線形性などと呼ばれる。
\end{remark}

\begin{remark}
共役について、$\overline{0} = 0$と$\overline{1}^2 = \overline{1}$がわかる。
$\overline{1} = 0$だとすると、任意の$a \in K$に対して$\overline{a} = 0\cdot\overline{a} = 0$で、
共役対称性から$\lrangle{\vb*{v}, \vb*{u}} = 0$で非退化性から$V = O$以外あり得ない。
なので$V$が自明な線形空間でない時は$\overline{1} = 1$である。
また、共役対称性より二回共役を取ると$\overline{\overline{\lrangle{\vb*{u}, \vb*{v}}}} = \lrangle{\vb*{u}, \vb*{v}}$が成り立ち、内積は全射になるので、任意の$a \in K$に対して
$$
\overline{\overline{a}} = a
$$
が成り立つ。
\end{remark}

スカラー$K$に共役の構造が入ったら、数ベクトル$\vb*{v} = (v_1, \cdots, v_N) \in K^N$と行列$A = (a_{i j})^{i = 1, \cdots, M}_{j = 1, \cdots, N}$に対する共役を成分ごとに共役をとったもの
$$
\overline{\vb*{v}} = (\overline{v_1}, \cdots, \overline{v_N}),
\quad
\overline{A} = \mqty(\overline{a_{1 1}} & \cdots & \overline{a_{1 N}} \\ \vdots & \ddots & \vdots \\ \overline{a_{M 1}} & \cdots & \overline{a_{M N}})
$$
として定められる。

数ベクトル空間$K^N$上の内積をスカラー積を使って
$$
\lrangle{\vb*{u}, \vb*{v}}
= \overline{\vb*{u}}\cdot\vb*{v}
= \overline{u_1} v_1+\cdots+\overline{u_N} v_N
$$
として定めたものを$K^N$の\emph{標準内積}という。
ただし、標準内積がちゃんと内積になっているかは非退化性の条件が成り立っているかどうか確認する必要がある。

\begin{definition}[非退化な体]
共役が定義される体$K$が\emph{非退化}であるとは、任意の$n = 1, 2, 3, \cdots$と任意の$n$個のスカラー$c_1, \cdots, c_n \in K$について
$$
\overline{c_1}c_1+\cdots+\overline{c_n}c_n = 0
$$
ならば、$c_1 = \cdots = c_n = 0$が成り立つことをいう。
\end{definition}

非退化な体$K$上では任意の$N = 1, 2, 3, \cdots$に対して数ベクトル空間$K^N$は標準内積について内積空間になる。

\begin{example}
実数体$\mathbb{R}$上の共役を$\overline{a} = a$で定めるとこれは非退化である。
\end{example}

\begin{example}
複素数体$\mathbb{C}$上の共役を\emph{複素共役}$\overline{a+b i} = a-b i$で定めるとこれは非退化である。
\end{example}

\begin{example}
一般の非退化とは限らない体$K$上の共役を$\overline{a} = a$で定め、$K$を$K$上の線形空間として見た時内積を単なる積$\lrangle{u, v} = u v$で定めると$K$は内積空間になる。
つまり、$N = 1$の時$\mathbb{C}^N$は二通りの内積の入れ方があることになる。
\end{example}

\section{シュミットの直交化}

内積空間$V$上の二つのベクトル$\vb*{u}$と$\vb*{v}$が\emph{直交する}とは
$$
\lrangle{\vb*{u}, \vb*{v}} = 0
$$
が成り立つことをいう。
定義上は$\vb*{u} = \vb*{0}$または$\vb*{v} = \vb*{0}$が成り立っていたら$\vb*{u}$と$\vb*{v}$は必ず直交している。
また、ベクトル$\vb*{v}$が\emph{単位ベクトル}とは
$$
\lrangle{\vb*{v}, \vb*{v}} = 1
$$
が成り立つことをいう。

$V$の基底$\vb*{v}_1, \cdots, \vb*{v}_N$が各$i \ne j$, $i, j = 1, \cdots, N$に対して$\vb*{v}_i$と$\vb*{v}_j$が直交することを満たす時、\emph{直交基底}であるという。
直交基底がさらに各$i = 1, \cdots, N$に対して$\vb*{v}_i$が単位ベクトルである時、\emph{正規直交基底}であるという。

本節の目標は$V$の基底が与えられた時に直交基底あるいは正規直交基底に取り替えることである。
そのことを表現するために$V$の$N = 1, \cdots$個の零ベクトルでないベクトル$\vb*{v}_1, \cdots, \vb*{v}_N$が各$i \ne j$, $i, j = 1, \cdots, N$に対して$\vb*{v}_i$と$\vb*{v}_j$が直交することを満たす時、$V$の\emph{直交系}であるということにする。
また、各$\vb*{v}_i$が単位ベクトルである時、$V$の\emph{正規直交系}であるという。

\begin{proposition}[直交系と線形独立]
$\vb*{v}_1, \cdots, \vb*{v}_N$を$K$上の内積空間$V$の直交系とする時、それらは線形独立である。
\end{proposition}

\begin{proof}
スカラー$c_1, \cdots, c_N \in K$に対して
$$
c_1\vb*{v}_1+\cdots+c_N\vb*{v}_N = \vb*{0}
$$
が成り立ったとする時、各$i = 1, \cdots, N$に対して、$\vb*{v}_i$との内積を取って
$$
\lrangle{\vb*{v}_i, c_1\vb*{v}_1+\cdots+c_N\vb*{v}_N} = c_i\lrangle{\vb*{v}_i, \vb*{v}_i} = 0.
$$
よって$c_i = 0$なので、$\vb*{v}_1, \cdots, \vb*{v}_N$は線形独立である。
\end{proof}

この命題とその証明から内積空間$V$の直交系$\vb*{v}_1, \cdots, \vb*{v}_N$が与えられるとそれは線形独立で$V$の線形部分空間$W = \lrangle{\vb*{v}_1, \cdots, \vb*{v}_N}$が得られる。
そのベクトル$\vb*{v} = c_1\vb*{v}_1+\cdots+c_N\vb*{v}_N \in W$と$i = 1, \cdots, N$に対して、$\vb*{v}_i$との内積を取ると$\lrangle{\vb*{v}_i, \vb*{v}} = c_i\lrangle{\vb*{v}_i, \vb*{v}_i}$でありこのスカラー値$c_i$のことを$\vb*{v}$の$\vb*{v}_i$\emph{成分}と呼ぶ。

逆に線形独立なベクトルが与えられたときに成分を引くことで直交系を構成するというのがシュミットの直交化である。

\begin{theorem}[シュミットの直交化]
$V$を内積空間として$\vb*{v}_1, \cdots, \vb*{v}_N$を線形独立なベクトルとして、次の方法によって$V$の新しいベクトル$\vb*{u}_1, \cdots, \vb*{u}_N$を構成する。
$$
\begin{aligned}
&\vb*{u}_1 = \vb*{v}_1,
\quad
\vb*{u}_2 = \vb*{v}_2-\frac{\lrangle{\vb*{u}_1, \vb*{v}_2}}{\lrangle{\vb*{u}_1, \vb*{u}_1}}\vb*{u}_1,
\quad
\vb*{u}_3 = \vb*{v}_3-\frac{\lrangle{\vb*{u}_1, \vb*{v}_3}}{\lrangle{\vb*{u}_1, \vb*{u}_1}}\vb*{u}_1-\frac{\lrangle{\vb*{u}_2, \vb*{v}_3}}{\lrangle{\vb*{u}_2, \vb*{u}_2}}\vb*{u}_2, \\
&\quad \cdots,
\quad
\vb*{u}_N = \vb*{v}_N-\frac{\lrangle{\vb*{u}_1, \vb*{v}_N}}{\lrangle{\vb*{u}_1, \vb*{u}_1}}\vb*{u}_1-\cdots-\frac{\lrangle{\vb*{u}_{N-1}, \vb*{v}_N}}{\lrangle{\vb*{u}_{N-1}, \vb*{u}_{N-1}}}\vb*{u}_{N-1}.
\end{aligned}
$$
このとき$\vb*{u}_1, \cdots, \vb*{u}_N$は$V$の直交系であり、$\lrangle{\vb*{u}_1, \cdots, \vb*{u}_N} = \lrangle{\vb*{v}_1, \cdots, \vb*{v}_N}$が成り立つ。
\end{theorem}

\begin{proof}
$N$についての数学的帰納法による。
$N = 1$の時、$\vb*{u}_1 = \vb*{v}_1 \ne \vb*{0}$より成立する。
$N$個で成立した時、$N+1$個目のベクトルを
$$
\vb*{u}_{N+1} = \vb*{v}_{N+1}-\frac{\lrangle{\vb*{u}_1, \vb*{v}_{N+1}}}{\lrangle{\vb*{u}_1, \vb*{u}_1}}\vb*{u}_1-\cdots-\frac{\lrangle{\vb*{u}_N, \vb*{v}_{N+1}}}{\lrangle{\vb*{u}_N, \vb*{u}_N}}\vb*{u}_N
$$
で定めると、$\vb*{u}_1, \cdots, \vb*{u}_N$は$V$の直交系より、各$i = 1, \cdots, N$に対して$\vb*{u}_i$との内積を取って、
$$
\lrangle{\vb*{u}_i, \vb*{u}_{N+1}} = \lrangle{\vb*{u}_i, \vb*{v}_{N+1}}-\frac{\lrangle{\vb*{u}_i, \vb*{v}_{N+1}}}{\lrangle{\vb*{u}_i, \vb*{u}_i}}\lrangle{\vb*{u}_i, \vb*{u}_i} = 0.
$$
また、$\vb*{u}_{N+1} = \vb*{0}$とすると
$$
\vb*{v}_{N+1}
= \frac{\lrangle{\vb*{u}_1, \vb*{v}_{N+1}}}{\lrangle{\vb*{u}_1, \vb*{u}_1}}\vb*{u}_1+\cdots+\frac{\lrangle{\vb*{u}_N, \vb*{v}_{N+1}}}{\lrangle{\vb*{u}_N, \vb*{u}_N}}\vb*{u}_N
\in \lrangle{\vb*{u}_1, \cdots, \vb*{u}_N}
= \lrangle{\vb*{v}_1, \cdots, \vb*{v}_N}
$$
となるが、これは$\vb*{v}_1, \cdots, \vb*{v}_N, \vb*{v}_{N+1}$が線形独立であるという仮定に反する。
よって、$\vb*{u}_1, \cdots, \vb*{u}_N, \vb*{u}_{N+1}$は$V$の直交系である。
他の部分の証明も容易であり、定理の主張を得る。
\end{proof}

\begin{remark}
$\vb*{u}_1, \cdots, \vb*{u}_N$の構成方法から
$$
\mqty(\vb*{v}_1 & \cdots & \vb*{v}_N)
= \mqty(\vb*{u}_1 & \cdots & \vb*{u}_N)\mqty(1 & \cdots & * \\ & \ddots & \vdots \\ & & 1)
$$
と表示できる。
\end{remark}

\begin{corollary}
任意の有限次元内積空間$V$には直交基底が存在する。
\end{corollary}

正規化のためには平方根を導入する必要がある。

\begin{definition}[正規化可能]
$K$を非退化な体として、その\emph{正値の部分}を
$$
K_{\ge 0} = \lrset{ \overline{c_1}c_1+\cdots+\overline{c_n}c_n \mid n = 1, 2, 3, \cdots, c_1, \cdots, c_n \in K }
$$
とする。
ここで正値$c \in K_{\ge 0}$に対して、
$$
\overline{x}x = c
$$
を満たす正値$x \in K_{\ge 0}$が一意に存在する時、$x$を$c$の\emph{正の平方根}といい$\sqrt{c}$と表す。
さらに任意の$c \in K_{\ge 0}$に対して正の平方根$\sqrt{c}$が存在する時、$K$は\emph{正規化可能}と呼ぶ。
\end{definition}

\begin{remark}
$0$は$0$の正の平方根であり、$1$は$1$の正の平方根である。
$\overline{x}x = c$と書いたが、正値$x \in K_{\ge 0}$の共役は
$$
\overline{x} = \overline{\overline{c_1}c_1+\cdots+\overline{c_n}c_n} = c_1\overline{c_1}+\cdots+c_n\overline{c_n} = x
$$
なので、通常の意味での平方根$x^2 = x x = c$と認識したのでよい。
\end{remark}

\begin{example}
実数体$\mathbb{R}$の正値の部分は非負の実数全体である。
非負の実数には正の平方根が存在するので、$\mathbb{R}$は正規化可能な体である。
同様にして、複素共役の入った複素数体$\mathbb{C}$の正値の部分は非負の実数全体なので、$\mathbb{C}$は正規化可能な体である。
しかしながら、非負の有理数$2 = \overline{1}\cdot 1+\overline{1}\cdot 1$に対して、$\sqrt{2}$は無理数なので、$\mathbb{Q}$は正規化可能でない。
\end{example}

\begin{example}
複素共役の入った代数的数の体$\bar{\mathbb{Q}}$の正値の部分は非負の代数的数であり、非負の代数的数$c$を多項式$f(x)$の零点とすると多項式$f(x^2)$の零点を考えることで、$\bar{\mathbb{Q}}$は正規化可能な体であることが結論づけられる。
\end{example}

\begin{proposition}[正規化]
正規化可能な体\emph{K}上の数ベクトル$\vb*{v}$に対して、$\lrangle{\vb*{v}, \vb*{v}}$は$K_{\ge 0}$の元である。
さらにその正の平方根を$\norm{\vb*{v}} = \sqrt{\lrangle{\vb*{v}, \vb*{v}}} \in K_{\ge 0}$とおく。
ここで、$\vb*{v} \ne \vb*{0}$に対して
$$
\tilde{\vb*{v}} = \norm{\vb*{v}}^{-1}\vb*{v}
$$
とおくとこれは単位ベクトルになっている。
\end{proposition}

このときの$\norm{\vb*{v}}$の\emph{ノルム}または\emph{大きさ}あるいは\emph{長さ}という。

正規化を各ベクトルに実施することにより直交基底から正規直交基底を作ることができる。

\section{随伴行列}

標準内積の入った数ベクトル空間$K^N$を考える。
$N$次正方行列$A$に対して、
$$
\lrangle{\vb*{u}, A\vb*{v}} = \lrangle{A^*\vb*{u}, \vb*{v}}
$$
が任意の$\vb*{u}, \vb*{v} \in K^N$に対して成り立つような$N$次正方行列$A^*$のことを$A$の\emph{随伴行列}という。
すぐ後でわかる通り随伴行列は$A$の共役転置行列で実現される。
$M\times N$型の行列$A$の\emph{共役転置行列}は
$$
\overline{A}^T = \mqty(\overline{a_{1 1}} & \cdots & \overline{a_{M 1}} \\ \vdots & \ddots & \vdots \\ \overline{a_{1 N}} & \cdots & \overline{a_{M N}})
$$
として定義される。

\begin{proposition}[随伴行列]
$N$次正方行列$A$の随伴行列は$A^* = \overline{A}^T$で与えられる。
\end{proposition}

\begin{proof}
標準内積を行列の積で書くことにより、
$$
\lrangle{\vb*{u}, A\vb*{v}}
= \overline{\vb*{u}}^T A\vb*{v}
= \overline{\overline{A}^T \vb*{u}}^T \vb*{v}
= \lrangle{\overline{A}^T \vb*{u}, \vb*{v}}.
$$
よって、$A^* = \overline{A}^T$である。
\end{proof}

ここからいくつかの行列の種類を導入する。

\begin{definition}[エルミート行列]
$N$次正方行列$A$が
$$
A^* = A
$$
を満たす時、$A$は\emph{エルミート行列}という。
\end{definition}

\begin{definition}[ユニタリ行列]
$N$次正方行列$A$が
$$
A^* A = A A^*
$$
であってそれらが正則な対角行列である時、$A$は\emph{直交行列}という。
さらに
$$
A^* A = A A^* = I_N
$$
を満たす時、$A$は\emph{正規直交行列}または\emph{ユニタリ行列}という。
\end{definition}

\begin{remark}
$A = \begin{pmatrix}\vb*{a}_1 & \cdots & \vb*{a}_N\end{pmatrix}$と分割すると、
$A$が直交行列であることと$\vb*{a}_1, \cdots, \vb*{a}_N$が直交基底であることは同値であり、
$A$が正規直交行列であることと$\vb*{a}_1, \cdots, \vb*{a}_N$が正規直交基底であることは同値であり、
\end{remark}

一般的な用語としては直交行列は実数上のユニタリ行列のことを指すが、本テキストでは直交基底という用語との兼ね合いでこのように定義する。

直交行列と正規直交行列の違いは正規化されているかどうかだが、
正規直交行列$A$の逆行列は計算するまでもなく直ちに随伴$A^*$であることがわかる。

\begin{definition}[正規行列]
$N$次正方行列$A$が
$$
A^* A = A A^*
$$
を満たす時、$A$は\emph{正規行列}という。
\end{definition}

\begin{remark}
エルミート行列やユニタリ行列は正規行列であり、正規行列は非常に広い行列の種類である。
\end{remark}

\section{正規行列の対角化}

対角化の結果に行く前に内積空間において三角化の結果を精密化する。

\begin{lemma}[直交行列での三角化]
数ベクトル空間$K^N$が内積空間になっているとする。
$N$次正方行列$A$に対して、固有多項式が\eqref{e:eigenfactor}と因数分解されたとすると、$A$は直交行列$P$を使って\eqref{e:tri}と三角化される。
さらに$K$が正規化可能な場合は$P$は正規直交行列として取れる。
\end{lemma}

\begin{proof}
定理\ref{t:tri}より正則行列$P$を使って\eqref{e:tri}と三角化される。
$P = \mqty(\vb*{v}_1 & \cdots & \vb*{v}_N)$と区分けすると$\vb*{v}_1, \cdots, \vb*{v}_N$は$K^N$の基底になっており、
この基底をシュミットの直交化して直交基底$\vb*{u}_1, \cdots, \vb*{u}_N$を得ると、
直交行列$\bar{P} = \mqty(\vb*{u}_1 & \cdots & \vb*{u}_N)$により、
$$
A = \bar{P}\mqty(1 & \cdots & * \\ & \ddots & \vdots \\ & & 1)\mqty(c_1 & \cdots & * \\ & \ddots & \vdots \\ & & c_N)\mqty(1 & \cdots & * \\ & \ddots & \vdots \\ & & 1)^{-1}\bar{P}^{-1}
$$
となる。
よって、右上三角行列の積は右上三角行列であることに注意して、
$$
A = \bar{P}\mqty(c_1 & \cdots & * \\ & \ddots & \vdots \\ & & c_N)\bar{P}^{-1}
$$
とできる。
さらに正規化して正規直交行列
$$
\tilde{P}
= \mqty(\norm{\vb*{u}_1}^{-1}\vb*{u}_1 & \cdots & \norm{\vb*{u}_N}^{-1}\vb*{u}_N)
= \bar{P}\mqty(\dmat{\norm{\vb*{u}_1}^{-1}, \ddots, \norm{\vb*{u}_N}^{-1}})
$$
を定めると、
$$
A
= \tilde{P}\mqty(\dmat{\norm{\vb*{u}_1}, \ddots, \norm{\vb*{u}_N}})\mqty(c_1 & \cdots & * \\ & \ddots & \vdots \\ & & c_N)\mqty(\dmat{\norm{\vb*{u}_1}^{-1}, \ddots, \norm{\vb*{u}_N}^{-1}})\tilde{P}^{-1}
= \tilde{P}\mqty(c_1 & \cdots & * \\ & \ddots & \vdots \\ & & c_N)\tilde{P}^{-1}
$$
とできる。
\end{proof}

\begin{theorem}[正規行列の対角化]
体$K$を非退化とする。
$N$次正規行列$A$に対して、固有多項式が\eqref{e:eigenfactor}と因数分解されたとすると、$A$は直交行列$P$を使って
$$
A = P\mqty(\dmat{c_1, \ddots, c_N})P^{-1}
$$
と対角化される。
さらに$K$が正規化可能な場合は$P$は正規直交行列として取れる。
\end{theorem}

証明の流れは上の補題で$A$を三角行列$T$に三角化したとすると、
$A$の正規性を継承して$T$も正規になり、正規な三角行列は対角行列しかないことを示す。
$A$が正規直交行列(ユニタリ行列)で三角化していたらこれでいいが、今回は少し違うので修正が必要である。

\begin{proof}
$A$は直交行列$P = \mqty(\vb*{v}_1 & \cdots & \vb*{v}_N)$と三角行列$T = (c_{i j})^{i = 1, \cdots, N}_{j = 1, \cdots, N}$を使って$A = P T P^{-1}$と表されて、このとき$T$は対角行列であることを示す。
ここで、$A^* A = (P^{-1})^* T^* P^* P T P^{-1}$と$A A^* = P T P^{-1} (P^{-1})^* T^* P^*$で$A$は正規なのでこの二つが等しいので、
$$
T^* P^* P T (P^* P)^{-1} = P^* P T (P^* P)^{-1} T^*.
$$
ここで、$P$は直交行列より$P^* P$は対角行列
$$
D = P^* P = \mqty(\dmat{\lrangle{\vb*{v}_1, \vb*{v}_1}, \ddots, \lrangle{\vb*{v}_N, \vb*{v}_N}})
$$
なので対角成分を$d_1, \cdots, d_N$とおくと、
$$
D T D^{-1} = (d_i c_{i j} d_j^{-1})
$$
である。
よって、$T^* D T D^{-1} = D T D^{-1} T^*$から、各$i, j = 1, \cdots, N$に対して
$$
\sum_{k = 1}^N \overline{c_{k i}} d_k c_{k j} d_j^{-1}
= \sum_{k = 1}^N d_i c_{i k} d_k^{-1} \overline{c_{j k}}.
$$
これを対角成分で$i = j$が大きい方から考えると、$T$は三角行列であることに注意して$i = j = N$の時、
$$
\overline{c_{1 N}} d_1 c_{1 N} d_N^{-1}+\cdots+\overline{c_{N N}} d_N c_{N N} d_N^{-1}
= d_N c_{N N} d_N^{-1} \overline{c_{N N}}.
$$
つまり
$$
\overline{c_{1 N}}c_{1 N} d_1+\cdots+\overline{c_{N-1 N}} c_{N-1 N} d_{N-1} = 0
$$
で、$K$が非退化である条件が使える状況になっていて、$c_{1 N} = \cdots = c_{N-1 N}$がわかる。
これを繰り返していけば$T$は対角行列にならざるを得ないことがわかり、定理の証明が完成する。
\end{proof}

\section{実対称行列の対角化}

まず、エルミート行列の固有値について調べる。

\begin{theorem}[エルミート行列の固有値]
エルミート行列$A$の全ての固有値$c$は
$$
\overline{c} = c
$$
を満たす。
\end{theorem}

\begin{proof}
固有ベクトルの一つを$\vb*{v} \ne \vb*{0}$とおくと$A\vb*{v} = c\vb*{v}$なので、
標準内積について
$$
\lrangle{\vb*{v}, A\vb*{v}} = \lrangle{\vb*{v}, c\vb*{v}} = c\lrangle{\vb*{v}, \vb*{v}}.
$$
また、$A$はエルミート行列より随伴行列$A^*$は$A$自身なので、
$$
\lrangle{\vb*{v}, A\vb*{v}} = \lrangle{A\vb*{v}, \vb*{v}} = \lrangle{c\vb*{v}, \vb*{v}} = \overline{c}\lrangle{\vb*{v}, \vb*{v}}.
$$
以上のことと$\vb*{v} \ne \vb*{0}$より$\overline{c} = c$である。
\end{proof}

この定理で出てくるスカラーの範囲を考える。
非退化な体$K$に対して
$$
\mathrm{R}(K) = \lrset{ a \in K \mid \overline{a} = a }
$$
とおき、$K$の\emph{実部}と呼ぶ。
$\mathrm{R}(K)$は$K$の加法と乗法で閉じていて、$K$のいわゆる部分体になっている。


\chapter{三角化}

\section{フロベニウスの定理とケイリー・ハミルトンの定理}

\begin{theorem}[フロベニウスの定理]
$A$を$K$上の$N$次正方行列、$f(x)$を$K$上の多項式とする。
このとき、$A$の固有値$c \in K$に対して、$f(c)$は$f(A)$の固有値である。
より詳しくは$A$の固有多項式が$c_1, \cdots, c_N \in K$を使って\eqref{e:eigenfactor}と因数分解されるならば、
$f(A)$の固有多項式は
$$
\det(x I-f(A)) = (x-f(c_1))\cdots(x-f(c_N))
$$
と因数分解される。
\end{theorem}

\begin{proof}
$c$を$A$の固有値とするとき、$A\vb*{v} = c\vb*{v}$となるベクトル$\vb*{v} \ne \vb*{0}$が取れる。
$n = 0, 1, 2, 3, \cdots$に対して、
$$
A^n\vb*{v} = c^n\vb*{v}
$$
であるからスカラー倍して和を取ることで、
$$
f(A)\vb*{v} = f(c)\vb*{v}
$$
である。
よって、$f(c)$は$f(A)$の固有値である。

$A$の固有多項式が\eqref{e:eigenfactor}と因数分解されるとき、定理\ref{t:tri}より\eqref{e:tri}と三角化される。
$n = 0, 1, 2, 3, \cdots$に対して、
$$
A^n
= P\mqty(c_1 & \cdots & * \\ & \ddots & \vdots \\ & & c_N)^n P^{-1}
= P\mqty(c_1^n & \cdots & * \\ & \ddots & \vdots \\ & & c_N^n)P^{-1}
$$
であるからスカラー倍して和を取ることで、
$$
f(A) = P\mqty(f(c_1) & \cdots & * \\ & \ddots & \vdots \\ & & f(c_N))P^{-1}
$$
である。
よって、命題\ref{t:eigenfactor}より、この定理の結論が従う。
\end{proof}

\begin{theorem}[ケイリー・ハミルトンの定理]
$A$を$K$上の$N$次正方行列として、固有多項式$f_A(x) = \det(x I-A)$が\eqref{e:eigenfactor}と因数分解されたとする。
このとき、$f_A(A) = O$が成り立つ。
\end{theorem}

\begin{remark}
この定理は$\det(x I-A)$の$x$に$A$を代入できると拡大解釈すれば$\det(A I-A) = \det O = 0$となり正しそうであるが、
実際の証明はしっかり$f_A(A)$の定義に則って行う必要がある。
\end{remark}

\begin{proof}
まず$A$が右上三角行列
$$
T = \mqty(c_1 & \cdots & * \\ & \ddots & \vdots \\ & & c_N)
$$
の場合に示す。
このとき固有多項式は
$$
f_T(x) = (x-c_1)\cdots(x-c_N)
$$
であり、
$$
f_T(T) = (T-c_1 I_N)\cdots(T-c_N I_N)
$$
が成り立つ。
あとはこれが零行列であることを$N$についての数学的帰納法で示す。
$N = 1$の時は$T = \mqty(c_1)$なので$T-c_1 I_1 = O_1$である。
$N-1$で成立する時、
$$
(T-c_1 I_N)\cdots(T-c_{N-1} I_N) = \mqty(O_{N-1} & * \\ \vb*{0}_{N-1} & *),
\quad (T-c_N I_N) = \mqty(* & * \\ \vb*{0}_{N-1} & 0)
$$
なので、積を取ると零行列になる。
以上より$f_T(T) = O$である。
一般の$A$に対しては定理\ref{t:tri}より、$A$は\eqref{e:tri}と三角化されて右上三角行列を$T$とおくと、
$f_A(x) = f_T(x) = (x-c_1)\cdots(x-c_N)$で
$$
f_A(A) = P f_A(T) P^{-1}
$$
より、$f_A(A) = O$がわかる。
\end{proof}

以上の二つの定理を使えば例えば以下のことがわかる。

\begin{proposition}[べき零行列]
代数的閉体$K$上の$N$次正方行列$A$がある$n = 1, 2, 3, \cdots$で
$$
A^n = O
$$
を満たす時、$A^N = O$が成り立つ。
\end{proposition}

このような行列のことを\emph{べき零行列}という。

\begin{proof}
$f(x) = x^n$とおいてフロベニウスの定理を用いると、$A$の固有値を$c \in K$とおくと$f(c) = c^n$は$f(A) = A^n$の固有値である。
ここで$A^n = O$の固有値は$0$しかないので、$c^n = 0$がわかり$A$の固有値は全て$0$であることがわかる。
よって、$A$の固有多項式は$f_A(x) = x^N$であり、ケイリー・ハミルトンの定理より$f_A(A) = A^N = O$が従う。
\end{proof}

\section{広義固有空間}

固有多項式の零点の重複度(代数的重複度)と固有空間の次数(幾何学的重複度)には差がある可能性があるのであった。
その差を埋めるにはどうすればいいだろうか。
そのためのアイデアが固有空間を拡張した広義固有空間である。

\begin{definition}[広義固有空間]
$A$を$K$上の正方行列として、$x \in K$に対して
$$
\tilde{W}(x) = \lrset{ \vb*{v} \in K^N \mid (x I-A)^n\vb*{v} = \vb*{0}, n = 0, 1, 2, 3, \cdots }
$$
を定め、固有値$x = c$に対して$\tilde{W}(c)$を$A$の固有値$c$に対する\emph{広義固有空間}という。
\end{definition}

$\tilde{W}(x)$は$K^N$の線形部分空間である。

\begin{lemma}[広義固有空間の線形独立性]
$T$を$K$上の線形空間$V$上の線形変換として、$c, d \in K$を異なる固有値とする。
このとき広義固有空間について$\tilde{W}(c)\cap \tilde{W}(d) = O_V$が成り立つ。
\end{lemma}

\begin{proof}
\end{proof}

\section{ジョルダン標準形}


\chapter{種々の行列}

\section{ベクトルのテンソル積}

$\vb*{a}$と$\vb*{b}$をそれぞれ$M$次と$N$次のベクトルとする時、
$\vb*{a}$を$M$次の縦ベクトル、$\vb*{b}^T$を$N$次の横ベクトルとみなして行列の積$\vb*{a}\vb*{b}^T$は$M\times N$型の行列であり、ベクトル$\vb*{a}$と$\vb*{b}$の\emph{テンソル積}と呼ばれ$\vb*{a}\otimes\vb*{b}$とも書かれる。
成分を使って書けば
$$
\vb*{a}\otimes\vb*{b}
= \vb*{a}\vb*{b}^T
= \begin{pmatrix}a_1 \\ \vdots \\ a_M\end{pmatrix}\begin{pmatrix}b_1 & \cdots & b_N\end{pmatrix}
=
\begin{pmatrix}
a_1 b_1 & \cdots & a_1 b_N \\
\vdots  & \ddots & \vdots  \\
a_M b_1 & \cdots & a_M b_N \\
\end{pmatrix}
$$
である。

また、$\vb*{a}$と$\vb*{b}$を$N$次のベクトルとする時、
$\vb*{a}^T$を横ベクトル、$\vb*{b}$を縦ベクトルとみなして行列の積$\vb*{a}^T\vb*{b}$はスカラーであり、ベクトル$\vb*{a}$と$\vb*{b}$の\emph{スカラー積}と呼ばれ$\vb*{a}\cdot\vb*{b}$とも書かれる。
成分を使って書けば
$$
\vb*{a}\cdot\vb*{b}
= \vb*{a}^T\vb*{b}
= \begin{pmatrix}a_1 & \cdots & a_N\end{pmatrix}\begin{pmatrix}b_1 \\ \vdots \\ b_N\end{pmatrix}
= a_1 b_1+\cdots+a_N b_N
$$
である。
行列の積には交換法則は一般には成り立たないが、スカラー積に対しては交換法則が成り立つことに注意する。
つまり、任意の$N$次のベクトル$\vb*{a}$と$\vb*{b}$に対して、
$$
\vb*{a}\cdot\vb*{b} = \vb*{b}\cdot\vb*{a}
$$
が成り立つ。

ここで、$\vb*{a}$と$\vb*{b}$を$N$次ベクトルとする時のテンソル積である$N$次の正方行列$A = \vb*{a}\otimes\vb*{b}$の$n$乗を考える。
$2$乗を計算すると、
$$
A^2 = (\vb*{a}\vb*{b}^T)(\vb*{a}\vb*{b}^T) = \vb*{a}(\vb*{b}^T\vb*{a})\vb*{b}^T = (\vb*{b}\cdot\vb*{a})A = (\vb*{a}\cdot\vb*{b})A
$$
とスカラー積倍になる。
よってこれを繰り返し用いることで、
$$
A^n = (\vb*{a}\cdot\vb*{b})^{n-1}A
\quad (n = 1, 2, 3, \cdots)
$$
を得る。

\section{巡回行列}


\chapter{二次式}

\section{二次形式}

\section{二次式と平方完成}

\begin{theorem}[平方完成]
$A$を$N$次対称正則行列、$\vb*{b}, \vb*{x}$を$N$次ベクトル、$c$をスカラーとする時、
$$
\vb*{x}\cdot A\vb*{x}+2\vb*{b}\cdot\vb*{x}+c
= (\vb*{x}+A^{-1}\vb*{b})\cdot A(\vb*{x}+A^{-1}\vb*{b})+c-A^{-1}\vb*{b}\cdot\vb*{b}
$$
が成り立つ。
\end{theorem}

\begin{example}
3つの実数$u_1, u_2, u_3$に対して
$$
E(u_1, u_2, u_3)
= (u_1-u_2)^2+(u_2-u_3)^2+(u_3-u_1)^2
= 2 u_1^2+2 u_2^2+2 u_3^2-2 u_1 u_2-2 u_2 u_3-2 u_3 u_1
$$
とおく、このとき6つの実数$u_1, u_2, u_3, v_1, v_2, v_3$に対して、
$$
E(u_1, v_2, v_3)+E(v_1, u_2, v_3)+E(v_1, v_2, u_3) \ge \frac{3}{5}E(u_1, u_2, u_3)
$$
が成り立つことを示す。
左辺を$E$とおいて$v_1, v_2, v_3$についての二次式とみなすことで
$$
E
= \begin{pmatrix}v_1 \\ v_2 \\ v_3\end{pmatrix}^T\begin{pmatrix}4 & -1 & -1 \\ -1 & 4 & -1 \\ -1 & -1 & 4\end{pmatrix}\begin{pmatrix}v_1 \\ v_2 \\ v_3\end{pmatrix}-2\begin{pmatrix}u_2+u_3 \\ u_3+u_1 \\ u_1+u_2\end{pmatrix}^T\begin{pmatrix}v_1 \\ v_2 \\ v_3\end{pmatrix}+2(u_1^2+u_2^2+u_3^2).
$$
ここで$A = \begin{pmatrix}4 & -1 & -1 \\ -1 & 4 & -1 \\ -1 & -1 & 4\end{pmatrix}$の逆行列の計算が必要になり、掃き出し法を実行してもよいが、ここでは以下のように考えてみよう。
つまり、$X = \begin{pmatrix}1 & 1 & 1 \\ 1 & 1 & 1 \\ 1 & 1 & 1\end{pmatrix}$とおくと、$A = 5 I-X$で$X^2 = 3 X$なので$A^2-7 A+10 I = O$、つまり
$$
A^{-1} = -\frac{1}{10}(A-7 I) = \frac{1}{10}(2 I+X)
$$
がわかる。
また、$A$は正定値であることに注意する。
ここで$\vb*{u} = \begin{pmatrix}u_1 \\ u_2 \\ u_3\end{pmatrix}$, $\vb*{v} = \begin{pmatrix}v_1 \\ v_2 \\ v_3\end{pmatrix}$とおくと、
$$
E = \vb*{v}\cdot A\vb*{v}-2(X-I)\vb*{u}\cdot\vb*{v}+2|\vb*{u}|^2.
$$
平方完成して$A$は正定値であることから、
$$
E \ge 2|\vb*{u}|^2-(X-I)\vb*{u}\cdot A^{-1}(X-I)\vb*{u}.
$$
ここで
$$
(X-I)A^{-1}(X-I) = \frac{1}{10}(X-I)(2 I+X)(X-I) = \frac{1}{5}(3 X+I)
$$
なので、
$$
E = E(u_1, v_2, v_3)+E(v_1, u_2, v_3)+E(v_1, v_2, u_3)
\ge \vb*{u}\cdot \frac{1}{5}(9 I-3 X)\vb*{u} = \frac{3}{5}E(u_1, u_2, u_3)
$$
である。
\end{example}



% \bibliographystyle{amsplain}
% \bibliography{references}

\begin{thebibliography}{10}

\bibitem[N]{N}
中安淳、
微分積分学1、
2022年。
https://ankys.github.io/notes/calc1t.pdf

\end{thebibliography}


\printindex

\end{document}
